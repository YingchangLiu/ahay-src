
\def\WW{\mathbf W}
\def\WWK{\widetilde{\mathbf W}}
\def\WK{\widetilde W}
\def\WP{\widetilde P}
\def\WqP{\widetilde {{\it q}P}}


\def\DelK
{
{\widetilde\nabla}
}

\def\UU{\mathbf U}
\def\kk{\mathbf k}

\def\SS{\mathbf S}
% ------------------------------------------------------------
\section{Separation method}
%\subsection{Isotropic media}
Separation of scalar and vector potentials can be achieved by
Helmholtz decomposition, which is applicable to any vector field
$\WW(x,y,z)$. By definition, the vector wavefield $\WW$ can be
decomposed into a curl-free scalar potential $\Theta$ and a
divergence-free vector potential $\boldsymbol \Psi$ according to the
relation~{\cite[]{akirichards.2002}}:
\beq  \label{Helmholtz}
\WW =
\nabla  \Theta +
\nabla \times {\boldsymbol \Psi} \, .
\eeq
\rEq{Helmholtz} is not used directly in practice, but the scalar and 
vector components are obtained indirectly by the application of the
$\DIV{}$ and $\CURL{}$ operators to the extrapolated elastic
wavefield:
\beqa \label{PandS}
\label{P}     P &=& \DIV \WW \, , \\
\label{S}   \SS &=& \CURL\WW \, . 
\eeqa
For isotropic elastic fields far from the source, quantities $P$ and
$\SS$ describe compressional and {shear wave modes},
respectively~\cite[]{akirichards.2002}.

\rEqs{P} and \ren{S} allow one to understand why $\DIV{}$ and $\CURL{}$ 
pass compressional and transverse wave modes, respectively. In the
{discretized} space domain, one can write:
%
\beq\label{IsoDivX}
P= 
\DIV \WW = D_x[W_x]+D_y[W_y]+D_z[W_z]\, ,
\eeq
%
where $D_x$, $D_y$, and $D_z$ represent spatial derivatives in the
$x$, $y$, and $z$ directions, respectively. Applying derivatives in
the space domain is equivalent to applying finite difference filtering
to the functions. Here, $D\lb\cdot\rb$ represents spatial filtering of
the wavefield with finite difference operators. In the Fourier domain,
one can represent the operators $D_x$, $D_y$, and $D_z$ by $i\, k_x$,
$i\, k_y$, and $i\, k_z$, respectively; therefore, one can write an
equivalent expression to \req{IsoDivX} as:
%%%%%
\def\WP{\widetilde{\mathbf P}}
\beqa\label{IsoDivK}
\WP=i\, \kk \cdot \WWK= i\, k_x\,\WK_x+i\, k_y\,\WK_y+i\, k_z\,\WK_z   \, ,
\eeqa
%
where $\kk=\{k_x,k_y,k_z\}$ represents the wave vector, and
$\WWK(k_x,k_y,k_z)$ is the 3D Fourier transform of the wavefield
$\WW(x,y,z)$. We see that in this domain, the operator $i\, \kk$
essentially projects the wavefield $\WWK$ onto the wave vector $\kk$,
which represents the polarization direction for P waves. Similarly,
the operator $\CURL{}$ projects the wavefield onto the direction
orthogonal to the wave vector $\kk$, which represents the polarization
direction for S waves~\cite[]{GEO55-07-09140919}. For illustration,
\rFg{Iso-polarvector} shows the polarization vectors of the P mode of
a 2D isotropic model as a function of normalized $k_x$ and $k_z$ ranging
from $-1$ to $1$ cycles. The polarization vectors are radial because the P
waves in an isotropic medium are polarized in the same directions as
the wave vectors.


%%%%%%%%%%%%%%%%%%%%%%%%%%%%%%%%%%%%%%%%%%%%%%%%%%%%%%%%%%%%%%
\inputdir{Matlab}
\multiplot{2}{Iso-polarvector,Ani-polarvector}{width=.45\textwidth}
{The {\it q}P and {\it q}S polarization vectors as a function of
normalized wavenumbers $k_x$ and $k_z$ ranging from $-1$ to $+1$
cycles, for (a) an isotropic model with $V_P=3$~km/s and
$V_S=1.5$~km/s, and (b) an anisotropic (VTI) model with
$V_{P0}=3$~km/s, $V_{S0}=1.5$~km/s, $\epsilon=0.25$ and
$\delta=-0.29$. The red arrows are the {\it q}P wave polarization vectors,
and the blue arrows are the {\it q}S wave polarization vectors.}




%\subsection{Homogeneous anisotropic media}
\cite{GEO55-07-09140919} suggest the idea that wave mode separation 
can be extended to anisotropic media by projecting the wavefields onto
the directions in which the P and S modes are polarized. This requires
that one should modify the wave separation \req{IsoDivK} by projecting the
wavefields onto the true polarization directions {\textbf U} to obtain
{\it quasi-}P ({\it q}P) waves:
\beq\label{AniDivK}
\WqP=i\, \UU(\kk) \cdot \WWK 
=i\, U_x\,\WK_x+i\, U_y\,\WK_y+i\, U_z\,\WK_z\, .                
\eeq
In anisotropic media, $\UU(k_x,k_y,k_z)$ is different from $\kk$, as
illustrated in \rfg{Ani-polarvector}, which shows the polarization
vectors of {\it q}P wave mode for a 2D VTI anisotropic model with normalized
$k_x$ and $k_z$ ranging from $-1$ to $1$ cycles. Polarization vectors are
not radial because {\it q}P waves in an anisotropic medium are not
polarized in the same directions as wave vectors, except in the
symmetry planes ($k_z=0$) and along the symmetry axis ($k_x=0$).

\cite{GEO55-07-09140919} demonstrate wave mode separation in 
the wavenumber domain using projection of the polarization vectors,
as indicated in \req{AniDivK}. However, for heterogeneous media, this
equation is defective because the polarization vectors are spatially
varying. One can write an equivalent expression to \req{AniDivK} in
the space domain for each grid point as:
\beq\label{AniDivX}
{\it q}P=\nabla_a\cdot \WW     = L_x[W_x] 
                      + L_y[W_y] 
                      + L_z[W_z] \, ,
\eeq
where $L_x$, $L_y$, and $L_z$ represent the inverse Fourier transforms
of $i\, U_x$, $i\, U_y$, and $i\, U_z$, respectively.
$L\left[\cdot\right]$ represents spatial {filtering of the wavefield
with anisotropic separators}.  $L_x$, $L_y$, and $L_z$ define the
pseudo derivative operators in the $x$, $y$, and $z$ directions for an
anisotropic medium, respectively, and they change from location to
location according to the material parameters.

We obtain the polarization vectors $\UU(\kk)$ by solving the
Christoffel equation ~\cite[]{akirichards.2002,Tsvankin}:
\beq\label{3dChristoffel}
\lb {\bf G} - \rho V^2 {\bf I} \rb \UU = 0 \, ,
\eeq
where {\textbf G} is the Christoffel matrix $G_{ij}=c_{ijkl}n_jn_l$,
in which $c_{ijkl}$ is the stiffness tensor, $n_j$ and $n_l$ are the
normalized wave vector components in the $j$ and $l$ directions,
$i,j,k,l=1,2,3$. The parameter $V$ corresponds to the eigenvalues of
the matrix {\textbf G}. The eigenvalues $V$ represent the phase
velocities of different wave modes and are functions of the wave vector
$\kk$ (corresponding to $n_j$ and $n_l$ in the matrix {\textbf G}).
{For plane waves propagating in any symmetry planes of a VTI medium, 
one can set $k_y$ to $0$ and get}
\def\c11{c_{11}}
\def\c55{c_{55}}
\def\c13{c_{13}}
\def\c33{c_{33}}
\beq\label{VtiChristoffel}
\lb 
 \mtrx{ c_{11} k_x^2 +c_{55} k_z^2 -\rho V^2 &0& \lp c_{13}+c_{55}\rp
k_xk_z\\
0 &c_{66}k_x^2+c_{55}k_z^2-\rho V^2&0\\
\lp c_{13}+c_{55}\rp k_xk_z  & 0&    c_{55} k_x^2 +c_{33} k_z^2 -\rho V^2
 }
\rb
\lb\mtrx{ U_x\\U_y\\U_z} \rb
=0 \, .
\eeq
{The middle row of this matrix characterizes the SH wave 
polarized in the $y$ direction, and {\it q}P and {\it q}SV modes are
uncoupled from the SH mode and are polarized in the vertical
plane. The top and bottom rows of this equation} 
allow one to compute \old{the components of} the polarization vector
{$\UU=\{U_x,U_z\}$}(the eigenvectors of the 
matrix \old{{\textbf G}}) of {P or SV} wave mode
given the stiffness tensor at every location of the medium.

%\subsection{Heterogeneous anisotropic media}
%Non-stationary filter
One can extend the procedure described here to heterogeneous media by
computing two different operator for each mode at every grid point. In the symmetry
planes of VTI media, the operators are 2D and depend on the local
values of the stiffness coefficients. For each point, I pre-compute
the polarization vectors as a function of the local medium parameters,
and transform them to the space domain to obtain the wave mode
separators. {I assume that the medium parameters vary smoothly
(locally homogeneous), but even for complex media, the localized
operators work in the same way as the long finite difference operators.}
If one represents the stiffness coefficients using Thomsen
parameters~\cite[]{GEO51-10-19541966}, then the pseudo derivative
operators $L_x$ and $L_z$ depend on $\epsilon$, $\delta$, $V_{P0}$ and
$V_{S0}$, which {can be} spatially varying
parameters. One can compute and store the operators for all grid points
in the medium, and then use these operators to separate P and S modes
from reconstructed elastic wavefields at different time steps.  Thus,
wavefield separation in VTI media can be achieved simply by
non-stationary filtering with operators $L_x$ and $L_z$.

%%%%%%%%%%%%%%%%%%%%%%%%%%%%%%%%%%%%%%%%%%%%%%%%%%%%%%%%%%%%%%
%%%%%%%%%%%%%%%%%%%%%%%%%%%%%%%%%%%%%%%%%%%%%%%%%%%%%%%%%%%%%%
\section{Operator properties}
In this section, I discuss the properties of the anisotropic ``derivative''
operators, including order of accuracy, size, and compactness.

\subsection{Operator orders}
As I have showed in the previous section, the isotropic separation
operators (divergence and curl) in \reqs{IsoDivX} and \ren{IsoDivK} are
exact in the $x$ and $k$ domains. 
{The exact derivative operators are infinitely long
series in the \emph{discretized} space domain.} In practice, when
evaluating the derivatives numerically, one needs to take some
approximations to make the operators short and computationally
efficient. Usually, difference operators are evaluated at different
orders of accuracy. The higher order the approximation is, the more
accurate and longer the operator becomes. For example, the $2^{nd}$
order operator has coefficients $(-\frac{1}{2},+\frac{1}{2})$, and the
more accurate $4^{th}$ order operator has coefficients
$(+\frac{1}{12},-\frac{2}{3},\frac{2}{3},-\frac{1}{12})$ \cite[]{Fornberg:1999}.

In the wavenumber domain, for isotropic media, as shown by the black
line in \rFg{operator}(b), the exact difference operator is $ik$.
Appendix A shows the $k$ domain equivalents of the $2^{nd}$, $4^{th}$,
$6^{th}$, and $8^{th}$ order finite difference operators, and they are
plotted in \rFg{operator}(b). The higher order operators have
responses closer to the exact operator $ik$ (black line). To obtain
vertical and horizontal derivatives of different orders of accuracy,
I weight the polarization vector $i\kk$ components $ik_x$ and $ik_z$
by the weights shown in \rFg{operator}(c). For VTI media, similarly,
I weight the anisotropic polarization vector $i\UU(\kk)$ components
$iU_x$ and $iU_z$ by these same weights. The weighted vectors are then
transformed back to space domain to obtain the anisotropic stencils.




% ------------------------------------------------------------
\inputdir{Matlab} 
\plot{operator}{width=\textwidth}{Comparison of
  derivative operators of different orders of accuracy ($2^{nd}$,
  $4^{th}$, $6^{th}$, and $8^{th}$ orders in space, as well as the
  approximation applied in {\cite{GEO55-07-09140919}--cosine taper})
  in both (a) the $x$ domain and (b) the $k$ domain. (c) Weights to
  apply to the components of the polarization vectors.}


\inputdir{oporder}
\multiplot{2}{iop2,mop2,iop4,mop4,iop6,mop6,iop8,mop8}
{width=.45\textwidth} { $2^{nd}$, $4^{th}$, $6^{th}$, and $8^{th}$
order derivative operators for an isotropic medium ($V_P=3$~km/s and
$V_S=1.5$~km/s) and a VTI medium ($V_{P0}=3$~km/s, $V_{S0}=1.5$~km/s,
$\epsilon=0.25$ and $\delta=-0.29$). The left column includes
isotropic operators, and the right column includes anisotropic
operators.  From top to bottom are operators with increasing orders of
accuracy.}


\subsection{Operator size and compactness}
%operator size vs. order
\rFg{iop2,mop2,iop4,mop4,iop6,mop6,iop8,mop8} shows the derivative
operators of $2^{nd}$, $4^{th}$, $6^{th}$, and $8^{th}$ orders in the
$z$ and $x$ directions for isotropic and VTI ($\epsilon$=0.25,
$\delta$=-0.29) media. As we can see,
isotropic operators become longer when the order of accuracy is
higher. Anisotropic operators, however, do not change much in size. One
can see that the central parts of the anisotropic operators look similar
to their corresponding isotropic operators and change with the order
of accuracy; while the outer parts of these anisotropic operators all
look similar, and do not change much with the order of accuracy. This
indicates that the central parts of the operators are determined by
the order of accuracy, while the outer parts are representation of the
degree of anisotropy.

%operator size vs. ansotropy
\rFg{aniopsize-mop0-order8,mop1-order8,mop2-order8} 
shows anisotropic derivative operators with 
same order of accuracy ($8^{th}$ order in space) for three VTI media
with different combinations of $\epsilon$ and $\delta$. These
operators have similar central parts, but different outer parts.  This
result is consistent with the previous observation that the central
part of an operator is determined by the order of accuracy, and the
outer part is controlled by anisotropy parameters. 

%{\rFg{aniopsize-diff0} shows the influence of
%approximation to finite difference ($2^{nd}$ and $8^{th}$
%order, \rFgs{oporder-mop8} and \subrfn{oporder-mop2}). The ``anisotropic''
%part (``diagonal tails'') are almost the same, and the difference comes from
%the central part.  \rFg{aniopsize-diff1} shows the difference
%between operators with different anisotropy (\rFgs{aniopsize-mop0-order8}
%and \subrfn{aniopsize-mop1-order8}). The difference mainly lies in the ``tails''
%of the operators.  }

A comparison between \rFgs{aniopsize-mop0-order8}
and \subrfn{aniopsize-mop1-order8} shows that when one has large
difference between $\epsilon$ and $\delta$, the operator is big in
size and when the difference of $\epsilon$ and $\delta$ stays the
same, the parameter $\delta$ affects the operator size.  A comparison
between \rFgs{aniopsize-mop1-order8} and \subrfn{aniopsize-mop2-order8}
shows that when the difference between $\epsilon$ and $\delta$ becomes
smaller and $\delta$ does not change, the operator get smaller in
size. This result is consistent with the polarization equation for VTI
media with weak anisotropy~\cite[]{Tsvankin}:
\beq
\nu_P=\theta+B\lb\delta+2\lp \epsilon-\delta \rp\sin^2\theta\rb\sin 2\theta \, ,
\eeq
where
\[ B\equiv \frac{1}{2f}=\frac{1}{2\lp1-V_{S0}^2/V_{P0}^2 \rp}.  \]
$V_{P0}$ and $V_{S0}$ are vertical P and S wave velocities, $\theta$
is the phase angle, and $\nu_p$ is the P wave polarization angle. This
equation demonstrates the deviation of anisotropic polarization
vectors with the isotropic ones: difference of $\epsilon$ and $\delta$
(which is approximately $\eta$ for weak anisotropy) and the parameter
$\delta$ control the deviation of $\nu_P$ from $\theta$ and therefore
the size of the anisotropic derivative operators. 
%The parameter $\eta$is more influential than $\delta$ to the size of
%the operators because it is doubled in the equation.

%figure4
\inputdir{aniopsize} \multiplot{3}{mop0-order8,mop1-order8,mop2-order8}
{width=.6\textwidth}
{$8^{th}$ order anisotropic pseudo derivative operators for three VTI
  media}

%figure5
%\multiplot{2}{diff,comp}{width=.5\textwidth}
\multiplot{2}{diff0,diff1}{width=.8\textwidth}{}

%\inputdir{oporder}
%\plot{diff}{width=8cm}{}
%\inputdir{aniopsize}
%\plot{diff}{width=8cm}{}


\subsection{Operator truncation}

The derivative operators for isotropic and anisotropic media are very
different in both shape and size {, and the operators vary with
the strength of anisotropy}. In theory, analytic isotropic derivatives
are point operators {in the \emph{continuous} limit}. If one can do
perfect Fourier transform to $ik_x$ and $ik_z$ (without doing the
approximations to different orders of accuracy as one does
in \rFg{operator}), one gets point derivative operators. This is because
$ik_x$ is constant in the $z$ direction (see \rFg{IsoU}), whose
Fourier transform is delta function; the exact expression of $ik_x$
in the $k$ domain also makes the operator point in the $x$
direction. This makes the isotropic derivative operators point
operators in the $x$ and $z$ direction. And when one applies
approximations to the operators, they are compact in the space domain.

However, even if one does perfect Fourier transformation to $iU_x$ and
$iU_z$ (without doing the approximations for different orders of
accuracy) for VTI media, the operators will not be point operators
because $iU_x$ and $iU_z$ are not constants in $z$ and $x$ directions,
respectively (see \rFg{AniU}). The $x$ domain operators spread out in
all directions (\rFgs{mop2}, \subrfn{mop4}, \subrfn{mop6},
and \subrfn{mop8}).

\inputdir{Matlab} \multiplot{2}{IsoU,AniU}{width=\textwidth} {(a)
  Isotropic and (b) VTI ($\epsilon=0.25$, $\delta=-0.29$) polarization
  vectors (\rFg{Matlab-operator}) projected on to the $x$ (left
  column) and $z$ directions (right column). {The isotropic
  polarization vectors components in the $z$ and $x$ directions depend
  only on $k_z$ and $k_x$, respectively. In contrast, the anisotropic
  polarization vectors components are functions of both $k_x$ and
  $k_z$.}}

This effect is illustrated by
\rFg{iop2,mop2,iop4,mop4,iop6,mop6,iop8,mop8}. When the order of
accuracy decreases, the isotropic operators become more compact
(shorter in space), while the anisotropic operators do not get more
compact. No matter how one improves the compactness of isotropic
operators, one does not get compact {\it anisotropic} operators in the
space domain by the same means.



Because the size of the anisotropic derivative operators is usually
large, it is natural that one would truncate the operators to save
computation.  \rFg{uA,mop5} shows a snapshot of an elastic wavefield
and corresponding derivative operators for a VTI medium with
$\epsilon=0.25$ and $\delta=-0.29$.  \rFg{pA1,pA3,pA5} shows the
attempt of separation using truncated operator size of (a)
$11\times11$, (b) $31\times31$ and (c) $51\times51$ out of the full
operator size $65\times65$. \rFg{pA1,pA3,pA5} shows that the truncation
causes the wave-modes incompletely separated. This is because the
truncation changes the directions of the polarization vectors, thus
projecting the wavefield displacements onto wrong directions.
\rFg{truncate1,truncate2,truncate3} presents
the P-wave polarization vectors before and after the truncation.  For
a truncated operator size of $11\times11$, the polarization vectors
deviate from the correct ones to a maximum of 10$^\circ$, but
even this difference makes the separation incomplete.



\inputdir{separate2}
%\multiplot{3}{uA,qA,pA}{width=0.75\textwidth}
%{(a) Anisotropic wavefield, (b) anisotropic potentials by $\DIV{}$ and
%$\CURL{}$ and (c) anisotropic potentials by pseudo derivative
%operators.}
% ------------------------------------------------------------
%\multiplot{2}{uA,mop5}{width=0.75\textwidth} {(a) A snapshot of an
%  elastic wavefield showing the vertical (left) and horizontal (right)
%  components for a VTI medium ($\epsilon=0.25$ and
%  $\delta=-0.29$). (b) $8^{th}$ order anisotropic pseudo derivative
%  operators in $z$ (left) and $x$ (right) direction for this VTI
%  medium. The boxes show the truncation of the operator to sizes of
%  $11\times11$, $31\times31$, and $51\times51$.}

%\multiplot{3}{pA1,pA3,pA5}{width=0.6\textwidth}
%{separation by $8^{th}$ order anisotropic pseudo derivative operators
%of different sizes: (a) $11\times11$, (b) $31\times31$, (c)
%$51\times51$, shown in {\rFg{separate2-mop5}}. {The plot shows the
%larger the size of the operators, the better the separation is.}}




\inputdir{Matlab}


\multiplot{3}{truncate1,truncate2,truncate3}{height=.25\textheight}
{The deviation of polarization vectors by truncating the size of the
  space-domain operator to (a) $11\times11$, (b) $31\times31$, (c)
  $51\times51$ out of $65\times65$. The left column shows polarization
  vectors from ${-1}$ to ${+1}$ cycles in both $x$ and $z$
  directions, and the right column zooms to $0.3$ to $0.7$ cycles. The
  green vectors are the exact polarization vectors, and the red ones
  are the effective polarization vectors after truncation of the
  operator in the $x$ domain.}




