\published{Journal of Geophysics and Engineering, 19, 14-27, (2022)}

\title{Nonstationary pattern-based signal-noise separation using adaptive prediction-error filter}

\renewcommand{\thefootnote}{\fnsymbol{footnote}}

\ms{JGE-2021-0127}

\address{
    \footnotemark[1] College of Geo-exploration Science and Technology,\\
    Jilin University, Changchun, China}

\author{Zhisheng Zheng\footnotemark[1], Yang Liu\footnotemark[1], Cai Liu\footnotemark[1]}

\footer{JGE-2021-0127}
\lefthead{Zheng et al.}
\righthead{Pattern-based signal-noise separation}

\maketitle


\begin{abstract}
    Complex field conditions always create different interferences during seismic data acquisition,
    and there exist several types of noise in the recorded data, which affect the subsequent data
    processing and interpretation. To separate an effective signal from the noisy data, we adopted
    a pattern-based method with a two-step strategy, which involves two adaptive prediction-error
    filters (APEFs) corresponding to a nonstationary data pattern and noise pattern. By introducing
    shaping regularization, we first constructed a least-squares problem to estimate the filter
    coefficients of the APEF. Then, we solved another constrained least-square problem corresponding
    to the pattern-based signal-noise separation, and different pattern operators are adopted to
    characterize random noise and ground-roll noise. In comparison with traditional denoising methods,
    such as FXDECON, curvelet transform and local time-frequency (LTF) decomposition, we examined
    the ability of the proposed method by removing seismic random noise and ground-roll noise in
    several examples. Synthetic models and field data demonstrate the validity of the strategy
    for separating nonstationary signal and noise with different patterns.

\end{abstract}

\section{Introduction}
During seismic data acquisition, complex field conditions unavoidably lead to several
interferences. For instance, wind motion and electronic noise from geophones cause random
noise, propagation of seismic waves on land near the surface leads to ground-roll noise.
Meanwhile, it is natural that geologic events and geophysical data exhibit nonstationarity,
which is manifested in several intrinsic properties, e.g. spectra and statistical
characteristics, and change with time and space. To reduce the influences of noise and
improve the quality of useful signal for subsequent data processing and interpretation,
the signal-noise separation is always an important procedure, especially in new
acquisition techniques such as distributed acoustic sensing.

Many effective methods have been proposed for eliminating seismic random noise besides
mean and median filters tailored to images. According to the different properties of
target signals, several transform methods can truncate the energy in the transform
domain to suppress random noise, such as radon transform~\cite[]{Claerbout71,Trad02},
curvelet transform~\cite[]{Kumar09} and seislet transform~\cite[]{Fomel10,Liu15}.
Prediction filters are effective methods for random noise attenuation, which was first
introduced by~\cite{Canales84}. The prediction process can be achieved in the time-space
domain or the frequency-space domain~\cite[]{Wang99}. \cite{Liu18} proposed a fast-streaming
prediction filter to attenuate random noise. Recently, the deep-learning method has
also been found to cope with the denoising procedure; \cite{Yu19} developed a deep-learning
method for random noise attenuation by using a convolutional neural network.

Ground-roll noise is a strong coherent noise, which will interfere with land seismic
surveys. Frequency-based filtering and F-K filtering are the classical methods to
handle ground- roll noise problem. Wavelet transform~\cite[]{Miao98,Corso03}
and curvelet transform~\cite[]{Naghizadeh18} are practical methods for ground-roll
removal as well. Time-frequency analysis can separate different components by using
a muting filter in the time-frequency domain, \cite{Liu13} applied this method to
separate the ground-roll noise from the field data. Using a generator and discriminator,
a generative adversarial network (GAN) can be trained to produce a noise-free result,
\cite{Yuan20} developed GAN for ground-roll noise removal.

A pattern-based method is often used for signal-noise separation. \cite{Spitz99}
described a multiple subtraction technology by using a pattern recognition method.
\cite{Bednar99} made a comparison of a pattern-based multiple suppression method
and other approaches. \cite{Brown00} used a pattern-based method with nonstationary
prediction-error filter (PEF) for ground-roll removal. \cite{Guitton06} further
implemented such a pattern-based approach to separate multiples in a 3D dataset.

In this paper, we revisit the fundamental theory behind the pattern-based signal-noise
separation method~\cite[]{Claerbout10} and extend this with an adaptive prediction-error
filter (APEF). The pattern-based method~\cite[]{Brown00,Claerbout00} estimates operators
to characterize data patterns, and then solves the signal-noise separation problem.
Similarly, we exploited the pattern-based method but calculated the APEF~\cite[]{Liu11}
to represent the pattern of nonstationary data and the noise model. Different algorithms
were adopted to deal with random noise and ground-roll noise, and we tested the
nonstationary characteristics of the proposed method in synthetic models and field data
examples. Results of random noise attenuation and ground-roll removal on these examples
demonstrated that the pattern-based signal-noise separation with the nonstationary APEF
is effective in separating noise from signal.

\section{Theory}
\subsection{Signal and noise separation problem}

A simple assumption is that the dataset $\mathbf{d}$ can be considered as the
summation of signal $\mathbf{s}$ and noise $\mathbf{n}$:
\begin{equation}
    \label{eq:signoi1}
    \mathbf{s} + \mathbf{n} =  \mathbf{d} .
\end{equation}
Usually, signal $\mathbf{s}$ represents reflection events or geologic events,
and noise $\mathbf{n}$ can be regarded as random noise or ground-roll noise.
Let operators $\mathbf{S}$ and $\mathbf{N}$ denote the patterns of signal and
noise, respectively. PEF is reasonable to be the pattern operator as it
approximates the inverse energy spectrum of the corresponding component.
\cite{Claerbout10} described the pattern operator as the absorbing operator,
for instance, operator $\mathbf{N}$ may absorb noise $\mathbf{n}$
($\mathbf{0} \approx \mathbf{Nn}$). And it can destroy the corresponding
noise component from data $\mathbf{d}$ and leave signal $\mathbf{s}$:
\begin{equation}
    \label{eq:Nsn}
    \begin{aligned}
        \mathbf{0}
         & = \mathbf{N} ( \mathbf{s} + \mathbf{n} - \mathbf{d} ) \\
         & = \mathbf{N} ( \mathbf{s} + \mathbf{n}) - \mathbf{Nd} \\
         & \approx \mathbf{Ns} - \mathbf{Nd}\; .
    \end{aligned}
\end{equation}
Meanwhile, operator $\mathbf{S}$ absorbs or destroys signal component
$\mathbf{s}$:
\begin{equation}
    \label{eq:S}
    \mathbf{0} \approx \mathbf{Ss} \; ,
\end{equation}
which is used to restrict the shape of signal $\mathbf{s}$. By using above
relationships, the pattern-based method raises a constrained least-squares
problem, and solving such a problem can separate signal $\mathbf{s}$ and
noise $\mathbf{n}$ from data volume $\mathbf{d}$.
\begin{equation}
    \label{eq:NsNd}
    \min_{\mathbf{s}} \| \; \mathbf{Ns} - \mathbf{Nd} \; \|_{2}^{2}
    + \epsilon^{2} \; \| \; \mathbf{Ss} \; \|_{2}^{2} \; ,
\end{equation}
where $ \epsilon > 0 $ is the scaling factor of regularization term, which
balancing the energy between estimated signal $\mathbf{\bar{s}}$ and
noise $\mathbf{\bar{n}}$.

In practice, field data d and the noise model are often available, but the
clean signal is not. We here considered the noise model to be a dataset
containing the properties of noise $\mathbf{n}$, and the noise model can
be obtained as a roughly separated noise section. Therefore, data pattern
$\mathbf{D}$ and noise pattern $\mathbf{N}$ are easily estimated from the field data
and noise model, while it cannot directly obtain signal pattern $\mathbf{S}$.
Normally, one makes a compromise by replacing $\mathbf{S}$ with $\mathbf{D}$:
with $\mathbf{D}$:
\begin{equation}
    \label{eq:NsNd1}
    \min_{\mathbf{s}} \| \; \mathbf{Ns} - \mathbf{Nd} \; \|_{2}^{2}
    + \epsilon^{2} \; \| \; \mathbf{Ds} \; \|_{2}^{2} \; ,
\end{equation}
where $\mathbf{D}$ may be unsuitable for constraining the shape of
signal $\mathbf{s}$, and finally lead to undesirable separation result.
Assuming that noise $\mathbf{n}$ and signal $\mathbf{s}$ are uncorrelated,
\cite{Spitz99} proposed an approximation $\mathbf{S} = \mathbf{N}^{-1}\mathbf{D}$,
which is based on the relationship between the pattern operator and the
energy spectrum of the corresponding component, and the regularization term
becomes $ \|\mathbf{N}^{-1}\mathbf{Ds}\|_{2}^{2} $. Furthermore, one can
multiply equation~\ref{eq:NsNd1} by $\mathbf{N}$ to avoid the acquisition
of $\mathbf{N}^{-1}$:
\begin{equation}
    \label{eq:NsNd2}
    \min_{\mathbf{s}} \| \; \mathbf{N} (\mathbf{Ns} - \mathbf{Nd}) \; \|_{2}^{2}
    + \epsilon^{2} \; \| \; \mathbf{N} (\mathbf{Ds} / \mathbf{N}) \; \|_{2}^{2} \; ,
\end{equation}
then the signal-noise separation problem turns into:
\begin{equation}
    \label{eq:NNsNNd}
    \min_{\mathbf{s}} \| \; \mathbf{NNs} - \mathbf{NNd} \; \|_{2}^{2}
    + \epsilon^{2} \; \| \; \mathbf{Ds} \; \|_{2}^{2} \; .
\end{equation}

Equation~\ref{eq:NNsNNd} avoids the requirement for pattern operator $\mathbf{S}$
of the clean signal, and minimizing the equation leads to the expression of
the estimated signal:
\begin{equation}
    \label{eq:sig}
    \mathbf{\bar{s}} =
    \left( \frac{\mathbf{N}^T \mathbf{N} \mathbf{N}^T \mathbf{N} }
    {\mathbf{N}^T \mathbf{N} \mathbf{N}^T \mathbf{N}
        + \epsilon^2 \mathbf{D}^T \mathbf{D}} \right) \mathbf{d} \;,
\end{equation}
and the formal solution of the estimated noise is:
\begin{equation}
    \label{eq:noi}
    \mathbf{\bar{n}} =
    \left( \frac{\epsilon^2 \mathbf{D}^T \mathbf{D}}
    {\mathbf{N}^T \mathbf{N} \mathbf{N}^T \mathbf{N}
        +\epsilon^2 \mathbf{D}^T \mathbf{D}} \right) \mathbf{d} \;.
\end{equation}
The conjugate gradient algorithm is implemented to calculate the numerical
solution of equations~\ref{eq:sig} and~\ref{eq:noi} (Appendix section).
We will discuss the estimation of the nonstationary pattern operators in
the next section.


\subsection{Estimation of nonstationary pattern operators}
Seismic events appear to be stationary in a small time-space window,
but their behaviors will change with time and space. The patching method
assumes seismic events with constant slope within the time-space window,
which is used to deal with nonstationarity, but it may fail when
encountering steeply changing slope. We consider here APEF as the pattern
operator for better characterizing the pattern of nonstationary seismic
data. Similar to decreasing the patch size to a data sample, the APEF
without patching windows can handle the spectrum variability of seismic
data in the time-space domain.

To estimate the nonstationary pattern of a 2D seismic section $d(t,x)$,
prediction coefficients An of the APEF can be obtained as:
\begin{equation}
    \label{eq:pred2}
    \bar{\mathbf{A}}_{n} (t,x) = \arg \min_{A_{n}} \| \; \mathbf{d}(t,x)
    - \sum_{n=1}^{N} \mathbf{A}_{n}(t,x) \mathbf{d}_{n}(t,x) \; \|_2^2
    + \lambda^{2} \; \sum_{n=1}^{N} \| \; \mathbf{R} [\mathbf{A}_{n}(t,x)] \; \|_2^2\;,
\end{equation}
where  $\mathbf{d}_{n}(t,x) = \mathbf{d}(t-i,x-j)$ represents the adjacent
data around $\mathbf{d}(t,x)$. $i$ and $j$ are the index of time shift and
spatial shift, respectively. $\lambda$ is the scaling parameter and
$\mathbf{R[\bullet]}$ is the shaping regularization operator~\cite{Fomel09}.

To obtain the whitening output, one needs to design APEF $\mathbf{D}$ of data
with the causal filter structure. For example, a five-sample (time) $\times$
three-sample (space) template is shown as:
\begin{equation}
    \begin{array}{ccc}
        .          & A_{3}(t,x) & A_{8}(t,x)  \\
        .          & A_{4}(t,x) & A_{9}(t,x)  \\
        1          & A_{5}(t,x) & A_{10}(t,x) \\
        A_{1}(t,x) & A_{6}(t,x) & A_{11}(t,x) \\
        A_{2}(t,x) & A_{7}(t,x) & A_{12}(t,x)
    \end{array} .
    \label{eq:causalpef}
\end{equation}
The structure of filter coefficients influences the prediction result, and it
could be limited by the range of local slope and the variability of seismic
events along the space and time axes. For steeply dipping events, it suggests
a large prediction window in the time direction. Obviously, the pattern of random
noise is different from that of ground-roll noise, and we calculated noise
pattern $\mathbf{N}$ according to the following approaches:

(i) Random noise: it supposes that the energy of random noise is spatially
uncorrelated, and its statistical property may slightly change with time.
To characterize the model of random noise, noise pattern N can be set as the
shape of a column. The following is an example of the noise pattern structure
with 4 (time) $\times$ 1 (space) coefficients:
\begin{equation}
    \begin{array}{c}
        1          \\
        A_{1}(t,x) \\
        A_{2}(t,x) \\
        A_{3}(t,x)
    \end{array} .
    \label{eqn:noizpef1}
\end{equation}

We can generate a noise model containing the characteristics of noise $\mathbf{n}$,
and calculate APEF $\mathbf{N}$ from the noise model. Also, one can directly
estimate APEF $\mathbf{N}$ of random noise from dataset $\mathbf{d}$, especially
when there exists strong random noise in the dataset. $\mathbf{N}$ with one-column
shape can only capture the temporal spectrum of random noise, but ignores the
signal predictability along the space direction in the dataset.

(ii) Ground-roll noise: due to the difference of the dominant frequency, ground-roll
noise and the effective signal can usually be separated in the frequency domain.
Using a low-pass filter to the data can produce a noise model. Similarly, according
to the difference of slowness, the primaries can be muted in the radon domain, and
a ground-roll noise model can be obtained through the inverse radon transform. Here,
we first use a reliable low-pass filter to generate the ground-roll noise model, then
APEF $\mathbf{N}$ of the ground-roll noise is calculated according to the structure
similar to that of data as equation~\ref{eq:causalpef}. Due to the slower speed of
the ground-roll noise, it has steep events with larger local slope, and the filter
size needs to be adjusted to a larger length in the time direction.

Therefore, the proposed signal-noise separation method exploits a two-step strategy:
(i) estimating data pattern $\mathbf{D}$ and noise pattern $\mathbf{N}$ by using
the APEF, and (ii) separating signal and noise with the pattern-based method
(equation~\ref{eq:NNsNNd}). The further examination of the proposed method will
be shown in the data examples section.

\section{Data examples}

\subsection{Nonstationary signal and random noise separation}

We started with a synthetic model (figure~\ref{fig:cm-data}) containing a curve
event with varying slope. Figure~\ref{fig:cm-noiz} is the model data with the white
Gaussian noise added. The F-K spectra (figure~\ref{fig:cm-fkdata,cm-fknoiz}) show
that the strong random noise severely affects the curve model. For comparison,
we first used FXDECON, a standard industry method, to separate the
signal from the noisy data. We designed the FXDECON with four-sample
(space) filter size and ten-sample (space) sliding window. Although
the FXDECON method results in a highest SNR (Table~\ref{tbl:randomnoise}), it still fails
to deal with strong random noise. The strong random noise has been
suppressed in the estimated signal section (figure~\ref{fig:cm-fx}), but a large
part of signal is also destroyed and leaves the estimated noise
section (figure~\ref{fig:cm-fxnoiz}). Due to the strong energy of the random noise,
the high SNR is caused by the suppression of both noise and signal.
To demonstrate the effectiveness of the proposed nonstationary APEF,
we separated the random noise using the pattern-based method
(equation~\ref{eq:NNsNNd}) with the stationary PEF in the t-x domain.
The filter size of data pattern (PEF) D is selected with 11 (time)
$\times$ 4 (space), noise pattern (PEF) N is 5 (time) $\times$ 1
(space) coefficients. Figure~\ref{fig:cm-pef} shows that the stationary approach
fails in separating the random noise in the estimated signal section
as the stationary PEF is hard to characterize the nonstationary data
pattern. Meanwhile, some energy appears from the signal in the estimated
noise section (figure~\ref{fig:cm-pefnoiz}). Figure~\ref{fig:cm-ct} shows the
denoised result of the curvelet transform with percentage threshold;
this method effectively suppresses the random noise in the range from
about 30 to 100 Hz (figure~\ref{fig:cm-fkct}), but it generates
striped noise interference that affects the quality of the synthetic model.
For the proposed method, we configured the APEF with the same filter
size as that of the stationary PEF. The smoothing radii in the time
and space directions for data pattern (APEF) $\mathbf{D}$ is selected
to be 30 and 15 samples, respectively. And the noise pattern (APEF)
$\mathbf{N}$ has a 300-sample (time) $\times$ 1-sample (space) smoothing
radius. Figure~\ref{fig:cm-apef} contains a part of the low-level noise,
while the curve event has been recovered. The proposed method shows
better signal protection ability, and obtains the denoised result
at a high SNR level (Table~\ref{tbl:randomnoise}). Here the SNR is
calculated by equation~\ref{eq:snr}:
\begin{equation}
    \label{eq:snr}
    SNR = 10 \log_{10} (
    \frac{\| \mathbf{s} \|_{2}^{2}}
    {\| \mathbf{s} - \mathbf{\bar{s}} \|_{2}^{2}})
\end{equation}
where $\mathbf{s}$ denotes effective signal, $\mathbf{\bar{s}}$ denotes
denoised result. In the F-K spectra (figure~\ref{fig:cm-fkfx,cm-fkpef,cm-fkct,cm-fkapef}),
the energy of the curve event can be identified only in the denoised
results using the curvelet transform and proposed method.

\newcommand{\tabincell}[2]{\begin{tabular}{@{}#1@{}}#2\end{tabular}}
\tabl{randomnoise}{Comparison of the SNR of the random noise attenuation results}
{
    \begin{center}
        \begin{tabular}{p{60pt}p{60pt}p{60pt}p{60pt}p{60pt}p{60pt}}
            \hline
                & Synthetic model & FXDECON & Stationary PEF & Curvelet transform & \tabincell{c}{The \\proposed \\method} \\
            \hline
            SNR & -9.432          & -3.957  & -5.176         & -5.414             & -5.029            \\
            \hline
        \end{tabular}
    \end{center}
}

\inputdir{curvemod}
\multiplot{2}{cm-data,cm-noiz}
{width=0.47\columnwidth}
{(a) Clean data with curve event and (b) data with random noise.}

\multiplot{2}{cm-fkdata,cm-fknoiz}
{width=0.47\columnwidth}
{(a) F-K spectra of clean data and (b) noisy data.}

\multiplot{8}{cm-fx,cm-fxnoiz,cm-pef,cm-pefnoiz,cm-ct,cm-ctnoiz,cm-apef,cm-apefnoiz}
{width=0.47\columnwidth}
{   (a) Estimated signal and (b) random noise by the FXDECON.
    (c) Estimated signal and (d) random noise by the pattern-based method with the stationary PEF.
    (e) Estimated signal and (f) random noise by the curvelet transform.
    (g) Estimated signal and (h) random noise by the proposed method.}

\multiplot{4}{cm-fkfx,cm-fkpef,cm-fkct,cm-fkapef}
{width=0.47\columnwidth}
{   (a)F-K spectrum of the denoised result using the FXDECON.
    (b) F-K spectrum of the denoised result using the pattern-based method with the stationary PEF.
    (c) F-K spectrum of the denoised result using the curvelet transform.
    (d) F-K spectrum of the denoised result using the proposed method.}

Figure~\ref{fig:nmo-data,nmo-fkdata} shows a poststack section, where the random
noise influences the continuity of the reflection events. The data includes dipping
beds and a fault. The main challenge in this example is that the random
noise displays the nonstationary energy distribution.
Figure~\ref{fig:nmo-fx,nmo-fxnoiz} shows the separation results by using FXDECON,
the filter size has eight sample (space) coefficients for each data sample and
a 30-sample (space) sliding window for handling the variation of the
signals. The FXDECON method fails in separating the nonstationary
signal and random noise; weak energy random noise is still present in
the signal section (figure~\ref{fig:nmo-fx}) and part of the
signal leaks into the noise section (figure~\ref{fig:nmo-fxnoiz}).
By using the pattern-based method with the stationary PEF, we got
the denoised result in figure~\ref{fig:nmo-hpef,nmo-hpefnoiz}.
The data pattern (PEF) $\mathbf{D}$ is selected with 11 (time) $\times$
4 sample (space) coefficients, noise pattern (PEF) $\mathbf{D}$ has 9
(time) $\times$ 1 sample (space) coefficients.
In figure~\ref{fig:nmo-hpef,nmo-hpefnoiz}, the denoised result contains some high
frequency noise, and part of the signal is removed. As shown in figure~\ref{fig:nmo-fkhpef},
the energy of effective signal at about 18 Hz is partially filtered out.
In the denoised result of curvelet transform with percentage threshold
(figure~\ref{fig:nmo-ct,nmo-ctnoiz}), the random noise causes a stronger smearing of
the events. Then we deal with the post- stack section by using the proposed method.
The data APEF $\mathbf{D}$ has 9 (time) $\times$ 4 sample (space)
coefficients for each sample and the smoothing radius is selected to
be 60 (time) and 20 samples (space). The noise APEF $\mathbf{N}$ is
designed as a 9 (time) $\times$ 1 sample (space) coefficients with a
60 (time) $\times$ 1 sample (space) smoothing radius. The estimated
signal section displays that the continuity of the reflection layers
with better smoothness is enhanced, and the fault is well preserved
(figure~\ref{fig:nmo-apef}). The energy of the events and fault hardly
leak into the noise section (figure~\ref{fig:nmo-apefnoiz}).
The F-K spectrum in figure~\ref{fig:nmo-fkapef} is cleaner and the effective signal
energy is concentrated. It indicates that the proposed method has a
better capability for noise suppression and signal protection than
other methods.

\inputdir{nmostack}
\multiplot{2}{nmo-data,nmo-fkdata}
{width=0.47\columnwidth}
{(a) Poststack section with random noise and (b) the corresponding F-K spectrum.}

\multiplot{2}{nmo-fx,nmo-fxnoiz}
{width=0.47\columnwidth}
{Random noise separation results of the poststack section using the FXDECON.
    (a) Section of estimated signal and (b) section of separated random noise.}

\multiplot{2}{nmo-hpef,nmo-hpefnoiz}
{width=0.47\columnwidth}
{Random noise separation results of the poststack section using the
    pattern-based method with the stationary PEF.
    (a) Section of estimated signal and (b) section of separated random noise.}

\multiplot{2}{nmo-ct,nmo-ctnoiz}
{width=0.47\columnwidth}
{Random noise separation results of the poststack section using the curvelet transform.
    (a) Section of estimated signal and (b) section of separated random noise.}

\multiplot{2}{nmo-apef,nmo-apefnoiz}
{width=0.47\columnwidth}
{Random noise separation results of the poststack section using the proposed method.
    (a) Section of estimated signal and (b) section of separated random noise.}

\multiplot{4}{nmo-fkfx,nmo-fkhpef,nmo-fkct,nmo-fkapef}
{width=0.47\columnwidth}
{   (a)F-K spectrum of the denoised result using the FXDECON.
    (b) F-K spectrum of the denoised result using the pattern-based method with the stationary PEF.
    (c) F-K spectrum of the denoised result using the curvelet transform.
    (d) F-K spectrum of the denoised result using the proposed method.}


\subsection{Nonstationary signal and ground-roll noise separation}

As a kind of coherent noise in land survey, ground-roll noise always
displays high amplitude, low frequency and low velocity. We created a
synthetic model to examine the validity of the proposed method
(figure~\ref{fig:sp-synth}). The shot gather has low-frequency ground-roll
noise and four hyperbolic reflection events, which overlap in the
frequency domain (figure~\ref{fig:sp-compspt}). The linear ground-roll
noise is generated by combining linear events with different dominant
frequency, and its frequency range is limited from 0 to about 30 Hz.
For comparison, the high-pass filter with a 25-Hz cutoff frequency
produces the ground-roll noise separation result
(figure~\ref{fig:sp-bp,sp-bpnoiz}).
The nonstationary signal and ground-roll noise show the energy leakage
in each other's sections because the frequency bandwidth of the
ground-roll noise overlaps with that of the reflection events.
Figure~\ref{fig:sp-ltft,sp-ltftnoiz} shows the noise separation result
by using the local time-frequency (LTF) decomposition~\cite[]{Liu13}.
In the time-frequency-space domain, we designed a muting filter to remove
the ground-roll noise components, which leads to a higher SNR result
(Table~\ref{tbl:ground-roll}). By using the proposed method, the ground-roll
noise can be separated in to the following steps~\cite[]{Fomel02}:

(i) Use a low-pass filter to roughly separate the ground-roll noise
as an initial estimation.\\
(ii) Make a mask from the initial noise model, and subsequent
separation process is restricted in this time-space window.\\
(iii) Calculate noise pattern $\mathbf{N}$ from the noise model using APEF.\\
(iv) Calculate data pattern $\mathbf{D}$ from the origin data using APEF.\\
(v) Further separate the signal and ground-roll noise based on
equation~\ref{eq:NNsNNd}.

The noise model is made by the low-pass filter with a 12-Hz cutoff
frequency. We selected 12 (time) $\times$ 3 samples (space) filter
coefficients for noise APEF $\mathbf{N}$ to predict the energy of
the steep ground-roll noise, the smoothing radius is set to be 20
(time) $\times$ 10 samples (space). Meanwhile, data APEF $\mathbf{D}$
has 7 (time) $\times$ 4 samples (space) filter coefficients, the
smoothing radius is 40 (time) $\times$ 30 samples (space).
The separated signal by using the proposed method is well recovered
(figure~\ref{fig:sp-apef}), and the SNR of the denoised results is
comparable to that of LTF decomposition (Table~\ref{tbl:ground-roll}).
The estimated noise section further verifies the successful signal-noise
separation (figure~\ref{fig:sp-apefnoiz}). Although some noise at low
frequency is still present (figure~\ref{fig:sp-fkapef}) and affects
the SNR, the energy of the effective signal is much cleaner after
using the proposed method.

\tabl{ground-roll}{Comparison of the SNR of the ground-roll noise separation results}
{
    \begin{center}
        \begin{tabular}{p{70pt}p{70pt}p{70pt}p{70pt}p{70pt}}
            \hline
                & Synthetic model & \tabincell{c}{High-pass                 \\filtering} & \tabincell{c}{LTF\\decomposition} & \tabincell{c}{The\\proposed\\method} \\
            \hline
            SNR & -13.3           & 6.514                   & 8.988 & 8.001 \\
            \hline
        \end{tabular}
    \end{center}
}

\inputdir{simplegroll}
\multiplot{2}{sp-synth,sp-compspt}
{width=0.47\columnwidth}
{(a) Synthetic shot gather containing ground-roll and (b) comparison of the spectrum.
    Solid line is the synthetic model, dashed lines the ground-roll noise
    (5 Hz dominant frequency) and reflection events (30 Hz dominant frequency).}

\multiplot{2}{sp-bp,sp-bpnoiz}
{width=0.47\columnwidth}
{Noise separation results of the synthetic shot gather using the high-pass filtering.
    (a) Section of estimated signal and (b) section of separated ground-roll noise.}

\multiplot{2}{sp-ltft,sp-ltftnoiz}
{width=0.47\columnwidth}
{Noise separation results of the synthetic shot gather using the LTF decomposition.
    (a) Section of estimated signal and (b) section of separated ground-roll noise.}

\multiplot{2}{sp-apef,sp-apefnoiz}
{width=0.47\columnwidth}
{Noise separation results of the synthetic shot gather using the proposed method.
    (a) Section of estimated signal and (b) section of separated ground-roll noise.}

\multiplot{4}{sp-fksynth,sp-fkbp,sp-fkltft,sp-fkapef}
{width=0.47\columnwidth}
{(a) F-K spectrum of the synthetic model.
    (b) F-K spectrum of the denoised result using the high-pass filtering.
    (c) F-K spectrum of the denoised result using the LTF decomposition.
    (d) F-K spectrum of the denoised result using the proposed method.}


Figure~\ref{fig:dune-data,dune-spt} shows the profile and the corresponding
spectrum of a 3D land shot gather from Saudi Arabia, where the ground-roll
noise is three-dimensional and appears as a hyperbolic shape in this
receiver line. Due to the difference of dominant frequency between the
ground-roll noise and reflection events, it may suggest separating them
based on different frequency. The frequency-based separation method leads
to a worse result with a simple filter design, and the estimated
signal by using the high-pass filter with 20-Hz cutoff frequency
contains more low-level ground-roll noise (figure~\ref{fig:dune-bp}).
The LTF decomposition method with a carefully selected muting filter
gets better results than those from high-pass filtering, but it still
leaves some ground-roll noise in the middle part of the estimated signal
section (figure~\ref{fig:dune-ltft}). We used the initial noise model
(figure~\ref{fig:dune-bpnoiz}) to characterize the ground-roll noise,
noise APEF $\mathbf{N}$ is designed with 12 (time) $\times$ 3 samples
(space) filter size and 20 (time) $\times$ 15 samples (space) smoothing
radius. Data APEF $\mathbf{D}$ uses a smaller filter size with 5 (time)
$\times$ 4 samples (space) coefficients and 30 (time) $\times$ 25 samples
(space) smooth radius to capture signal energy. In the F-K spectrum
(figure~\ref{fig:dune-aspec}), there are also some weaker energies below 20 Hz,
but the ground-roll noise has been separated from the denoised results
(figure~\ref{fig:dune-sign}). The proposed method achieved the separation
goal that the underlying reflection events clearly appear in the estimated
section, and the ground-roll noise is well suppressed.

\inputdir{dune}
\multiplot{2}{dune-data,dune-spt}
{width=0.47\columnwidth}
{(a) A receiver line of the 3D shot gather and (b) the corresponding spectrum.}

\multiplot{2}{dune-bp,dune-bpnoiz}
{width=0.47\columnwidth}
{Noise separation results of the field data using the high-pass filtering.
    (a) Section of estimated signal and (b) section of separated ground-roll noise.}

\multiplot{2}{dune-ltft,dune-ltftnoiz}
{width=0.47\columnwidth}
{Noise separation results of the field data using the LTF decomposition.
    (a) Section of estimated signal and (b) section of separated ground-roll noise.}

\multiplot{2}{dune-sign,dune-noiz}
{width=0.47\columnwidth}
{Noise separation results of the field data using the proposed method.
    (a) Section of estimated signal and (b) section of separated ground-roll noise.}

\multiplot{4}{dune-dspec,dune-bspec,dune-lspec,dune-aspec}
{width=0.47\columnwidth}
{   (a) F-K spectrum of the shot gather.
    (b) F-K spectrum of the denoised result using the high-pass filtering.
    (c) F-K spectrum of the denoised result using the LTF decomposition.
    (d) F-K spectrum of the denoised result using the proposed method.}

\section{Discussion}
We considered that the APEF can characterize the properties of data and
treated the corresponding APEF as the pattern operator. Data pattern
operator $\mathbf{D}$ and noise pattern operator $\mathbf{N}$ can be obtained
by solving the corresponding APEF. Since APEF uses the shaping regularization
constraint, there are two main parameters that affect the filter: one is the
filter size, and the other is the smoothing radius~\cite[]{Liu11}.
These two parameters are empirical, and they are related to the characteristics
of the events, including spatial distribution, local slope, etc. We applied the
pattern operators to the corresponding dataset and adjusted the parameters of
the APEF by observing whether the corresponding data components were absorbed or not.
For example, the noise component can be absorbed by using pattern operator
$\mathbf{N}$ in the noise model ($ \mathbf{0} \approx \mathbf{Nn} $).


\section{Conclusion}
We introduced a new pattern-based approach for nonstationary signal-noise
separation. Our method used the APEF as the pattern operator, which was suitable
for characterizing the nonstationary properties of seismic data and noise
in the time-space domain. After calculating the data pattern $\mathbf{D}$ and
the noise pattern $\mathbf{N}$, we could separate the signal and noise by solving
a constrained least-squares problem. We adopted different algorithms to deal
with the random noise and ground-roll noise separation problem. Numerical examples
showed that the proposed method provided a robust signal-noise separation,
even in the presence of random noise with nonstationary energy distribution and
strongly curved ground roll. Multiple suppression and diffraction separation were
also other applications of this method.


\section{Acknowledgement}
Thanks to Sergey Fomel for the enlightening discussion. This work is supported
by the National Natural Science Foundation of China (grant nos. 41974134, 41874125,
and 41774127) and the National key Research and Development Program of China
(grant no. 2018YFC0603701).

\appendix
\section{Conjugate gradient algorithm}

The pattern-based method can be described as the constrained linear
equation system:
\begin{equation}
    \begin{aligned}
        \mathbf{0} & \approx \mathbf{NNs} - \mathbf{NNd}
        \mathbf{0} & \approx \epsilon \mathbf{Ds}
    \end{aligned}
\end{equation}
where $\mathbf{NN}$ denotes the chain operator, which applies operator
$\mathbf{N}$ twice. The following conjugate gradient algorithm~\cite[]{Wang16}
can be used to solve such problem.
\begin{algorithm}{Conjugate gradients with regularization}
    {\mathbf{NN}, \mathbf{S},\mathbf{d}, \mathbf{s},\epsilon}
    \mathbf{s \= 0} \\
    \mathbf{r \= - NNd} \\
    \begin{FOR}{n \= 1, 2, \ldots, Niter} \\
        \mathbf{g} \= \mathbf{NN}^T\,\mathbf{r} + \epsilon\,\mathbf{S}^{T}\,\mathbf{s}\\\
        \mathbf{G} \= \mathbf{NN}\,\mathbf{g} + \epsilon\,\mathbf{S}\,\mathbf{g}\\
        \rho \= \mathbf{g}^T\,\mathbf{g} \\
        \begin{IF}{n = 1}
            \beta \= 0
            \ELSE
            \beta \= \rho/\hat{\rho}
        \end{IF} \\
        \left[\begin{array}{l}
                \mathbf{p} \\
                \mathbf{P}
            \end{array}\right] \=
        \left[\begin{array}{l}
                \mathbf{g} \\
                \mathbf{G}
            \end{array}\right] + \beta\,
        \left[\begin{array}{l}
                \mathbf{p} \\
                \mathbf{P}
            \end{array}\right] \\
        \alpha \= - \rho/(\mathbf{P}^T\,\mathbf{P}) \\
        \left[\begin{array}{l}
                \mathbf{s} \\
                \mathbf{r}
            \end{array}\right] \=
        \left[\begin{array}{l}
                \mathbf{s} \\
                \mathbf{r}
            \end{array}\right] + \alpha\,
        \left[\begin{array}{l}
                \mathbf{p} \\
                \mathbf{P}
            \end{array}\right] \\
        \hat{\rho} \= \rho
    \end{FOR} \\
    \RETURN \mathbf{s}
\end{algorithm}


\bibliographystyle{seg}
\bibliography{SEG,paper}
