\published{Geophysical Journal International, 229, 370-389, (2022)}

\title{Seismic data interpolation using streaming prediction filter in the frequency domain}

\renewcommand{\thefootnote}{\fnsymbol{footnote}}

\ms{GJI-S-21-0486}

\address{
    \footnotemark[1] College of Geo-exploration Science and Technology,\\
    Jilin University, Changchun, China}

\author{Zhisheng Zheng\footnotemark[1], Yang Liu\footnotemark[1], Cai Liu\footnotemark[1]}

\footer{GJI-S-21-0486}
\lefthead{Zheng et al.}
\righthead{Interpolation by SPF}

\maketitle

\begin{abstract}
  Surface conditions and economic factors restrict field geometries,
  so seismic data acquisition typically obtains field data with
  irregular spatial distribution, which can adversely affect the
  subsequent data processing and interpretation. Therefore, data
  interpolation techniques are used to convert field data into
  regularly distributed data and reconstruct the missing traces.
  Recently, the mainstream methods have implemented iterative
  algorithms to solve data interpolation problems, which require
  substantial computational resources and restrict their application
  in high dimensions.  In this study, we proposed the $f$-$x$ and
  $f$-$x$-$y$ streaming prediction filters (SPFs) to reconstruct
  missing seismic traces without iterations. According to the
  streaming computation framework, we directly derived an analytic
  solution to the overdetermined least-squares problem with local
  smoothness constraints for estimating SPFs in the frequency
  domain. We introduced different processing paths and filter forms to
  reduce the interference of missing traces, which can improve the
  accuracy of filter coefficients. Meanwhile, we utilized a two-step
  interpolation strategy to guarantee the effective interpolation of
  the irregularly missing traces.  Numerical examples show that the
  proposed methods effectively recover the missing traces in seismic
  data when compared with the traditional Fourier Projection Onto
  Convex Sets (POCS) method. In particular, the frequency domain SPFs
  are suitable for high-dimensional seismic data interpolation with
  the advantages of low computational cost and reasonable
  nonstationary signal reconstruction.

\end{abstract}


\section{Introduction}

With the limitation of field geometries, uniform data acquisition is
rarely achieved in practice; instead, it always displays irregular or
undersampled data distribution in spatial directions. However, many
subsequent processing steps, such as multiple elimination and
migration, are based on the prerequisite of regular data distribution.
Seismic data interpolation has become a key technique in seismic data
processing workflows. A failed interpolation method may create
artifacts, which affect the accuracy of seismic imaging. Recently,
high-density and wide-azimuth seismic acquisition can achieve
large-scale field data; however, increasing computational costs and
nonstationary signal recovery have posed challenges to data
interpolation.  Many interpolation methods have been proposed to
reconstruct missing data, which are based on signal processing or
seismic kinematic/dynamic principles.  Low-rank
methods~\citep{Trickett10,Gao13,Chen16,Gao17} assume that seismic data
are low rank in the mapping domain, and rank-reduction operators are
used to recover the missing data.  Plane wave
decomposition~\citep{Fomel02, Hellman16} reconstructs missing data by
using local slope information.  Compressive sensing framework with
different sparse domains, e.g., Fourier transform~\citep{Abma06,
  Wang10} and seislet transform~\citep{Fomel10, Liu10, Gao15},
iteratively recover the missing traces.  Machine learning has become a
popular research direction, and is also used for data
interpolation~\citep{Jia17,Oliveira18, Mandelli19, Kaur19, Zhang20}.
However, these methods always encounter high computational cost.

Prediction filter (PF) provide an important approach for seismic data
interpolation. \citet{Spitz91} proved that high-frequency components
can be estimated with the PF calculated from low-frequency components,
and an $f$-$x$ interpolation method beyond aliasing was proposed.
\citet{Sacchi97} used an autoregression moving average (ARMA) model to
calculate the PF and interpolated near-offset missing gaps in the
$f$-$x$ domain. \citet{Porsani99} proposed a half-step PF to
efficiently interpolate missing data. \citet{Wang02} extended
different kinds of PFs in high dimensions to implement the $f$-$x$
interpolation algorithm. \citet{Abma05} made comparison of several PF
interpolation methods. \citet{Naghizadeh09} used an exponentially
weighted recursive least squares (EWRLS) for the adaptive PF to
interpolate data in the $f$-$x$ domain. \citet{Wangy10} created the
virtual traces from marine seismic data and utilized the matching
filter or nonstationary prediction error filter (PEF) to fill gaps.
\citet{Liu11} proposed an approach to interpolate aliased data based
on adaptive PEF and regularized nonstationary autoregression (RNA) in
the $t$-$x$-$y$ domain. \citet{Li17} proposed multidimensional
adaptive PEF to reconstruct seismic data in the frequency domain.
\citet{Liug18} introduced an efficient method based on the $f$-$x$ RNA
for regular and irregular missing data interpolation in the $f$-$x$
domain. \citet{Liu19} designed the multiscale and multidirectional
PEF to improve the accuracy of filter coefficients while
reconstructing seismic data.

It is difficult to balance the computational cost and the
interpolation accuracy while dealing with large-scale data in most
iterative methods. However, the framework of the streaming
computation~\citep{Sacchi09, Fomel16} can directly solve the
nonstationary autoregression problem. \citet{Liuy18} proposed an
orthogonal $t$-$x$ streaming PF (SOPF) for random noise attenuation,
and~\citet{Guo20} provided an initial idea for the $f$-$x$ SPF with a
1D spatial constraint and tried to efficiently eliminate seismic
random noise. Therefore, we further improved the noniterative
framework of the streaming computation in the frequency domain and
used the SPF with new frequency-domain constraints for seismic data
interpolation. The new adaptive PF can reduce the computational cost
in the interpolation problem while capturing details of nonstationary
seismic data.  In this study, we first discussed the two-step
interpolation strategy for PF. With the two-step strategy, we derived
the theory of the new $f$-$x$ SPF and $f$-$x$-$y$ SPF based on the
streaming framework.  The relevant interpolation algorithm and filter
design were developed to help the SPF solve the problem of the
nonstationary data reconstruction with a feasible computational
cost. Numerical examples demonstrate the effectiveness and efficiency
of the proposed methods.

\section{Theory}

\subsection{Two-step interpolation strategy}

The classical method for restoring missing data is to ensure that the
restored data, after specified filtering, have minimum
energy~\citep{Claerbout92}.  Data interpolation by prediction
filtering can be divided into two problems~\citep{Liu11}: filter
estimation and data reconstruction with filters.  The first step is to
calculate the filter by minimizing the following autoregression
problem:
\begin{equation}
    \label{eq:step1}
    \min_{\mathbf{F}} \| \mathbf{g} - \mathbf{G}^{\mathsf{T}} \mathbf{F} \|_{2}^{2},
\end{equation}
where $\{\bullet\}^{\mathsf{T}}$ denotes the transpose operator,
$\mathbf{F}$ is adaptive PF, $\mathbf{g}$ is the data matrix, and
$\mathbf{G}$ is the translation of $\mathbf{g}$.  In the second step,
the missing data are restored by minimizing the least-squares problem:
\begin{equation}
    \label{eq:step2}
    \min_{\hat{\mathbf{g}}} \| \hat{\mathbf{g}} - \mathbf{G}^{\mathsf{T}}\mathbf{F} \|_{2}^{2},
\end{equation}
where $\hat{\mathbf{g}}$ is the reconstructed data.

In the time-space or frequency-space domain, Eq.~(\ref{eq:step1}) and
(\ref{eq:step2}), established by the prediction filtering theory, are
designed to solve different variables, and they are solved
individually.  The number of unknown filter coefficients in adaptive
PF $\mathbf{F}$ is often greater than the number of known data in
$\mathbf{g}$, in other words, the number of unknowns is greater than
the number of equations.  Eq.~(\ref{eq:step1}) usually presents an
ill-posed problem, and different regularization terms have been
introduced to stabilize the filter solution $\mathbf{F}$.  Further
discussion will be provided in the next section. Meanwhile, we
extended the framework of the streaming computation in the frequency
domain and proposed a local and multidimensional smoothness to
constrain Eq.~(\ref{eq:step1}) in the frequency domain.

\subsection{$f$-$x$ streaming prediction filter}

In the $f$-$x$ domain, the PF predicts the data along the spatial
direction, and the relationship between data and prediction filter can
be summarized as
\begin{equation}
    \label{eq:fxar}
    \begin{aligned}
        g_{m,n}
         & \approx \sum_{p} g_{m,n-p} f_{m,n,p}                    \\
         & \approx \mathbf{G}_{m,n}^{\mathsf{T}} \mathbf{F}_{m,n},
    \end{aligned}
\end{equation}
where $m$ and $n$ are the indices of the seismic data sample $g_{m,n}$
along the frequency $f$ axis and space $x$ axis, respectively.  Vector
$\mathbf{F}_{m,n}$ is the group of filter coefficients $f_{m,n,p}$ in
the adaptive prediction filter, and each group $\mathbf{F}_{m,n}$
corresponds to a data sample $g_{m,n}$.  Vector $\mathbf{G}_{m,n}$
denotes several data points $g_{m,n-p}$ with spatial shift $p$ near
$g_{m,n}$.  Spatial shift $p$ is related to the filter size (the
number of filter coefficients in $\mathbf{F}_{m,n}$), and the filter
size should theoretically be larger than or equal to the number of
seismic events contained in the local space window.  When the causal
filter structure is considered, as shown in Fig.~\ref{fig:causal2d},
the spatial shift is chosen as $ p \in [1, 3] $, vector
$\mathbf{G}_{m,n}$ can be expressed as $ \mathbf{G}_{m,n} = \{
g_{m,n-1}, g_{m,n-2}, g_{m,n-3} \} $, and $ \mathbf{F}_{m,n} = \{
f_{m,n,1}, f_{m,n,2}, f_{m,n,3} \} $.  For the non-causal filter
structure (Fig.~\ref{fig:noncausal2d}), the spatial shift is $ p \in
[-3,-1] \cup [1, 3] $, and vector $\mathbf{G}_{m,n}$ and
$\mathbf{F}_{m,n}$ can be represented as $ \mathbf{G}_{m,n} = \{
g_{m,n+3}, g_{m,n+2}, g_{m,n+1}, g_{m,n-1}, g_{m,n-2}, g_{m,n-3} \} $,
$ \mathbf{F}_{m,n} = \{ f_{m,n,-3}, f_{m,n,-2}, f_{m,n,-1},
f_{m,n,1},\\ f_{m,n,2}, f_{m,n,3} \} $.

With Eq.~(\ref{eq:fxar}), we established the following minimization
problem, like Eq.~(\ref{eq:step1}), to calculate filter
$\mathbf{F}_{m,n}$:
\begin{equation}
    \label{eq:fxcost1}
    \min_{\mathbf{F}_{m,n}} \| g_{m,n}
    - \mathbf{G}_{m,n}^{\mathsf{T}} \mathbf{F}_{m,n} \|_{2}^{2}.
\end{equation}
According to the above explanation of vector $\mathbf{F}_{m,n}$, there
are several unknown filter coefficients, yet we established only one
equation.  Eq.~(\ref{eq:fxcost1}) describes an ill-posed problem,
which requires constraints to obtain a stable solution.  The framework
of the streaming computation~\citep{Sacchi09, Fomel16, Liuy18}
establishes the constraint relationship by using local smoothness.  We
extended this method into the frequency domain, including the $f$-$x$
domain and the $f$-$x$-$y$ domain (the next section) to stabilize the
filter coefficient solution.  Additionally, we discussed in detail the
effects of filter structure and processing path, and provided the
corresponding interpolation algorithms. Here, multiple constraints
based on local smoothness are used to constrain the solution of
Eq.~(\ref{eq:fxcost1}):
\begin{equation}
    \label{eq:fxcost2}
    \begin{aligned}
        \min_{\mathbf{F}_{m,n}} \| g_{m,n}
         & - \mathbf{G}_{m,n}^{\mathsf{T}} \mathbf{F}_{m,n} \|_{2}^{2}
        + \lambda_{f}^{2} \| \mathbf{F}_{m,n} - \mathbf{F}_{m-1,n} \|_{2}^{2}     \\
         & + \lambda_{x}^{2} \| \mathbf{F}_{m,n} - \mathbf{F}_{m,n-1} \|_{2}^{2},
    \end{aligned}
\end{equation}
where $\lambda_{f}$ and $\lambda_{x}$ are the weights of the
regularization terms along the frequency $f$ and space $x$ axes.  $
\lambda_{f}^{2} \| \mathbf{F}_{m,n} - \mathbf{F}_{m-1,n} \|_{2}^{2} $
shows the local smoothness along the frequency direction
($\lambda_{f}\mathbf{F}_{m,n} \approx \lambda_{f}\mathbf{F}_{m-1,n}
$).  Likewise, $ \lambda_{x}^{2} \| \mathbf{F}_{m,n} -
\mathbf{F}_{m,n-1} \|_{2}^{2} $ controls the local smoothness along
the space direction as $\lambda_{x}\mathbf{F}_{m,n} \approx
\lambda_{x}\mathbf{F}_{m,n-1} $.  The block matrix
Eq.~(\ref{eq:fxmatrix}) has the same solution as
Eq.~(\ref{eq:fxcost2}), and it demonstrates the effect of the local
smoothness constraints; when $\mathbf{F}_{m-1,n}$ and
$\mathbf{F}_{m,n-1}$ are considered known, the local smoothness
conditions ($\lambda_{f}\mathbf{F}_{m,n} \approx
\lambda_{f}\mathbf{F}_{m-1,n} $ and $\lambda_{x}\mathbf{F}_{m,n}
\approx \lambda_{x}\mathbf{F}_{m,n-1} $), as newly added equations,
can be used to stabilize the solution of $\mathbf{F}_{m,n}$:
\begin{equation}
    \label{eq:fxmatrix}
    \begin{bmatrix}
        \mathbf{G}_{m,n}^{\mathsf{T}} \\
        \lambda_{f} \mathbf{I}        \\
        \lambda_{x} \mathbf{I}
    \end{bmatrix}
    \mathbf{F}_{m,n}
    \approx
    \begin{bmatrix}
        g_{m,n}                        \\
        \lambda_{f} \mathbf{F}_{m-1,n} \\
        \lambda_{x} \mathbf{F}_{m,n-1}
    \end{bmatrix}.
\end{equation}
One can obtain the following least-squares solution of
Eq.~(\ref{eq:fxcost2}) and~(\ref{eq:fxmatrix}):
\begin{equation}
    \label{eq:fxls}
    \begin{aligned}
        \mathbf{F}_{m,n}=
         & {( \lambda_{f}^{2} \mathbf{I}
        + \lambda_{x}^{2} \mathbf{I}
        + \mathbf{G}_{m,n}^{*}\mathbf{G}_{m,n}^{\mathsf{T}} )}^{-1} \\
         & ( g_{m,n}\mathbf{G}_{m,n}^{*}
        + \lambda_{f}^{2}\mathbf{F}_{m-1,n}
        + \lambda_{x}^{2}\mathbf{F}_{m,n-1} ),
    \end{aligned}
\end{equation}
where $\{\bullet\}^{*}$ denotes the conjugate operator. Let
\begin{equation}
    \begin{cases}
        \lambda^{2} = \lambda_{f}^{2} + \lambda_{x}^{2} \\
        \lambda^{2} \mathbf{\tilde{F}}_{m,n}
        = \lambda_{f}^{2}\mathbf{F}_{m-1,n}
        + \lambda_{x}^{2}\mathbf{F}_{m,n-1}
    \end{cases},
\end{equation}
we get a simplified equation:
\begin{equation}
    \label{eq:simplels}
    \mathbf{F}_{m,n} =
    {( \lambda^{2} \mathbf{I}
    + \mathbf{G}_{m,n}^{*} \mathbf{G}_{m,n}^{\mathsf{T}} )}^{-1}
    ( g_{m,n} \mathbf{G}_{m,n}^{*}
    + \lambda^{2} \mathbf{\tilde{F}}_{m,n} ).
\end{equation}
Meanwhile, $(\lambda^{2}\mathbf{I} +
\mathbf{G}_{m,n}^{*}\mathbf{G}_{m,n}^{\mathsf{T}}) $ has the
analytical inversion $ \frac{1}{\lambda^{2}}(\mathbf{I} -
\frac{\mathbf{G}_{m,n}^{*}\mathbf{G}_{m,n}^{\mathsf{T}}} {\lambda^{2}
  + \mathbf{G}_{m,n}^{\mathsf{T}}\mathbf{G}_{m,n}^{*} } ) $ when one
extends the Sherman-Morrison formula~\citep{Sherman50, Bartlett51,
  Hager89} to the complex space. We can obtain the analytical solution
of the $f$-$x$ SPF:
\begin{equation}
    \label{eq:fxsolution}
    \begin{aligned}
        \mathbf{F}_{m,n}
         & = \frac{1}{\lambda^{2}}(\mathbf{I}
        - \frac{ \mathbf{G}_{m,n}^{*}\mathbf{G}_{m,n}^{\mathsf{T}} }
        { \lambda^{2} + \mathbf{G}_{m,n}^{\mathsf{T}}\mathbf{G}_{m,n}^{*} })
        ( g_{m,n} \mathbf{G}_{m,n}^{*}
        + \lambda^{2} \mathbf{\tilde{F}}_{m,n} ) \\
         & = \mathbf{\tilde{F}}_{m,n}
        + \frac{ g_{m,n} - \mathbf{G}_{m,n}^{\mathsf{T}}\mathbf{\tilde{F}}_{m,n} }
        { \lambda^{2} + \mathbf{G}_{m,n}^{\mathsf{T}}\mathbf{G}_{m,n}^{*} }
        \mathbf{G}_{m,n}^{*}.
    \end{aligned}
\end{equation}
Eq.~(\ref{eq:fxsolution}) shows a recursive relationship from the
previous filters ($\mathbf{F}_{m-1,n}$ and $\mathbf{F}_{m,n-1}$) to
current filter $\mathbf{F}_{m,n}$. For data interpolation,
Eq.~(\ref{eq:step2}) becomes a well-posed problem, and the unknown
data sample can be calculated by
\begin{equation}
    \label{eq:fxstep2}
    \hat{g}_{m,n} = \mathbf{G}_{m,n}^{\mathsf{T}} \mathbf{F}_{m,n}.
\end{equation}

In 2D data interpolation, we proposed the following
Algorithm~\ref{alg:fxinterp} to reconstruct the missing seismic data
by using the 2D $f$-$x$ SPF.  To start, $\mathbf{F}_{m-1,n}$ and
$\mathbf{F}_{m,n-1}$ are initialized to $\mathbf{0}$. When the
processing path in Algorithm~\ref{alg:fxinterp} is followed, both
$\mathbf{F}_{m-1,n}$ and $\mathbf{F}_{m,n-1}$ are known.  We designed
a space-causal filter form (Fig.~\ref{fig:causal2d}) for the $f$-$x$
SPF. Fig.~\ref{fig:noncausal2d} suggests that the space-noncausal form
may involve the interference of the unknown data samples, but the
causal one can avoid this problem.

\renewcommand{\algorithmicrequire}{\textbf{Input:}}
\renewcommand{\algorithmicensure}{\textbf{Output:}}
\begin{algorithm}
    % \setstretch{1.2}
    \caption{$f$-$x$ SPF interpolation algorithm}
    \label{alg:fxinterp}
    \begin{algorithmic}[1]
        \REQUIRE
        $g_{m,n}$, $\mathbf{G}_{m,n}$, $\lambda_{f}$, and $\lambda_{x}$;
        \ENSURE
        $\hat{g}_{m,n}$;

        \STATE $ \mathbf{F}_{m-1,n} = \mathbf{F}_{m,n-1} = \mathbf{0} $
        \FOR {loop over frequency direction}
        \FOR {loop over space direction}
        \IF {encounter known data point}
        \STATE $ \mathbf{F}_{m,n} = \mathbf{\tilde{F}}_{m,n} +
            \frac{ g_{m,n} - \mathbf{G}_{m,n}^{\mathsf{T}} \mathbf{\tilde{F}}_{m,n} }
            { \lambda^{2} + \mathbf{G^{T}}_{m,n} \mathbf{G^{*}}_{m,n} }
            \mathbf{G}_{m,n}^{*} $
        \ELSIF {encounter missing data point}
        \STATE $ \hat{g}_{m,n} = \mathbf{G}_{m,n}^{\mathsf{T}} \mathbf{F}_{m,n} $
        \ENDIF
        \ENDFOR
        \ENDFOR
    \end{algorithmic}
\end{algorithm}

\inputdir{.}
\multiplot{2}{causal2d,noncausal2d}
{width=0.47\columnwidth}
{Space-causal filter form (a) and space-noncausal filter form (b)
    in the $f$-$x$ domain. Solid circle denotes known sample,
    and hollow circle denotes unknown sample.}

\subsection{$f$-$x$-$y$ streaming prediction filter}

The extension to the 3D $f$-$x$-$y$ domain is straightforward as the
$f$-$x$-$y$ SPF can also efficiently perform data interpolation in
high dimensions.  In the $f$-$x$-$y$ domain, the prediction
relationship for seismic data in a certain frequency slice is
expressed as
\begin{equation}
    \label{eq:fxyar}
    \begin{aligned}
        g_{m,n,l}
         & \approx \sum_{p} \sum_{q} g_{m,n-p,l-q} f_{m,n,l,p,q}       \\
         & \approx \mathbf{G}_{m,n,l}^{\mathsf{T}} \mathbf{F}_{m,n,l},
    \end{aligned}
\end{equation}
where $m$, $n$, and $l$ denote the indices of data sample $g_{m,n,l}$
along the $f$, $x$, and $y$ axes, respectively.  Vector
$\mathbf{F}_{m,n,l}$ is a group of filter coefficients in the
$f$-$x$-$y$ SPF, vector $\mathbf{G}_{m,n,l}$ contains the data points
with spatial shifts $p$ ($x$ axis) and $q$ ($y$ axis) near
$g_{m,n,l}$.  Similar to the case of the $f$-$x$ SPF, the range of
spatial shifts $p$ and $q$ depends on the number of seismic events in
the local window of space $x$ and space $y$.  As shown in
Fig.~\ref{fig:causal3d,noncausal3d}, when $p \in \{ -1, 1\}$ and $q
\in \{ 1, 3\} $, vector $\mathbf{G}_{m,n,l}$ can be flattened as
$\mathbf{G}_{m,n,l} = \{ g_{m,n+1,l-1}, g_{m,n+1,l-2}, \cdots
,g_{m,n-1,l-3}\}$, and $\mathbf{F}_{m,n,l} = \{ f_{m,n,l,-1, 1},
f_{m,n,l,-1, 2}, \cdots ,f_{m,n,l, 1, 3}\}$.  For the space-noncausal
case, $ p \in \{-1, 1\} $, $q \in \{-3, 3\}$, and $\left | p \right |
+ \left | q \right | \neq 0 $, $\mathbf{G}_{m,n,l}$ can be expressed
as follows: $ \mathbf{G}_{m,n,l} = \{ g_{m,n+1,l+3},\\ g_{m,n+1,l+2},
\cdots, g_{m,n,l+1}, g_{m,n,l-1}, \cdots , g_{m,n-1,l-3} \} $, and $
\mathbf{F}_{m,n,l} = \{ f_{m,n,l, -1, -3}, f_{m,n,l, -1, -2},\\
\cdots, f_{m,n,l, 0, -1}, f_{m,n,l, 0, 1}, \cdots, f_{m,n,l, 1, 3}
\}$.

The cost function is defined as
\begin{equation}
    \label{eq:fxycost1}
    J_{m,n,l} = \| g_{m,n,l} - \mathbf{G}_{m,n,l}^{\mathsf{T}} \mathbf{F}_{m,n,l} \|_{2}^{2}.
\end{equation}
Assuming that the adaptive PF at position $(m,n,l)$ is similar to
another near positions $(m-1,n,l)$, $(m,n-1,l)$, and $(m,n,l-1)$ in
the $f$-$x$-$y$ domain, we can minimize the new cost function
Eq.~(\ref{eq:fxycost2}) to solve the overdetermined problem:
\begin{equation}
    \label{eq:fxycost2}
    \begin{aligned}
        \hat{J}_{m,n,l}
         & = J_{m,n,l} + \lambda_{f}^{2} \| \mathbf{F}_{m,n,l} - \mathbf{F}_{m-1,n,l} \|_{2}^{2} \\
         & + \lambda_{x}^{2} \| \mathbf{F}_{m,n,l} - \mathbf{F}_{m,n-1,l} \|_{2}^{2}             \\
         & + \lambda_{y}^{2} \| \mathbf{F}_{m,n,l} - \mathbf{F}_{m,n,l-1} \|_{2}^{2},
    \end{aligned}
\end{equation}
where $\lambda_{f}$, $\lambda_{x}$, and $\lambda_{y}$ are the constant
weights of the regularization terms, which describe the local
smoothness constraints along different directions. The simplified
block matrix can be written as
\begin{equation}
    \label{eq:fxymatrix}
    \begin{bmatrix}
        \mathbf{G}_{m,n,l}^{\mathsf{T}} \\
        \lambda_{f} \mathbf{I}          \\
        \lambda_{x} \mathbf{I}          \\
        \lambda_{y} \mathbf{I}
    \end{bmatrix}
    \mathbf{F}_{m,n,l}
    \approx
    \begin{bmatrix}
        g_{m,n,l}                        \\
        \lambda_{f} \mathbf{F}_{m-1,n,l} \\
        \lambda_{x} \mathbf{F}_{m,n-1,l} \\
        \lambda_{y} \mathbf{F}_{m,n,l-1}
    \end{bmatrix}.
\end{equation}
Similar to the 2D case, the analytic solution of the $f$-$x$-$y$ SPF
is given by
\begin{equation}
    \label{eq:fxysolution}
    \mathbf{F}_{m,n,l} = \mathbf{\tilde{F}}_{m,n,l} +
    \frac{ g_{m,n,l} - \mathbf{G}_{m,n,l}^{\mathsf{T}} \mathbf{\tilde{F}}_{m,n,l} }
    { \lambda^{2} + \mathbf{G}_{m,n,l}^{\mathsf{T}} \mathbf{G}_{m,n,l}^{*} }
    \mathbf{G}_{m,n,l}^{*},
\end{equation}
where
\begin{equation}
    \begin{cases}
        \lambda^{2} = \lambda_{f}^{2}+\lambda_{x}^{2}+\lambda_{y}^{2} \\
        \lambda^{2} \mathbf{\tilde{F}}_{m,n,l}
        = \lambda_{f}^{2} \mathbf{F}_{m-1,n,l}
        + \lambda_{x}^{2} \mathbf{F}_{m,n-1,l}
        + \lambda_{y}^{2} \mathbf{F}_{m,n,l-1}
    \end{cases}.
\end{equation}

Furthermore, the unknown data sample in the 3D data cube can
be calculated as
\begin{equation}
    \label{eq:fxystep2}
    \hat{g}_{m,n,l} = \mathbf{G}_{m,n,l}^{\mathsf{T}} \mathbf{F}_{m,n,l}.
\end{equation}

Because the $f$-$x$-$y$ SPF predicts data along two spatial
directions, we defined a new processing path in the space plane with
zigzag shape (Fig.~\ref{fig:causal3d}) to prevent unnecessary filter
initialization. Meanwhile, the $f$-$x$-$y$ SPF was assigned to the
proposed filter form (Fig.~\ref{fig:causal3d}), which can better
reduce the influence of unknown data samples than the noncausal filter
in spatial directions (Fig.~\ref{fig:noncausal3d}).  Following the
processing path of Algorithm~\ref{alg:fxyinterp} for data processing,
$\mathbf{F}_{m-1,n,l}$, $\mathbf{F}_{m,n-1,l}$, and
$\mathbf{F}_{m,n,l-1}$ can be seen as known, requiring only
calculating Eq.~(\ref{eq:fxysolution}) and~(\ref{eq:fxystep2}) to
obtain the results.

\renewcommand{\algorithmicrequire}{\textbf{Input:}}
\renewcommand{\algorithmicensure}{\textbf{Output:}}
\begin{algorithm}
    % \setstretch{1.2}
    \caption{$f$-$x$-$y$ SPF interpolation algorithm}
    \label{alg:fxyinterp}
    \begin{algorithmic}[1]
        \REQUIRE
        $g_{m,n,l}$, $\mathbf{G}_{m,n,l}$, $\lambda_{f}$, $\lambda_{x}$,
        and $\lambda_{y}$;
        \ENSURE
        $\hat{g}_{m,n,l}$;

        \STATE $ \mathbf{F}_{m-1,n,l}=\mathbf{F}_{m,n-1,l}
            =\mathbf{F}_{m,n,l-1} = \mathbf{0} $
        \FOR {loop over frequency direction}
        \FOR {loop over space $y$ direction}

        \IF {space $y$ == even number}
        \FOR {forward loop over space $x$ direction}
        \IF {encounter known data point}
        \STATE
        $ \mathbf{F}_{m,n,l} = \mathbf{\tilde{F}}_{m,n,l} +
            \frac{ g_{m,n,l} - \mathbf{G}_{m,n,l}^{\mathsf{T}} \mathbf{\tilde{F}}_{m,n,l} }
            { \lambda^{2} + \mathbf{G^{T}}_{m,n,l} \mathbf{G^{*}}_{m,n,l} }
            \mathbf{G}_{m,n,l}^{*} $
        \ELSIF {encounter missing data point}
        \STATE $ \hat{g}_{m,n,l} = \mathbf{G}_{m,n,l}^{\mathsf{T}} \mathbf{F}_{m,n,l} $
        \ENDIF
        \ENDFOR
        \ELSIF {space $y$ == odd number}
        \FOR {backward loop over space $x$ direction}
        \IF {encounter known data point}
        \STATE
        $ \mathbf{F}_{m,n,l} = \mathbf{\tilde{F}}_{m,n,l} +
            \frac{ g_{m,n,l} - \mathbf{G}_{m,n,l}^{\mathsf{T}} \mathbf{\tilde{F}}_{m,n,l} }
            { \lambda^{2} + \mathbf{G^{T}}_{m,n,l} \mathbf{G^{*}}_{m,n,l} }
            \mathbf{G}_{m,n,l}^{*} $
        \ELSIF {encounter missing data point}
        \STATE $ \hat{g}_{m,n,l} = \mathbf{G}_{m,n,l}^{\mathsf{T}} \mathbf{F}_{m,n,l} $
        \ENDIF
        \ENDFOR
        \ENDIF

        \ENDFOR
        \ENDFOR
    \end{algorithmic}
\end{algorithm}

\multiplot{2}{causal3d,noncausal3d}
{width=0.47\columnwidth}
{Proposed filter form (a) and space-noncausal filter form (b) in
    the $f$-$x$-$y$ domain. Solid circle denotes known trace,
    and hollow circle denotes unknown trace.}

\section{Numerical examples}

\subsection{2D data interpolation using the $f$-$x$ SPF}

We generated a 2D synthetic model with one linear event and two curve
events to evaluate the interpolation ability of the $f$-$x$ SPF.  The
curve events bend in opposite directions (Fig.~\ref{fig:mod}
and~\ref{fig:fkmod}), which challenges the adaptability of
interpolation method.  For a missing trace interpolation test
(Fig.~\ref{fig:gap}), we removed $40\%$ of the randomly selected
traces, which caused the appearance of aliasing
(Fig.~\ref{fig:fkgap}).  For comparison, we used the 2D Fourier
Project Onto Convex Sets (POCS) to recover the missing traces
(Fig.~\ref{fig:pocs}). The interpolation result from the 2D Fourier
POCS showed that the linear event was interpolated, but many
discontinuities were present on the curve events.  The interpolated
error was slightly larger in the locations missing traces
(Fig.~\ref{fig:errpocs}), which is caused by strongly variable slopes.
The seislet POCS method \citep{Gan16} shows a clean interpolation
result (Fig.~\ref{fig:stpocs}), but it produces interpolation errors
where the events intersect (Fig.~\ref{fig:errstpocs}).  We designed
the $f$-$x$ SPF with $\lambda_{f}=0.2$, $\lambda_{x}=0.5$, and 30
(space) filter coefficients.  The proposed method provided successful
amplitude preservation, and the missing traces were interpolated
reasonably well (Fig.~\ref{fig:fxspf}).  The difference between the
interpolated and original traces showed that most of the redundant
fluctuations and artifacts were smaller than those created by the
Fourier POCS (Fig.~\ref{fig:errfxspf}).  The $F$-$K$ spectra
(Fig.~\ref{fig:fkpocs,fkstpocs,fkfxspf}) showed that the $f$-$x$ SPF
recovered the missing data and successfully suppressed aliasing.

\inputdir{curve}
\multiplot{4}{mod,fkmod,gap,fkgap}{width=0.47\columnwidth} {Synthetic
  model (a) and $F$-$K$ spectrum (b).  Model with $40\%$ of the data
  traces randomly removed (c) and $F$-$K$ spectrum (d).}

\multiplot{6}{pocs,errpocs,stpocs,errstpocs,fxspf,errfxspf}
{width=0.47\columnwidth} {Reconstructed result (a) and interpolation
  error (b) using the 2D Fourier POCS, reconstructed result (c) and
  interpolation error (d) using the 2D seislet POCS, reconstructed
  result (e) and interpolation error (f) using the 2D $f$-$x$ SPF.}

\multiplot{3}{fkpocs,fkstpocs,fkfxspf} {width=0.47\columnwidth}
{$F$-$K$ spectrum of the interpolation result using the 2D Fourier
  POCS (a), $F$-$K$ spectrum of the interpolation result using the 2D
  seislet POCS (b), $F$-$K$ spectrum of the interpolation result using
  the 2D $f$-$x$ SPF (c).}

Fig.~\ref{fig:sean} shows a marine shot gather from a deepwater Gulf
of Mexico survey with $40\%$ of the data traces randomly removed
(Fig.~\ref{fig:gapsn}).  The $F$-$K$ spectra (Fig.~\ref{fig:fksn}
and~\ref{fig:fkgapsn}) reflect the impact caused by missing data.
Fig.~\ref{fig:pocssn} shows the interpolated result using the 2D
Fourier POCS, which fails to recover steeply dipping events at the
far-offset positions (Fig.~\ref{fig:errpocssn}).  In the interpolation
result of seislet POCS, discontinuities of seismic events are present
(Fig.~\ref{fig:stpocssn}), and some energy remains in the
interpolation error profile (Fig.~\ref{fig:errstpocssn}).  We designed
the $f$-$x$ SPF with $\lambda_{f}=0.05$, $\lambda_{x}=0.1$, and 20
(space) coefficients. Fig.~\ref{fig:fxspfsn} and \ref{fig:errfxspfsn}
show the interpolated result using the proposed method and the
difference between the interpolated and original traces plotted at the
same clip value. The reconstructed data naturally filled the broken
events; meanwhile, the steeply dipping events and diffraction events
were reasonably interpolated.  The $F$-$K$ spectrum of the
interpolation result using the $f$-$x$ SPF (Fig.~\ref{fig:fkfxspfsn})
is similar to that of the original data (Fig.~\ref{fig:fksn}); it
suppresses the low-frequency interference compared to the seislet POCS
(Fig.~\ref{fig:fkstpocssn}), and gives a cleaner spectrum than the
Fourier POCS (Fig.~\ref{fig:fkpocssn}).

\inputdir{sean} 
\multiplot{4}{sean,fksn,gapsn,fkgapsn} {width=0.47\columnwidth} {Field
  data (a) and $F$-$K$ spectrum (b). Data with $40\%$ of the seismic
  traces randomly removed (c) and $F$-$K$ spectrum (d).}

\multiplot{6}{pocssn,errpocssn,stpocssn,errstpocssn,fxspfsn,errfxspfsn}
{width=0.47\columnwidth} {Reconstructed result (a) and interpolation
  error (b) using the 2D Fourier POCS, reconstructed result (c) and
  interpolation error (d) using the 2D seislet POCS, reconstructed
  result (e) and interpolation error (f) using the 2D $f$-$x$ SPF.}

\multiplot{3}{fkpocssn,fkstpocssn,fkfxspfsn} {width=0.47\columnwidth}
{$F$-$K$ spectrum of the interpolation result using the 2D Fourier
  POCS (a), $F$-$K$ spectrum of the interpolation result using the 2D
  seislet POCS (b), $F$-$K$ spectrum of the interpolation result using
  the 2D $f$-$x$ SPF (c).}

\subsection{3D data interpolation using the $f$-$x$-$y$ SPF}

For the 3D data interpolation test, we selected a 3D synthetic model
(Fig.~\ref{fig:qdome}) containing curve events and faults, and the
data cube randomly removed $70\%$ of the seismic traces, where the
faults were hard to distinguish (Fig.~\ref{fig:gapqd}).  The
$F$-$K_{x}$-$K_{y}$ spectrum of the synthetic model and the missing
data model are shown in Fig.~\ref{fig:fkqd} and~\ref{fig:fkgapqd},
respectively. Because currently seislet transform can only handle 2D
datasets, here we evaluated the recovery ability of the $f$-$x$-$y$
SPF by comparing with the 2D seislet POCS method and the 3D Fourier
POCS method.  The parameters of the $f$-$x$-$y$ SPF are
$\lambda_{f}=0.0005$, $\lambda_{x}=0.001$, $\lambda_{y}=0.0008$, and
11 (space $x$) $\times$ 6 (space $y$) filter coefficients.  The 2D
seislet POCS does not reconstruct the missing data well
(Fig.~\ref{fig:stpocsqd}), and generates large interpolation errors
(Fig.~\ref{fig:errstpocsqd}).  The interpolated results
(Fig.~\ref{fig:pocsqd} and \ref{fig:spfqd}) show that both the 3D
Fourier POCS and the $f$-$x$-$y$ SPF can recover the missing traces
even when faults are present. However, the interpolation errors
(Fig.~\ref{fig:errpocsqd} and \ref{fig:errspfqd}) display that the
$f$-$x$-$y$ SPF produces less error and preserves the amplitude of the
events reasonably better than the 3D Fourier POCS.  The comparison of
the $F$-$K_{x}$-$K_{y}$ spectra is shown in
Fig.~\ref{fig:fkstpocsqd,fkpocsqd,fkspfqd}, the $f$-$x$-$y$ SPF
reduces the influence of aliasing.  More importantly, the $f$-$x$-$y$
SPF significantly reduces the computational cost by avoiding the
iterative algorithm especially in higher dimensions.  Compared with
the Fourier POCS, the proposed methods solve the problem without
iterations, which reduces the computational cost
(Table~\ref{tb:cost}).  Table~\ref{tb:time} shows the time consumption
of each method, and the computation platform uses 2.0GHz E5-2650 CPU.

To further evaluate the interpolation ability of the proposed methods,
we defined the signal-to-noise ratio (SNR) as a measurement:
\begin{equation}
    \label{eq:snr}
    SNR = 10 \log_{10} \left(
    \frac{\Arrowvert \mathbf{D} \Arrowvert_{2}^{2}}
    {\Arrowvert\mathbf{D} - \hat{\mathbf{D}}\Arrowvert_{2}^{2}}
    \right),
\end{equation}
where $\mathbf{D}$ denotes original data, and $\hat{\mathbf{D}}$
denotes interpolation result. The traces in the 3D model
(Fig.~\ref{fig:qdome}) have been randomly removed from $5\%$ to
$95\%$. The interpolation results of the $f$-$x$-$y$ SPF are shown in
Fig.~\ref{fig:gap3d-5,intp3d-5,gap3d-95,intp3d-95}.  The proposed SPF
interpolation method in the frequency domain reconstructs data even
under the severely degraded circumstance.  Fig.~\ref{fig:snr3d} shows
that the SPF in higher dimensions can effectively improve the SNR.

\inputdir{qdome} 
\multiplot{4}{qdome,fkqd,gapqd,fkgapqd} {width=0.47\columnwidth}
{Synthetic 3D model (a) and $F$-$K_{x}$-$K_{y}$ spectrum (b).  Model
  with $70\%$ of the data traces randomly removed (c) and
  $F$-$K_{x}$-$K_{y}$ spectrum (d).}

\multiplot{6}{stpocsqd,errstpocsqd,pocsqd,errpocsqd,spfqd,errspfqd}
{width=0.47\columnwidth} {Reconstructed result (a) and interpolation
  error (b) using the 2D seislet POCS, reconstructed result (c) and
  interpolation error (d) using the 3D Fourier POCS, reconstructed
  result (e) and interpolation error (f) using the 3D $f$-$x$-$y$
  SPF.}

\multiplot{3}{fkstpocsqd,fkpocsqd,fkspfqd} {width=0.47\columnwidth}
{$F$-$K_{x}$-$K_{y}$ spectrum of the interpolation result using the 2D
  seislet POCS (a), $F$-$K_{x}$-$K_{y}$ spectrum of the interpolation
  result using the 3D Fourier POCS (b), $F$-$K_{x}$-$K_{y}$ spectrum
  of the interpolation result using the 3D $f$-$x$-$y$ SPF (c).}

\inputdir{snr}
\multiplot{4}{gap3d-5,intp3d-5,gap3d-95,intp3d-95}
{width=0.47\columnwidth} {Model with $5\%$ of the data traces randomly
  removed (a), interpolated data using the 3D $f$-$x$-$y$ SPF
  corresponding to Fig.~\ref{fig:gap3d-5} (b), model with $95\%$ of
  the data traces randomly removed (c), and interpolated data using
  the 3D $f$-$x$-$y$ SPF corresponding to Fig.~\ref{fig:gap3d-95}
  (d).}

\plot{snr3d} {width=0.47\columnwidth} {The SNR changing with different
  degree of trace missing (range from $5\%$ to $95\%$). Dash line is
  the SNR of the model with trace missing and solid line is the SNR of
  interpolation result.}

We used 3D field common mid-point (CMP) gathers after normal moveout
(NMO) to further test the proposed method (Fig.~\ref{fig:gapcmp}).
Fig.~\ref{fig:mask} shows data binning, where the blank space denotes
approximately $75\%$ of the missing traces. A large amount of missing
traces lead to the spatial aliasing artifacts as shown in
Fig.~\ref{fig:fkgapcmp}.  The 2D seislet POCS works well with the
horizontal events, but the data interpolation result is not reasonable
where curve and horizontal events intersect
(Fig.~\ref{fig:stpocscmp}).  The 3D Fourier POCS can recover most
horizontal events, however, some large gaps remain in
Fig.~\ref{fig:pocscmp}.  We chose $\lambda_{f}=6$, $\lambda_{x}=60$,
$\lambda_{y}=15$, and 101 (space $x$) $\times$ 5 (space $y$) filter
coefficients for the $f$-$x$-$y$ SPF.  The design of the filter
coefficients is large, which increases the computational
time-consumption of the $f$-$x$-$y$ SPF.  The interpolation result
using the 3D $f$-$x$-$y$ SPF (Fig.~\ref{fig:spfcmp}) shows that the
missing traces are reasonably reconstructed, the broken events are
well recovered, and the continuity of both linear and curve events are
interpolated well. A close-up comparison at TraceY=8
(Fig.~\ref{fig:stpocszoom}, \ref{fig:pocszoom}, and~\ref{fig:spfzoom})
shows that the $f$-$x$-$y$ SPF can handle more gaps and recover the
amplitude of nonstationary events better than the 2D seislet POCS and
the 3D Fourier POCS.  The $F$-$K_{x}$-$K_{y}$ spectra
(Fig.~\ref{fig:fkgapcmp,fkstpocscmp,fkpocscmp,fkspfcmp}) demonstrate
that the $f$-$x$-$y$ SPF can handle aliasing, and the energy is
converged in the $F$-$K_{x}$-$k_{y}$ spectrum
(Fig.~\ref{fig:fkspfcmp}).

\begin{table}
    % \setlength{\arraycolsep}{0.0em}
    \renewcommand\tabcolsep{3.5pt} 
    \renewcommand{\arraystretch}{1.3} 
    \caption{
        Comparison of computational cost: the $f$-$x$ SPF, the 2D
        Fourier POCS, the $f$-$x$-$y$ SPF, and the 3D Fourier POCS. }
    \label{tb:cost}
    \centering
    \begin{threeparttable}
        \begin{tabular}{|c|c|c|}
            \toprule
                            & Storage space
                            & Computational time                                  \\
            \midrule
            2D Fourier POCS & $ O(N_{f}N_{k_{x}}) $
                            & $ O(N_{t}N_{x}\log(N_{t}N_{x}) N_{iter}) $          \\
            $f$-$x$ SPF     & $ O(N_{a}N_{x}) $
                            & $ O(N_{a}N_{f}N_{x}) $                              \\
            3D Fourier POCS & $ O(N_{f}N_{k_{x}}N_{k_{y}}) $
                            & $ O(N_{t}N_{x}N_{y}\log(N_{t}N_{x}N_{y})N_{iter}) $ \\
            $f$-$x$-$y$ SPF & $ O(N_{a}N_{x}N_{y}) $
                            & $ O(N_{a}N_{f}N_{x}N_{y}) $                         \\
            \bottomrule
        \end{tabular}

        \begin{tablenotes}[para, flushleft]
            % \footnotesize
            % \item\textit{Note:}
            $N_{t}$ is the data size along the time axis,
            $N_{f}$ is the data size along the frequency axis,
            $N_{x}$ and $N_{y}$ are the data size along the space
            $x$ and $y$ axis, respectively.
            $N_{k_{x}}$ and $N_{k_{y}}$ are the data size along
            the wavenumber $k_{x}$ and $k_{y}$ axis, respectively.
            $N_{a}$ is the filter size,
            and $N_{iter}$ is the number of iterations.

        \end{tablenotes}
    \end{threeparttable}

\end{table}

\begin{table}
    % \setlength{\arraycolsep}{0.0em}
    \renewcommand\tabcolsep{3.5pt} 
    \renewcommand{\arraystretch}{1.3} 
    \caption{Comparison of time consumption.}
    \label{tb:time}
    \centering
    \resizebox{\textwidth}{20mm}{
        \begin{threeparttable}
            \begin{tabular}{|c|c|c|c|c|}
                \toprule
                             & 2D synthetic model   & 2D field data        & 3D synthetic model      & 3D field data           \\
                Model size   & $501 \times 201$     & $500 \times 180$     & $200\times151\times100$ & $1000\times400\times16$ \\
                \midrule
                seislet POCS & $4.23\times10^{1}$s  & $3.89\times10^{1}$s  & $1.81\times10^{3}$s     & $7.41\times10^{3}$s     \\
                Fourier POCS & $3.15\times10^{0}$s  & $3.55\times10^{0}$s  & $1.03\times10^{2}$s     & $4.61\times10^{2}$s     \\
                SPF          & $4.00\times10^{-1}$s & $2.68\times10^{-1}$s & $3.38\times10^{1}$s     & $3.63\times10^{2}$s     \\
                \bottomrule
            \end{tabular}
        \end{threeparttable}}
\end{table}


\inputdir{cmp}
\multiplot{2}{gapcmp,mask} {width=0.47\columnwidth} {3D field data
  with trace missing (a) and data binning of the field data (b).}

\multiplot{2}{stpocscmp,stpocszoom} {width=0.47\columnwidth}
{Reconstructed result using the 2D seislet POCS (a), and close-up of
  the interpolated result (b).}

\multiplot{2}{pocscmp,pocszoom} {width=0.47\columnwidth}
{Reconstructed result using the 3D Fourier POCS (a), and close-up of
  the interpolated result (b).}

\multiplot{2}{spfcmp,spfzoom} {width=0.47\columnwidth} {Reconstructed
  result using the 3D $f$-$x$-$y$ SPF (a), and close-up of the
  interpolated result (b).}

\multiplot{4}{fkgapcmp,fkstpocscmp,fkpocscmp,fkspfcmp}
{width=0.47\columnwidth} {The $F$-$K_{x}$-$K_{y}$ spectrum of the 3D
  field data (a), the $F$-$K_{x}$-$K_{y}$ spectrum of the
  reconstructed result using the 2D seislet POCS (b), the
  $F$-$K_{x}$-$K_{y}$ spectrum of the reconstructed result using the
  3D Fourier POCS (c), the $F$-$K_{x}$-$K_{y}$ spectrum of the
  reconstructed result using the 3D $f$-$x$-$y$ SPF (d).}

\section{Discussion}

We first discussed the value selection of the weights $\lambda_{f}$,
$\lambda_{x}$, and $\lambda_{y}$.  Currently, we considered them as
empirical parameters, but there are still some ways to determine their
value range.  1) In Eq.~(\ref{eq:fxsolution})
and~(\ref{eq:fxysolution}), $\lambda^2$ and
$\mathbf{G}_{m,n,l}^{\mathsf{T}}\mathbf{G}_{m,n,l}^{*}$ (or
$\mathbf{G}_{m,n}^{\mathsf{T}}\mathbf{G}_{m,n}^{*}$) are the
denominator of the filter update term.  It is expected that
$\lambda^{2}$ can influence the filter update, so $\lambda^2$ and
$\mathbf{G}_{m,n,l}^{\mathsf{T}}\mathbf{G}_{m,n,l}^{*}$ (or
$\mathbf{G}_{m,n}^{\mathsf{T}}\mathbf{G}_{m,n}^{*}$) should have a
similar order of magnitude, where $\lambda^{2} =
\lambda_{f}^{2}+\lambda_{x}^{2}+\lambda_{y}^{2}$ (or $\lambda^{2} =
\lambda_{f}^{2}+\lambda_{x}^{2}$).  2) In terms of the prediction
filtering theory in the frequency domain, the filter predicts the data
sample along the spatial direction, not along the frequency
direction. $\lambda_{x}$ and $\lambda_{y}$ in the spatial direction
should be similar, and their values can be scaled according to the
size of dataset. $\lambda_{f}$ mainly stabilizes the filter in the
frequency direction, and it generally smaller than $\lambda_{x}$ and
$\lambda_{y}$.  3) The regularization terms' weight affect the data
interpolation result to a certain extent. The values of $\lambda_{f}$,
$\lambda_{x}$, and $\lambda_{y}$ can be set according to the previous
method, and then they can be adjusted according to the interpolation
effect.

For the case of regular decimation, we tested the synthetic 2D and 3D
model in Fig.~\ref{fig:alias,intp} and~\ref{fig:aliasqd,intpqd}.  In
the 2D case, we used ten seismic traces to initialize the $f$-$x$ SPF,
in which the filter could reconstruct the missing data.  As shown in
Fig.~\ref{fig:intp}, the artifacts affect the data interpolation
quality for traces with large slope differences between the seismic
events.  For the 3D data (Fig.~\ref{fig:aliasqd,intpqd}), the
$f$-$x$-$y$ SPF can directly recover the missing traces. Although the
regular decimation of the 3D model has strong spatial aliasing, we
still obtained a reasonable interpolation result.

\inputdir{curve}
\multiplot{2}{alias,intp} {width=0.47\columnwidth} {Regular decimation
  of synthetic 2D model (a), and interpolated result using the 2D
  $f$-$x$ SPF (b).}

\inputdir{qdome}
\multiplot{2}{aliasqd,intpqd} {width=0.47\columnwidth} {Regular
  decimation of synthetic 3D model (a), and interpolated result using
  the 3D $f$-$x$-$y$ SPF (b).}

%% low SNR
% curve
We also tested the effectiveness of the proposed method in the case of
low SNR.  We added stronger noise to both the 2D and 3D synthetic
model (Fig.~\ref{fig:noisemod} and~\ref{fig:noiseqdome}), which both
had randomly decimated seismic traces.  The strong random noise
influenced the local slope calculation, which further affected the
seislet transform.  By using the 2D seislet POCS, the interpolation
result of the 2D model (Fig.~\ref{fig:noisestpocs}
and~\ref{fig:noiseerrstpocs}) shows some smearing, and the 3D model
cannot be reconstructed (Fig.~\ref{fig:noisestpocsqd}
and~\ref{fig:noiseerrstpocsqd}).  For the 2D Fourier POCS method,
there are some parts of the upward curve that are not recovered in the
2D model (Fig.~\ref{fig:noisepocs}
and~\ref{fig:noiseerrpocs}). Additionally, the 3D Fourier POCS method
produces a reasonable reconstruction result
(Fig.~\ref{fig:noisepocsqd}), although leakage signal of the seismic
events presents in the interpolation error profile
(Fig.~\ref{fig:noiseerrpocsqd}).  Because the PF can be used to
attenuate random noise, and it may reduce the impact of random noise
to some extent, our proposed methods (the $f$-$x$ SPF and the
$f$-$x$-$y$ SPF) yield better reconstruction results
(Fig.~\ref{fig:noisefxspf},~\ref{fig:noiseerrfxspf},
\ref{fig:noisespfqd}, and~\ref{fig:noiseerrspfqd}) under low SNR than
other methods.

\inputdir{noise_curve}
\multiplot{2}{noisemod,noisegap} {width=0.47\columnwidth} {Synthetic
  model with random noise (a), model with 40\% of the data traces
  randomly removed (b).}

\multiplot{6}{noisestpocs,noiseerrstpocs,noisepocs,noiseerrpocs,noisefxspf,noiseerrfxspf}
{width=0.47\columnwidth} {Reconstructed result (a) and interpolation
  error (b) using the 2D seislet POCS, Reconstructed result (c) and
  interpolation error (d) using the 2D Fourier POCS, reconstructed
  result (e) and interpolation error (f) using the 2D $f$-$x$ SPF.}

\inputdir{noise_qdome}
\multiplot{2}{noiseqdome,noisegapqd} {width=0.47\columnwidth}
{Synthetic 3D model with random noise (a), model with $70\%$ of the
  data traces randomly removed (b).}

\multiplot{6}{noisestpocsqd,noiseerrstpocsqd,noisepocsqd,noiseerrpocsqd,noisespfqd,noiseerrspfqd}
{width=0.47\columnwidth} {Reconstructed result (a) and interpolation
  error (b) using the 2D seislet POCS, reconstructed result (c) and
  interpolation error (d) using the 3D Fourier POCS, reconstructed
  result (e) and interpolation error (f) using the 3D $f$-$x$-$y$
  SPF.}

\section{Conclusion}

We introduced a fast approach to adaptive PF for missing data
interpolation in the frequency domain.  Instead of using the iterative
optimization algorithm, we proposed a two-step interpolation strategy
with noniterative SPF in the $f$-$x$ and $f$-$x$-$y$ domains.  The
proposed method employs a local and multidimensional similarity to
constrain the autoregression equations for adaptive PFs in the
frequency domain, which are based on the streaming computation
framework. The SPF in the frequency domain provides a fast and
reasonably accurate estimation of nonstationary seismic data.  To
guarantee the interpolation results, we also designed the filter
structure and the processing path according to the characteristics of
the interpolation problem. The synthetic and field examples show that
the proposed SPF in the frequency domain can depict nonstationary
signal variation and provide a reliable description of complex
wavefield with low computational cost even when analyzing large-scale
seismic data. The properties are suitable for missing data
interpolation in practice. Finally, we discussed the problems of
parameter selection, interpolation of regularly decimated data, and
interpolation of low SNR data; the proposed methods can cope with such
problems.

\section{Acknowledgement}
This work was supported by the National Natural Science Foundation of
China (grant nos. 41974134 and 41774127) and the National key Research
and Development Program of China (grant no. 2018YFC0603701).


\bibliographystyle{seg}
\bibliography{paper}
