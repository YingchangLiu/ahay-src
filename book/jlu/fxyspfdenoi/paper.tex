\published{IEEE Transactions on Geoscience and Remote Sensing, 60, 1-9, Early Access, (2022)}

\title{Noniterative f-x-y streaming prediction filtering for random noise attenuation on seismic data}

\renewcommand{\thefootnote}{\fnsymbol{footnote}}

\ms{IEEE-TGRS-2021-00406}

\address{
    \footnotemark[1] College of Geo-exploration Science and Technology,\\
    Jilin University, Changchun, China}

\author{Yang Liu\footnotemark[1], Zhisheng Zheng\footnotemark[1]}

\footer{IEEE-TGRS-2021-00406}
\lefthead{Liu and Zheng}
\righthead{Noise attenuation by the $f$-$x$-$y$ SPF}

\maketitle

\begin{abstract}
  Random noise is unavoidable in seismic exploration, especially under
  complex-surface conditions and in deep-exploration environments. The
  current problems in random noise attenuation include preserving the
  nonstationary characteristics of the signal and reducing
  computational cost of broadband, wide-azimuth, and high-density data
  acquisition. To obtain high-quality images, traditional prediction
  filters (PFs) have proved effective for random noise attenuation,
  but these methods typically assume that the signal is
  stationary. Most nonstationary PFs use an iterative strategy to
  calculate the coefficients, which leads to high computational
  costs. In this study, we extended the streaming prediction theory to
  the frequency domain and proposed the $f$-$x$-$y$ streaming
  prediction filter (SPF) to attenuate random noise.  Instead of using
  the iterative optimization algorithm, we formulated a constraint
  least-squares problem to calculate the SPF and derived an analytical
  solution to this problem.  The multi-dimensional streaming
  constraints are used to increase the accuracy of the SPF. We also
  modified the recursive algorithm to update the SPF with the snaky
  processing path, which takes full advantage of the streaming
  structure to improve the effectiveness of the SPF in high
  dimensions. In comparison with 2D $f$-$x$ SPF and 3D $f$-$x$-$y$
  regularized nonstationary autoregression (RNA), we tested the
  practicality of the proposed method in attenuating random
  noise. Numerical experiments show that the 3D $f$-$x$-$y$ SPF is
  suitable for large-scale seismic data with the advantages of low
  computational cost, reasonable nonstationary signal protection, and
  effective random noise attenuation.

\end{abstract}

\section{Introduction}

In seismic exploration, random noise is unavoidable because it is
composed of environmental noise, the interference of wind motion, and
the noise from geophones \cite{Yilmaz01}. Meanwhile, complex
subsurface media typically cause the energy loss experienced by
seismic signals, which show low amplitude in deep exploration
conditions. These factors result in low-quality data and make
signal-to-noise ratio (SNR) gradually low. In obtaining high-quality
seismic images and improving the SNR, one key problem is the
nonstationary characteristics of the signal. Seismic data are
time-varying in nature, and the nonstationary properties of seismic
data display that energy, track, time-frequency spectra of seismic
events, and statistical characteristics of random noise change with
time and space. The denoising methods that consider such nonstationary
features can better preserve the valid signals.  The other problem
with denoising is the increasing computational costs, although
broadband, wide-azimuth, and high-density data acquisition can lead to
high-resolution and high-fidelity images.  Many authors have proposed
methods for random noise attenuation based on different theories. The
mean filter \cite{Bonar12} and median filter \cite{Liu09, Wu18} are
effective denoising methods for images, but they may somewhat smear
seismic signals when complicated structures and low SNR are
encountered. Mathematical transforms such as wavelet transforms
\cite{Berkner98, Yu07, Langston19} and seislet transforms
\cite{Fomel10, Liu10, Liu15vd} can characterize the nonstationary
properties of seismic signals and provide reasonable signal and noise
separation based on their compression ability. Recently, deep learning
or machine learning techniques \cite{Djarfour14, Kimiarfar16, Yu19,
  Zhu19} have also been proposed to suppress random noise; however,
the initialization of neural networks requires a large number of
samples and high computational cost. For supervised and
semi-supervised learning, the preparation of the training set may need
the help of traditional methods to generate denoised results as a
reference.

Prediction filters (PFs) have proved effective for random noise
attenuation, and they can be implemented in the time-space or
frequency-space domain. When seismic events have varying slope, the
configuration of the filter size influences the filtering results,
especially the filter size of $t$-$x$ PFs along the time axis. There
are few impacts on $f$-$x$ PFs because they only estimate data along
the spatial directions. Besides, seismic events with different
dominant frequencies are overlapped in the $t$-$x$ domain, and they
can be naturally separated in the $f$-$x$ domain. Abma and
Claerbout~\cite{Abma95} discussed the differences in PFs in the
$f$-$x$ and $t$-$x$ domains.  The $f$-$x$ prediction filter for
denoising was first introduced by Canales~\cite{Canales84}, and
further developed by G{\"u}l{\"u}nay~\cite{Gulunay86} to a standard
industry method known as “FXDECON”, which is equivalent to a $t$-$x$
domain prediction filter, selecting the entire trace along the time
direction.  Liu~\textit{et al.}~\cite{Liu12} developed the $f$-$x$
adaptive prediction filters to suppress random noise by using
regularized nonstationary autoregression (RNA); the regularization
term was used to limit the global smoothness of the filter
coefficients. The method was further extended from a two-dimensional
(2D) to three-dimensional (3D) case for random noise attenuation
\cite{Liug13}. The $f$-$x$-$y$ RNA provides preferable adaptive
features because it uses an iterative algorithm to calculate the
frequency-space-varying filter coefficients, which leads to a large
storage and high computational time, especially in large-scale data
processing.

Local similarity constraints have been proposed to directly calculate
the adaptive prediction filter without iterations, which can save the
computational resources. Starting with the prediction equation for a
certain data point, Sacchi and Naghizadeh~\cite{Sacchi09} transformed
the ill-posed problem of adaptive prediction filter into a local
smoothing problem, and introduced a quadratic regularization term to
stabilize the solution of the local prediction filter.  Fomel and
Claerbout~\cite{Fomel16} proposed the concept of streaming prediction
error filter (SPEF) to update the filter as each new data value
arrives in the time-space domain. This method combines the prediction
equation with locally similar constraints to solve the overdetermined
linear system. Arising from different starting points, these two
methods share the same least-squares solution and significantly reduce
the computational cost by avoiding the iterative algorithm.  Liu and
Li~\cite{Liuy18} further proposed a streaming orthogonal prediction
filter (SOPF) in the $t$-$x$ domain, which applies signal-and-noise
orthogonalization based on the streaming prediction theory and
provides a fast solution for the adaptive prediction filter to
suppress random noise. Guo~\textit{et al.}~\cite{Guo20} attempted to
eliminate seismic random noise by using the $f$-$x$ SPF only with 1D
spatial constraint.

In this study, we derived the theory of the new $f$-$x$-$y$ streaming
prediction filter based on the local smoothness constraints in high
dimensions. The multi-dimensional constraints make the filter involve
the property of local similarity not only along the spatial directions
(space $x$ and space $y$), but also the frequency direction.  It is
not a simple 3D extension with more spatial axis from 2D
SPF~\cite{Guo20}, we took advantage of 3D data with space $y$
constraint and suppressed oscillation with frequency constraint,
meanwhile, a special filter-updating path was developed to help the
SPF solve the random noise attenuation problem in higher
dimensions. We compared the feasibility of the 3D $f$-$x$-$y$ SPF in
attenuating random noise with the 2D $f$-$x$ SPF and the 3D
$f$-$x$-$y$ RNA on two synthetic models. The field data example
confirms that the 3D $f$-$x$-$y$ SPF with the matching processing path
has a reasonable denoising ability and a low computational cost in
practice.

\section{Theory}

\subsection{3D $f$-$x$-$y$ streaming prediction filtering}

A 2D seismic section $s(t,x)$ containing linear events can be
described as a plane wave function in the $t$-$x$ domain.  In the
$f$-$x$ domain, the linear events in seismic section $\tilde{S}(f,x)$
are decomposed into a series of sinusoids.  These sinusoids are
superimposed and become harmonics at each frequency, which shows the
prediction relationship of seismic traces in a frequency slice:
\begin{equation}
    \label{eq:plane}
    \sum_{p=1}^{P}a_{m,p}\tilde{S}_{m,n-p} = \tilde{S}_{m,n},
\end{equation}
where $m \in [1,M]$ and $n \in [1,N]$ are the indices of the seismic
sample along the $f$ axis and $x$ axis, respectively. $p \in [1,P]$ is
the index of filter coefficients along the $x$ direction.
$\tilde{S}_{m,n}$ denotes the data point in $\tilde{S}(f,x)$ and
$a_{m,p}$ indicates the filter coefficient in the $f$-$x$ domain. When
curve events or amplitude-varying wavelets are shown in the seismic
data, filter coefficients change from one data point to the next,
which help to manage the nonstationary case:
\begin{equation}
    \label{eq:curve}
    \sum_{p=1}^{P}a_{m,n,p}\tilde{S}_{m,n-p}
    = \mathbf{S^{T}}_{m,n} \mathbf{A}_{m,n}
    = \tilde{S}_{m,n},
\end{equation}
where $\{ \mathbf{T} \}$ denotes the transpose operator,
$\mathbf{S}_{m,n}=[\tilde{S}_{m,n-1}, \\
\tilde{S}_{m,n-2}, \cdots, \tilde{S}_{m,n-p}]^{\mathbf{T}}$ denotes
the vector including the data points near $\tilde{S}_{m,n}$.
$\mathbf{A}_{m,n}=[a_{m,n,1}, a_{m,n,2}, \cdots,
a_{m,n,p}]^{\mathbf{T}} $ is the vector of coefficients in a 2D
adaptive prediction filter. Let $P=3$, Fig.~\ref{fig:fig1} illustrates
how (\ref{eq:curve}) works.  Equation~(\ref{eq:curve}) denotes that
the filter predicts data point along the spatial direction rather than
the frequency direction. Therefore, an extension to the 3D $f$-$x$-$y$
domain is straightforward:
\begin{equation}
    \label{eq:curve3d}
    \begin{aligned}
         & \sum_{p=-P}^{P}\sum_{q=-Q}^{Q}a_{m,n,l,p,q}\tilde{S}_{m,n-p,l-q} \\
         & = \mathbf{S^{T}}_{m,n,l} \mathbf{A}_{m,n,l}
        = \tilde{S}_{m,n,l} \quad (|p|+|q| \neq 0),
    \end{aligned}
\end{equation}
where $l \in [1,L]$ is the index of the data sample along the $y$
axis. $p \in [-P,P]$ and $q \in [-Q,Q]$ are the indices of filter
coefficients in two spatial directions,
$\mathbf{A}_{m,n,l}=[a_{m,n,l,-P,-Q}, \cdots, a_{m,n,l,p,q}, \cdots,
a_{m,n,l,P,Q}]^{\mathbf{T}} $ is the vector of 3D filter
coefficients. The adaptive prediction filter $\mathbf{A}_{m,n,l}$ is
defined as a space-noncausal structure and the filter size in the
spatial direction is $(2P+1)\times(2Q+1)-1$. The vector $
\mathbf{S}_{m,n,l}= [\tilde{S}_{m,n+P,l+Q}, \cdots,
\tilde{S}_{m,n-p,l-q}, \tilde{S}_{m,n-P,l-Q}]^{\mathbf{T}} $ contains
the data points near $\tilde{S}_{m,n,l}$.  Let $p=\{-2,-1,0,1,2\}$ and
$q=\{-2,-1,0,1,2\}$, Fig.~\ref{fig:fig3} demonstrates the distribution
of the vectors $\mathbf{S}_{m,n,l}$ and $\mathbf{A}_{m,n,l}$. Assuming
that the contained noise is white Gaussian noise, the filter can be
obtained by solving the minimization problem:
\begin{equation}
    \label{eq:LS1}
    \min_{\mathbf{A}_{m,n,l}} \left \| \mathbf{S^{T}}_{m,n,l}
    \mathbf{A}_{m,n,l} - \tilde{S}_{m,n,l} \right \| _{2}^{2},
\end{equation}
equation~(\ref{eq:LS1}) describes an ill-posed problem that the number
of the unknown filter coefficients is greater than that of the known
equations. Without any regularization, the equation will lead to an
unstable solution:
\begin{equation}
    \label{eq:solution1}
    \mathbf{A}_{LS}=(\mathbf{S^{*}}_{m,n,l} \mathbf{S^{T}}_{m,n,l})^{-1}
    \mathbf{S^{*}}_{m,n,l} \tilde{S}_{m,n,l},
\end{equation}
where $\{ * \}$ denotes the conjugate operator.

To solve the underdetermined problem (\ref{eq:LS1}), constraint
conditions based on local similarity/smoothness are used to stabilize
the solution of (\ref{eq:LS1}).  Assuming that the adaptive prediction
filter at position $(m,n,l)$ is similar to another one at position
$(m,n-1,l)$ in the $f$-$x$-$y$ domain, $ \lambda_{x}
\mathbf{A}_{m,n,l} \approx \lambda_{x} \mathbf{A}_{m,n-1,l} $ can be
treated as the constraint condition on the $x$ axis.  The
autoregression equation can be expressed as follows:
\begin{small}
    \begin{equation}
        \label{eq:matrixpointnorm}
        \begin{aligned}
            \begin{bmatrix}
                \tilde{S}_{m,n+P,l+Q} & \cdots                & \tilde{S}_{m,n-p,l-q} &
                \cdots                & \tilde{S}_{m,n-P,l-Q}                                                \\
                \lambda_{x}           &                       &                       &        &             \\
                                      & \ddots                &                       &        &             \\
                                      &                       & \lambda_{x}           &        &             \\
                                      &                       &                       & \ddots &             \\
                                      &                       &                       &        & \lambda_{x}
            \end{bmatrix}
            \begin{bmatrix}
                a_{m,n,l,-P,-Q} \\
                \vdots          \\
                a_{m,n,l,p,q}   \\
                \vdots          \\
                a_{m,n,l,P,Q}
            \end{bmatrix}
            \approx
            \begin{bmatrix}
                \tilde{S}_{m,n,l}             \\
                \lambda_{x} a_{m,n-1,l,-P,-Q} \\
                \vdots                        \\
                \lambda_{x} a_{m,n-1,l,p,q}   \\
                \vdots                        \\
                \lambda_{x} a_{m,n-1,l,P,Q}
            \end{bmatrix},
        \end{aligned}
    \end{equation}
\end{small}
and the simplified block matrix can be written as:
\begin{equation}
    \label{eq:matrixpointnormsimple}
    \begin{bmatrix}
        \mathbf{S^{T}}_{m,n,l} \\
        \lambda_{x} \mathbf{I}
    \end{bmatrix}
    \mathbf{A}_{m,n,l}
    \approx
    \begin{bmatrix}
        \tilde{S}_{m,n,l} \\
        \lambda_{x} \mathbf{A}_{m,n-1,l}
    \end{bmatrix}.
\end{equation}
Equation~(\ref{eq:matrixpointnormsimple}) is solvable since there are
$(2P+1)*(2Q+1)$ equations with $(2P+1)*(2Q+1)-1$ unknown coefficients,
which correspond to the following minimization problem:
\begin{equation}
    \label{eq:LS+x}
    \min_{\mathbf{A}_{m,n,l}} \left \|
    \mathbf{S^{T}}_{m,n,l} \mathbf{A}_{m,n,l}
    - \tilde{S}_{m,n,l} \right \|_{2}^{2}
    + \lambda_{x}^{2} \left \| \mathbf{A}_{m,n,l} - \mathbf{A}_{m,n-1,l}
    \right \|_{2}^{2},
\end{equation}
where $\lambda_{x}$ is the constant weight for the regularization term
along the $x$ axis. In the frequency $f$ direction, one can assume
that the SPFs change smoothly and treat the irregular perturbations as
the interference of noise. Meanwhile, the smoothness of the 3D SPFs
also exists in different spatial directions and may change at
different data point, therefore, we implemented local smoothness along
$f$, $x$, and $y$ axis as constraints to calculate the $f$-$x$-$y$
SPF. The block matrix form is:
\begin{equation}
    \label{eq:matrix2}
    \begin{bmatrix}
        \mathbf{S^{T}}_{m,n,l}        \\
        \lambda_{f}(m,n,l) \mathbf{I} \\
        \lambda_{x}(m,n,l) \mathbf{I} \\
        \lambda_{y}(m,n,l) \mathbf{I}
    \end{bmatrix}
    \mathbf{A}_{m,n,l} =
    \begin{bmatrix}
        \tilde{S}_{m,n,l}                       \\
        \lambda_{f}(m,n,l) \mathbf{A}_{m-1,n,l} \\
        \lambda_{x}(m,n,l) \mathbf{A}_{m,n-1,l} \\
        \lambda_{y}(m,n,l) \mathbf{A}_{m,n,l-1}
    \end{bmatrix}
\end{equation}
and the corresponding least-squares problem takes the
following form:
\begin{equation}
    \label{eq:LS2}
    \begin{aligned}
         & \min_{\mathbf{A}_{m,n,l}} \left \|
        \mathbf{S^{T}}_{m,n,l} \mathbf{A}_{m,n,l}
        - \tilde{S}_{m,n,l} \right \|_{2}^{2}                                          \\
         & + \lambda_{f}^{2}(m,n,l) \left \| \mathbf{A}_{m,n,l} - \mathbf{A}_{m-1,n,l}
        \right \|_{2}^{2}                                                              \\
         & + \lambda_{x}^{2}(m,n,l) \left \| \mathbf{A}_{m,n,l} - \mathbf{A}_{m,n-1,l}
        \right \|_{2}^{2}                                                              \\
         & + \lambda_{y}^{2}(m,n,l) \left \| \mathbf{A}_{m,n,l} - \mathbf{A}_{m,n,l-1}
        \right \|_{2}^{2},
    \end{aligned}
\end{equation}
where $\lambda_{f}(m,n,l)$, $\lambda_{x}(m,n,l)$ and
$\lambda_{y}(m,n,l)$ denotes the variable weights of regularization
terms along the frequency $f$ axis, space $x$ axis, and space $y$
axis, respectively. They measure the variable similarity or closeness
between the filter $\mathbf{A}_{m,n,l}$ and the adjacent filters
$\mathbf{A}_{m-1,n,l}$, $\mathbf{A}_{m,n-1,l}$, and
$\mathbf{A}_{m,n,l-1}$.  Due to the prediction filter can characterize
the energy spectrum of the input data \cite{Claerbout76}, the adaptive
filter shares analogical smoothness property with the 3D data, so the
variation of the weights may consist with the smooth version of data.
For simplicity, we select these weights with constant value,
$\lambda_{f}(m,n,l) = \lambda_{f}$, $\lambda_{x}(m,n,l) = \lambda_{x}$
and $\lambda_{y}(m,n,l) = \lambda_{y}$, to demonstrate the constrained
relationship.  The introduced regularization terms convert the
ill-posed problem to the overdetermined inverse problem, and the
least-squares solution of (\ref{eq:matrix2}) and (\ref{eq:LS2}) is:
\begin{small}
    \begin{equation}
        \label{eq:solution2}
        \begin{aligned}
            \mathbf{A}_{m,n,l} & = [ (\lambda_{f}^{2} + \lambda_{x}^{2} + \lambda_{y}^{2})
            \mathbf{I} + \mathbf{S^{*}}_{m,n,l} \mathbf{S^{T}}_{m,n,l} ]^{-1}
            ( \lambda_{f}^{2} \mathbf{A}_{m-1,n,l}                                         \\
                               & + \lambda_{x}^{2} \mathbf{A}_{m,n-1,l}
            + \lambda_{y}^{2} \mathbf{A}_{m,n,l-1}
            + \tilde{S}_{m,n,l} \mathbf{S^{*}}_{m,n,l})                                    \\
                               & = (\lambda^{2} \mathbf{I}
            + \mathbf{S^{*}}_{m,n,l} \mathbf{S^{T}}_{m,n,l} )^{-1}
            (\lambda^{2}\mathbf{\hat{A}}_{m,n,l} + \tilde{S}_{m,n,l} \mathbf{S^{*}}_{m,n,l}).
        \end{aligned}
    \end{equation}
\end{small}
where
\begin{small}
    \begin{equation}
        \label{eq:coef}
        \begin{cases}
            \lambda^{2} = \lambda_{f}^{2} + \lambda_{x}^{2} + \lambda_{y}^{2}, \\
            \mathbf{\hat{A}}_{m,n,l} = ( \lambda_{f}^{2} \mathbf{A}_{m-1,n,l} +
            \lambda_{x}^{2} \mathbf{A}_{m,n-1,l} +
            \lambda_{y}^{2} \mathbf{A}_{m,n,l-1} ) / \lambda^{2}.
        \end{cases}
    \end{equation}
\end{small}
The Sherman-Morrison formula is an analytic method for solving the
inverse matrix \cite{Hager89}:
\begin{small}
    \begin{equation}
        \label{eq:inverse}
        ( \lambda^{2}\mathbf{I} + \mathbf{S^{*}}_{m,n,l} \mathbf{S^{T}}_{m,n,l} )^{-1}
        = \frac{ 1 } { \lambda^{2}  }
        ( \mathbf{I} - \frac{ \mathbf{S^{*}}_{m,n,l} \mathbf{S^{T}}_{m,n,l} }
        { \lambda^{2} + \mathbf{S^{T}}_{m,n,l} \mathbf{S^{*}}_{m,n,l} }).
    \end{equation}
\end{small}
The derivation of the Sherman-Morrison formula in the complex space is
described in Appendix~\ref{S-M-F}.  Elementary algebraic
simplifications lead to the analytical solution:
\begin{small}
    \begin{equation}
        \label{eq:filter}
        \begin{aligned}
            \mathbf{A}_{m,n,l}
             & = ( \lambda^{2} \mathbf{I}
            + \mathbf{S^{*}}_{m,n,l} \mathbf{S^{T}}_{m,n,l} )^{-1}
            ( \tilde{S}_{m,n,l} \mathbf{S^{*}}_{m,n,l}
            + \lambda^{2} \mathbf{\hat{A}}_{m,n,l})                          \\
             & = \frac{ 1 } { \lambda^{2} } ( \mathbf{I} -
            \frac{ \mathbf{ S^{*}}_{m,n,l} \mathbf{S^{T} }_{m,n,l} }
            { \lambda^{2} + \mathbf{S^{T}}_{m,n,l} \mathbf{S^{*}}_{m,n,l} })
            ( \tilde{S}_{m,n,l} \mathbf{S^{*}}_{m,n,l}
            + \lambda^{2} \mathbf{\hat{A}}_{m,n,l} )                         \\
             & = \mathbf{\hat{A}}_{m,n,l} + \frac{ 1 } { \lambda^{2} }
            \tilde{S}_{m,n,l} \mathbf{S^{*}}_{m,n,l} - \frac{ 1 } {\lambda^{2}}
            \frac{ \mathbf{S^{*}}_{m,n,l} \mathbf{ S^{T}}_{m,n,l}
                \tilde{S}_{m,n,l} \mathbf{S^{*}}_{m,n,l} }
            { \lambda^{2} + \mathbf{ S^{T}}_{m,n,l} \mathbf{S^{*}}_{m,n,l} } \\
             & - \frac{\mathbf{S^{*}}_{m,n,l} \mathbf{S^{T}}_{m,n,l}
                \mathbf{\hat{A}}_{m,n,l} }
            { \lambda^{2} + \mathbf{S^{T}}_{m,n,l} \mathbf{S^{*}}_{m,n,l} }  \\
             & = \mathbf{\hat{A}}_{m,n,l} + \frac{ 1 } { \lambda^{2} }
            \frac{ \lambda^{2} \tilde{S}_{m,n,l} \mathbf{S^{*}}_{m,n,l} +
                \tilde{S}_{m,n,l} \mathbf{ S^{*}}_{m,n,l} \mathbf{S^{T}}_{m,n,l}
                \mathbf{S^{*}}_{m,n,l} } { \lambda^{2} + \mathbf{S^{T}}_{m,n,l}
            \mathbf{S^{*}}_{m,n,l} }                                         \\
             & - \frac{ 1 } { \lambda^{2} }
            \frac{ \mathbf{S^{*}}_{m,n,l}  \mathbf{S^{T}}_{m,n,l} \tilde{S}_{m,n,l}
                \mathbf{S^{*}}_{m,n,l} } { \lambda^{2} +
                \mathbf{S^{T}}_{m,n,l} \mathbf{S^{*}}_{m,n,l} }
            - \frac{ \mathbf{S^{*}}_{m,n,l} \mathbf{S^{T}}_{m,n,l}
                \mathbf{\hat{A}}_{m,n,l} } { \lambda^{2} + \mathbf{ S^{T}}_{m,n,l}
            \mathbf{S^{*}}_{m,n,l} }                                         \\
             & = \mathbf{\hat{A}}_{m,n,l} + \frac{ \tilde{S}_{m,n,l} -
                \mathbf{S^{T}}_{m,n,l} \mathbf{\hat{A}}_{m,n,l}  }
            { \lambda^{2} + \mathbf{ S^{T}}_{m,n,l}
                \mathbf{S^{*}}_{m,n,l} } \mathbf{ S^{*}}_{m,n,l},
        \end{aligned}
    \end{equation}
\end{small}
Equation~(\ref{eq:filter}) is a recursion equation, which suggests
that the filter $\mathbf{A}_{m,n,l}$ recursively updates in a certain
order. The residual can be written as:
\begin{equation}
    \label{eq:residual}
    r_{m,n,l}=  \lambda^{2} \frac{ \tilde{S}_{m,n,l} - \mathbf{S^{T}}_{m,n,l}
        \mathbf{\hat{A}}_{m,n,l} } { \lambda^{2} + \mathbf{S^{T}}_{m,n,l}
        \mathbf{S^{*}}_{m,n,l} }.
\end{equation}
Once obtaining the solution of the 3D $f$-$x$-$y$ SPF, one can compute
the noise-free data $\tilde{X}_{m,n,l}$ with the following equation:
\begin{equation}
    \label{eq:noise-free}
    \tilde{X}_{m,n,l} = \mathbf{S^{T}}_{m,n,l} \mathbf{A}_{m,n,l}.
\end{equation}
The configuration of parameters $\lambda_{f}$, $\lambda_{x}$, and
$\lambda_{y}$ is the basis for the proposed method. When the three
parameters are $0$, the corresponding regularization terms have no
effect on restricting the inverse problem and the result of the
$f$-$x$-$y$ SPF becomes (\ref{eq:solution1}). By choosing $\lambda_{y}
= 0$ and removing the $y$ axis, equation~(\ref{eq:filter}) is reduced
to the solution of the 2D $f$-$x$ SPF. On the contrary, when the three
parameters tend to infinity, more weight is applied on regularization
terms. A large denominator in (\ref{eq:filter}) indicates that the
filter cannot receive any updates to maintain its adaptive and
predictive properties. This denominator suggests that parameters
$\lambda_{x}^{2}$ and $\lambda_{y}^{2}$ in (\ref{eq:coef}) should have
the same order of magnitude as
$\mathbf{S^{T}}_{m,n,l}\mathbf{S^{*}}_{m,n,l}$, and the value of
$\lambda$ might be in the range of $(0, 10*\sqrt{ \max(
  \mathbf{S^{T}}_{m,n,l}\mathbf{S^{*}}_{m,n,l}) })$, which can balance
the noise suppression and signal protection.  Therefore, they can
smoothly adjust the change of filters.  Meanwhile, data distribution
in the frequency axis may change sharply, which is not as smooth as
those in the spatial directions; thus, $\lambda_{f}$ should be smaller
than $\lambda_{x}$ and $\lambda_{y}$.

\begin{table}[!t]
    \caption{
        Cost comparison between $f$-$x$-$y$ RNA and $f$-$x$-$y$ SPF. }
    \label{tbl:cost1}
    \centering
    \begin{threeparttable}
        \begin{tabular}{|c|c|c|}
            \hline
                            & Storage                             & Cost \\
            \hline
            $f$-$x$-$y$ RNA & $ O(N_{a}N_{f}N_{x}N_{y}) $
                            & $ O(N_{a}N_{f}N_{x}N_{y}N_{iter}) $        \\
            \hline
            $f$-$x$-$y$ SPF & $ O(N_{a}N_{x}N_{y}) $
                            & $ O(N_{a}N_{f}N_{x}N_{y}) $                \\
            \hline
        \end{tabular}

        \begin{tablenotes}[para, flushleft]
            $N_a$ is the filter size, $N_f$ is the data size along frequency
            axis, $N_x$ is the data size along $x$ axis, $N_y$ is the data
            size along $y$ axis, and $N_{iter}$ is the number of iterations.
        \end{tablenotes}
    \end{threeparttable}
\end{table}

\inputdir{.}
\multiplot{2}{fig1,fig2}{width=0.47\columnwidth} {Schematic
  illustration of $f$-$x$ prediction filter (a) and filter processing
  path (b).}

\multiplot{2}{fig3,fig4}{width=0.47\columnwidth} {Schematic
  illustration of $f$-$x$-$y$ prediction filter (a) and filter
  processing path in each frequency slice (b).}

\subsection{Data processing path in 3D case}

The processing path for the 2D $f$-$x$ SPF is shown in
Fig.~\ref{fig:fig2}, which initializes the filter at the beginning of
each line and updates the filter until the end of the line along the
spatial direction. We defined a new processing path for the 3D case;
the 3D $f$-$x$-$y$ SPF will process data from the beginning of a line
to the end, and then from the end of the next line to the beginning in
the spatial directions (Fig.~\ref{fig:fig4}).  This snaky processing
path avoids the weakness that the filter should be initialized at the
beginning of each line, which guarantees the effectiveness of initial
filters.

For calculating the 3D $f$-$x$-$y$ SPF, the neighboring filters
$\mathbf{A}_{m-1,n,l}$, $\mathbf{A}_{m,n-1,l}$, and
$\mathbf{A}_{m,n,l-1}$ need to be stored, and these filters will be
used when the stream reaches the adjacent point.  For easy program
implementation, we designed the cache space to store the neighboring
filters. The cache space for $\mathbf{A}_{m-1,n,l}$,
$\mathbf{A}_{m,n,l-1}$, and $\mathbf{A}_{m,n-1,l}$ are
$N_{a}N_{x}N_{y}$, $N_{a}N_{x}$, and $N_{a}$, respectively, where
$N_a$ is the filter size, $N_x$ is the data size along the $x$ axis,
and $N_y$ is the data size along the $y$ axis.  Compared with the 3D
$f$-$x$-$y$ RNA~\cite{Liug13}, the proposed method calculates the
filter coefficients without iterations, which reduces the requirement
of computational resources (Table~\ref{tbl:cost1}).  Repeatedly
processing the data with the SPF can further suppress the noise, but
part of the signal will also be weakened. Computational cost increases
with the number of repeated calculations, and a balance between
computational cost and noise suppression effect is needed.

The key steps of using the 3D $f$-$x$-$y$ SPF to attenuate
the random noise are as follows: \\
1. Initializing filter coefficients $ \mathbf{A}_{m-1,n,l}$,
$ \mathbf{A}_{m,n-1,l}$, and $\mathbf{A}_{m,n,l-1}$ with zeros.\\
2. Selecting reasonable parameters $\lambda_{f}$, $\lambda_{x}$,
and $\lambda_{y}$ and computing $ \lambda^{2} = \lambda_{f}^{2} + \lambda_{x}^{2} + \lambda_{y}^{2} $.\\
3. Calculating $\mathbf{S^{T}}_{m,n,l}\mathbf{S^{*}}_{m,n,l}$ and
$\mathbf{S^{T}}_{m,n,l}\mathbf{\hat{A}}_{m,n,l}$ in (\ref{eq:filter}).\\
4. Computing residual $r_{m,n,l}$ with (\ref{eq:residual}) and
updating filter
$\mathbf{A}_{m,n,l}$ with (\ref{eq:filter}).\\
5. Estimating noise-free data $\tilde{X}_{m,n,l}$ with (\ref{eq:noise-free}).\\
6. Looping steps 2-5 in snaky processing path until the entire process
is completed.

\section{Numerical examples}

\subsection{3D synthetic qdome model}

We start with the 3D qdome model~\cite{Claerbout08} containing curve
events and faults (Fig.~\ref{fig:qdmod}) to evaluate the proposed
method by handling the nonstationarity problem. The model size is 200
(time samples) $\times$ 150 (X traces) $\times$ 100 (Y traces).
Fig.~\ref{fig:qdnoise} displays the data with Gaussian noise added.
We compared the 3D $f$-$x$-$y$ SPF with the 2D $f$-$x$ SPF and the 3D
$f$-$x$-$y$ RNA~\cite{Liug13} to test their ability for random noise
attenuation. The filter length of the $f$-$x$ SPF is 5-sample
($x$). We also selected the scale parameters with 0.008
($\lambda_{f}$) and 0.06 ($\lambda_{x}$). Fig.~\ref{fig:qdspf2} shows
the denoised result obtained by using the $f$-$x$ SPF that eliminates
most of the random noise. However, there is still an obvious signal in
the noise section (Fig.~\ref{fig:qderrspf2}) because the 2D $f$-$x$
SPF has a low accuracy owing to the local similarity of filter
coefficients only along the $f$ and $x$ directions. A more effective
approach is to apply global smoothness. The denoised result obtained
by using the 3D $f$-$x$-$y$ RNA is shown in Fig.~\ref{fig:qdrna}.  The
filter size of the $f$-$x$-$y$ RNA is 5-sample ($x$) $\times$ 5-sample
($y$). The 3D $f$-$x$-$y$ RNA has a better result than the 2D $f$-$x$
SPF, and it is visually difficult to detect the signal in the
difference between the noisy data (Fig.~\ref{fig:qdnoise}) and the
denoised result (Fig.~\ref{fig:qdrna}).  The global smoothness
constraints along two spatial directions can help RNA to improve the
result (Fig.~\ref{fig:qderrrna}), but it also increases the
computational cost because it iteratively solves the regularized
least-squares problem (Table~\ref{tbl:cost1}).  We designed a 3D
$f$-$x$-$y$ SPF with 5-sample ($x$) $\times$ 5-sample ($y$)
coefficients for each sample and the scale parameters 0.008
($\lambda_{f}$), 0.06 ($\lambda_{x}$), and 0.06 ($\lambda_{y}$). The
proposed method produces a reasonable result (Fig.~\ref{fig:qdspf3}),
where the geological structure is improved. It is also difficult to
distinguish the signal in the removed noise
(Fig.~\ref{fig:qderrspf3}), which is an indication of successful
signal and noise separation. The signal-to-noise ratio (SNR) and time
consumption were used to analyze the performance of each method
(Table~\ref{tbl:cost2}). The SNR is defined as:
\begin{equation}
    \label{eq:snr}
    SNR=10\log_{10}\frac{||\mathbf{s}||_2^2}
    {||\mathbf{s}-\hat{\mathbf{s}}||_2^2},
\end{equation}
where $\mathbf{s}$ is the noise-free signal and $\hat{\mathbf{s}}$ is
the denoised signal. The computing platform uses Intel E5-2650 2.0GHz
CPU and the displayed time consumption is the average of ten
records. Although the denoised result obtained by using the $f$-$x$
SPF shows a relatively high SNR, the amplitude of the curved events is
partly damaged, which is shown in Fig.~\ref{fig:qderrspf2}.  The
$f$-$x$-$y$ RNA preserves a more detailed structure than the $f$-$x$
SPF at the cost of a lower SNR and longer computational time. In
general, the $f$-$x$-$y$ SPF provides a higher SNR while maintaining
the computational cost at a feasible level.

\inputdir{qdome}
\multiplot{2}{qdmod,qdnoise}{width=0.47\columnwidth} {3D synthetic
  model (a) and noisy data (b).}

\multiplot{6}{qdspf2,qderrspf2,qdrna,qderrrna,qdspf3,qderrspf3}
{width=0.47\columnwidth} {Denoised result by the $f$-$x$ SPF (a),
  noise removed by the $f$-$x$ SPF (b), denoised result by the
  $f$-$x$-$y$ RNA (c), noise removed by the $f$-$x$-$y$ RNA (d),
  denoised result by the $f$-$x$-$y$ SPF (e), and noise removed by the
  $f$-$x$-$y$ SPF (f).}

\subsection{3D synthetic CMP gather}

A 3D synthetic CMP gather created by \cite{Liug13} shows hyperbolic
events (Fig.~\ref{fig:cmpmod}). The data size is 126-sample (time)
$\times$ 101-sample (X) $\times$ 101-sample (Y). The corresponding
noisy data is shown in Fig.~\ref{fig:cmpnoise}.  The challenge in this
case is to account for both strong random noise and nonstationary
events. Fig.~\ref{fig:cmpspf2} shows the denoised result using the 2D
$f$-$x$ SPF with 5-sample ($x$) filter size, which uses the scale
parameters of 1.0 ($\lambda_{f}$) and 4.5 ($\lambda_{x}$).  Note that
the $f$-$x$ SPF can attain relatively high SNR
(Table~\ref{tbl:cost2}).  However, it cannot provide sufficient
protection for curve events; the events at far offsets can be
destroyed (Fig.~\ref{fig:cmpspf2}) and part of the signals are left
(Fig.~\ref{fig:cmperrspf2}).  Figure~\ref{fig:cmpct} is the denoised
result by using the 3D curvelet transform, we adopted the percentage
threshold with 10\% to mute the random noise, and the 3D curvelet
transform effectively suppresses the noise, but part of the signal
energy leaks into the noise profile at $-0.5 \sim 0.5$km (X)
(Figure~\ref{fig:cmperrct}).  We compared the proposed method with the
3D $f$-$x$-$y$ RNA, and the filter size is 5-sample ($x$) $\times$
5-sample ($y$). The denoised result (Fig.~\ref{fig:cmprna}) and the
removed noise (Fig.~\ref{fig:cmperrrna}) illustrate that the
$f$-$x$-$y$ RNA protects more signal by removing less noise.  The
parameters of the 3D $f$-$x$-$y$ SPF are selected as 1.0
($\lambda_{f}$), 4.5 ($\lambda_{x}$), and 4.5 ($\lambda_{y}$),
respectively. The filter size of the 3D SPF is the same as that of the
3D RNA. The proposed method recovers the curved events reasonable well
(Fig.~\ref{fig:cmpspf3}), similar to the $f$-$x$-$y$ RNA.  However,
the $f$-$x$-$y$ SPF saves more computational resources and reveals a
higher SNR than the $f$-$x$-$y$ RNA (Table~\ref{tbl:cost2}).

\inputdir{cmp}

\multiplot{2}{cmpmod,cmpnoise}{width=0.47\columnwidth} {Synthetic 3D
  CMP gather (a) and noisy data (b).}

\multiplot{8}{cmpspf2,cmperrspf2,cmpct,cmperrct,cmprna,cmperrrna,cmpspf3,cmperrspf3}
{width=0.39\columnwidth} {Denoised result by the $f$-$x$ SPF (a),
  noise removed by the $f$-$x$ SPF (b), denoised result by the 3D
  curvelet transform (c), noise removed by the 3D curvelet transform
  (d), denoised result by the $f$-$x$-$y$ RNA (e), noise removed by
  the $f$-$x$-$y$ RNA (f), denoised result by the $f$-$x$-$y$ SPF (g),
  noise removed by the $f$-$x$-$y$ SPF (h).}

\begin{table}[!t]
    \caption{
        Comparison of the SNR and time consumption
        among different methods.}
    \label{tbl:cost2}
    \centering
    \resizebox{\textwidth}{35mm}{
        \begin{threeparttable}[b]
            \begin{tabular}{ccccccc}
                \toprule
                \multirow{2}{*}{ \diagbox {Model (size)} {Method} }
                                          & \multicolumn{2}{c}{$f$-$x$ SPF}
                                          & \multicolumn{2}{c}{$f$-$x$-$y$ RNA}
                                          & \multicolumn{2}{c}{$f$-$x$-$y$ SPF} \\
                \cmidrule(r){2-3} \cmidrule(r){4-5} \cmidrule(r){6-7}
                                          & SNR (dB)\tnote{1}
                                          & Time (s)\tnote{2}
                                          & SNR (dB)
                                          & Time (s)
                                          & SNR (dB)
                                          & Time (s)                            \\
                \midrule
                qdome model (200*150*100) & $6.284$
                                          & $5.89*10^{0}$
                                          & $5.925$
                                          & $8.23*10^{2}$
                                          & $14.16$
                                          & $2.36*10^{1}$                       \\
                CMP gather (126*101*101)  & $2.422$
                                          & $2.63*10^{0}$
                                          & $1.324$
                                          & $1.53*10^{2}$
                                          & $2.544$
                                          & $9.26*10^{0}$                       \\
                field data (700*266*310)  & -
                                          & $8.79*10^{1}$
                                          & -
                                          & $1.52*10^{3}$
                                          & -
                                          & $1.37*10^{2}$                       \\
                \bottomrule
            \end{tabular}
            \begin{tablenotes}
                \item[1] {signal-noise-ratio of denoised result.}
                \item[2] {Time consumption is the average of ten records.}
            \end{tablenotes}
        \end{threeparttable}}
\end{table}

\subsection{3D field data}

For the field data test, we used a 3D time migration data to evaluate
the effectiveness of the SPF (Fig.~\ref{fig:realdata}). The data size
is 700-sample (time) $\times$ 266-sample (X) $\times$ 310-sample
(Y). Strong random noise from the surface conditions contaminates both
the simple layers at the shallow locations and complex structure at
the deeper positions.  We applied the 2D $f$-$x$ SPF with 5-sample
($x$) to recover the events and selected 55 ($\lambda_{f}$) and 280
($\lambda_{x}$) as the regularization terms for the improved $f$-$x$
SPF with frequency constraint. Fig.~\ref{fig:realspf2} and
\ref{fig:realerrspf2} show the denoised results and the removed noise
at the same clip value, respectively. Both shallow plane events and
deep dipping events show better lateral continuity, but the 2D SPF
also removes a part of the events because it uses the local smoothness
constraints without the $y$ direction. Meanwhile, information is
removed near $x=0$ and $y=0$ because of the inaccurate initial filter
coefficients.  We also compared the 2D SPF with the 2D $f$-$x$ EMD
prediction filter~\cite{Chen14} to test it ability for random noise
attenuation. The denoised result of $f$-$x$ EMD prediction filter is
shown in Fig.~\ref{fig:realfxemd}. The 2D $f$-$x$ EMD prediction
filter preserves signal better than the 2D SPF, but the difference
(Fig.~\ref{fig:realerrfxemd}) still shows obvious signal and the
method is difficult to be implemented in 3D case.  For comparison, we
used the 3D $f$-$x$-$y$ RNA to attenuate random noise.  The
$f$-$x$-$y$ RNA with 5-sample (X) $\times$ 5-sample (Y) filter size
outputs a smoother result and the lateral continuity is improved
(Fig.~\ref{fig:realrna}), where the reflection event in both shallow
and deep parts becomes clearer than original field
data. Fig.~\ref{fig:realerrrna} displays that only parts of dipping
events are slightly lost.  The proposed $f$-$x$-$y$ SPF can produce
more feasible results than the 2D version because the 3D SPF has extra
nonstationarity along the $y$ axis (Fig.~\ref{fig:realspf3}), where
the continuity of the events and the geological structure are
enhanced. The scale parameters are selected as 90 ($\lambda_{f}$), 600
($\lambda_{x}$), and 600 ($\lambda_{y}$) and the same 5-sample (X)
$\times$ 5-sample (Y) filter size as the 3D $f$-$x$-$y$ RNA.  The
difference (Fig.~\ref{fig:realerrspf3}) between the noisy data
(Fig.~\ref{fig:realdata}) and the denoised result
(Fig.~\ref{fig:realspf3}) shows more uniformly-distributed random
noise than the 3D $f$-$x$-$y$ RNA, which demonstrates that the
proposed filter is able to depict the variations in the nonstationary
signals and provide an accurate estimation of complex wavefields even
in the presence of strongly curved and conflicting events.
Furthermore, the $f$-$x$-$y$ SPF takes relatively low computational
time (Table~\ref{tbl:cost2}) that highlights its efficiency,
especially in higher dimensions.

\inputdir{real}
\plot{realdata}{width=0.6\columnwidth} {Three-dimensional field
  data.}

\multiplot{2}{realspf2,realerrspf2}{width=0.47\columnwidth} { The
  denoised result by using the 2D $f$-$x$ SPF (a) and the difference
  between the noisy data (Fig.~\ref{fig:realdata}) and the denoised
  result by using the 2D $f$-$x$ SPF (Fig.~\ref{fig:realspf2}) (b).}

\multiplot{2}{realfxemd,realerrfxemd}{width=0.47\columnwidth} {The
  denoised result by using the $f$-$x$ EMD prediction filter (a) and
  the difference between the noisy data (Fig.~\ref{fig:realdata}) and
  the denoised result by using the $f$-$x$ EMD prediction filter
  (Fig.~\ref{fig:realfxemd}) (b).}

\multiplot{2}{realrna,realerrrna}{width=0.47\columnwidth} {The
  denoised result by using the 3D $f$-$x$-$y$ RNA (a) and the
  difference between the noisy data (Fig.~\ref{fig:realdata}) and the
  denoised result by using the 3D $f$-$x$-$y$ RNA
  (Fig.~\ref{fig:realrna}) (b).}

\multiplot{2}{realspf3,realerrspf3}{width=0.47\columnwidth} {The
  denoised result by using the 3D $f$-$x$-$y$ SPF (a) and the
  difference between the noisy data (Fig.~\ref{fig:realdata}) and the
  denoised result by using the 3D $f$-$x$-$y$ SPF
  (Fig.~\ref{fig:realspf3}) (b).}

\section{Discussions}

The frequency-space PFs shows different characteristics from the
time-space PFs that depend on the sample rate along time direction.
PFs in time-space domain cannot predict the seismic events from all
directions (with all kinds of slopes) because the whole trace length
is seldom chosen as the filter size along the time direction.  The
frequency method can avoid this problem, and it may get an accuracy
prediction from the complex seismic data. However, computation in
frequency domain may result in artificial effect, so we designed the
local smoothness constraint to reduce the effect.

We highlighted the computational efficiency of the proposed
$f$-$x$-$y$ SPF because of not only the noniterative property but also
the memory saving ability. Assuming that a conventional 3D data volume
500 (frequency) $\times$ 500 (space $x$) $\times$ 500 (space $y$) is
about 1 gigabyte with complex number format, then the $f$-$x$-$y$ RNA
with 5-sample (X) $\times$ 5-sample (Y) filter structure needs over 23
gigabytes memory space to cache the filter coefficients for data
calculation, and the data exchange time is much more than that of
iterative calculation.  The proposed $f$-$x$-$y$ SPF needs only about
50 megabytes to cache filter coefficients when calculating.

\section{Conclusions}

In this study, we introduced a fast approach to nonstationary
prediction filter for random noise attenuation in the 3D $f$-$x$-$y$
domain. The proposed method employs a local similarity to constrain
the autoregression equation for nonstationary prediction filter in the
frequency-space domain, which belongs to the streaming prediction
theory. Constrained conditions in the 3D frequency-space dimensions
guarantee the accurate estimation of adaptive prediction filters and
reasonable prediction of complex structures. Instead of using an
iterative strategy, the new analytical solution in the frequency
domain for the least-squares problem allows the proposed method to
reduce computational complexity significantly. The matching snaky
processing path further improves the signal recovery ability of the
three-dimensional SPF. Although the $f$-$x$-$y$ SPF shows similar
accuracy to the $f$-$x$-$y$ RNA, the proposed method allows us to
better balance the target event protection, random noise suppression,
and computational efficiency.  Numerical examples using synthetic
models and field data show that the $f$-$x$-$y$ SPF can effectively
attenuate random noise and protect valid information in the
nonstationary seismic data.  The proposed method is superior in terms
of its low computational cost even when analyzing large-scale seismic
data.

\section{Acknowledgements}
This work is supported by National Natural Science Foundation of China
(grant nos. 41774127, 41974134 and 41522404) and National Key Research
and Development Program of China (grant no. 2018YFC0603701).

\appendix

\section{Sherman-Morrison formula in the complex space}
\label{S-M-F}
To get the inverse of $ ( \lambda^{2} \mathbf{I} +
\mathbf{S^{*}}_{m,n,l} \mathbf{S^{T}}_{m,n,l}) $, one can implement
Sherman-Morrison formula to transform the inverse matrix as
(\ref{eq:inverse}).  Whereas $ \mathbf{S^{*}}_{m,n,l} $ and $
\mathbf{S^{T}}_{m,n,l} $ are complex vectors.  Here, we drop the
subscript of vectors for a concise proof:
\begin{displaymath}
    \label{eq:prove of inverse}
    \begin{aligned}
         & (\lambda^{2} \mathbf{I} + \mathbf{S^{*}S^{T}}) \frac{ 1 } { \lambda^{2} }
        ( \mathbf{I} - \frac{ \mathbf{S^{*}S^{T}}}
        { \lambda^{2} + \mathbf{S^{T}S^{*}} })                                       \\
         & = \mathbf{ I } - \frac{ \mathbf{S^{*}S^{T}} }
        { \lambda^{2} + \mathbf{S^{T}S^{*}} } + \frac{ 1 } { \lambda^{2} }
        \mathbf{S^{*}S^{T}} - \frac{ 1 } { \lambda^{2} }
        \frac{ \mathbf{S^{*}S^{T}S^{*}S^{T}} } { \lambda^{2} + \mathbf{S^{T}S^{*}} } \\
         & = \mathbf{ I } + \frac{ 1 } { \lambda^{2} }
        \frac{ \lambda^{2} \mathbf{S^{*}S^{T}} + \mathbf{ S^{T}S^{*}}
        \mathbf{S^{*}S^{T} } } { \lambda^{2} + \mathbf{S^{T}S^{*}} }
        - \frac{ \mathbf{S^{*}S^{T}}} { \lambda^{2} + \mathbf{S^{T}S^{*}} }          \\
         & - \frac{ 1 } { \lambda^{2} } \frac{ \mathbf{S^{*}S^{T}}
        \mathbf{S^{*}S^{T}} } { \lambda^{2} + \mathbf{S^{T}S^{*}} }                  \\
         & = \mathbf{ I } + \frac{ \mathbf{S^{*}S^{T}}}
        { \lambda^{2} + \mathbf{S^{T}S^{*}} } + \frac{ 1 } { \lambda^{2} }
        \frac{ \mathbf{S^{T}S^{*}} \mathbf{S^{*}S^{T} } }
        { \lambda^{2} + \mathbf{S^{T}S^{*}} } - \frac{ \mathbf{S^{*}S^{T}}}
        { \lambda^{2} + \mathbf{S^{T}S^{*}} }                                        \\
         & - \frac{ 1 } { \lambda^{2} } \frac{ \mathbf{ S^{*}S^{T} }
        \mathbf{ S^{*}S^{T} } } { \lambda^{2} + \mathbf{ S^{T}S^{*} }}               \\
         & = \mathbf{ I },
    \end{aligned}
\end{displaymath}
where $\mathbf{I}$ denotes the identity matrix, $\mathbf{S}$ is the
complex column vector, $\mathbf{S^{T}}$ is the transpose of
$\mathbf{S}$, $\mathbf{S^{*}}$ is the conjugation of $\mathbf{S}$.
Therefore, $ \mathbf{S^{*}S^{T}} $ is a complex matrix, and $
\mathbf{S^{T}S^{*}} $ is constant. Therefore, we prove that
Sherman-Morrison formula can be applied in the complex space.

\bibliographystyle{seg}
\bibliography{paper}

