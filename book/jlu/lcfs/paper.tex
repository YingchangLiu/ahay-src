\published{Applied Geophysics, 17, 844-856, (2020)}

\title{Continuous time-varying Q-factor estimation method in the
  time-frequency domain}

\renewcommand{\thefootnote}{\fnsymbol{footnote}}
{}
    
\address{
College of Geo-explorqation Science and Technology,\\
Jilin University \\
No.938 Xi minzhu street \\
Changchun, China, 130026}

\author{Qinghan Wang, Yang Liu, Cai Liu, and Zhisheng Zheng}

\maketitle

\begin{abstract}
  The Q-factor is an important physical parameter for characterizing
  the absorption and attenuation of seismic waves propagating in
  underground media, which is of great significance for improving the
  resolution of seismic data, oil and gas detection, and reservoir
  description. In this paper, the local centroid frequency is defined
  using shaping regularization and used to estimate the Q values of
  the formation. We propose a continuous time-varying Q-estimation
  method in the time-frequency domain according to the local centroid
  frequency, namely, the local centroid frequency shift (LCFS)
  method. This method can reasonably reduce the calculation error
  caused by the low accuracy of the time picking of the target
  formation in the traditional methods. The theoretical and real
  seismic data processing results show that the time-varying Q values
  can be accurately estimated using the LCFS method. Compared with the
  traditional Q-estimation methods, this method does not need to
  extract the top and bottom interfaces of the target formation; it
  can also obtain relatively reasonable Q values when there is no
  effective frequency spectrum information. Simultaneously, a
  reasonable inverse Q filtering result can be obtained using the
  continuous time-varying Q values.
\end{abstract}

\section{Introduction}
The viscoelasticity and heterogeneity of subsurface media result in
amplitude attenuation and phase distortion of seismic waves
propagating in the media, considerably reducing the seismic data
resolution. The study of the absorption and attenuation
characteristics of underground media is of great significance to the
inversion of geophysical properties and the distribution analysis of
oil and gas reservoirs. The Q-factor is an important parameter for
describing the above characteristics that reflect the formation
structure and the comprehensive effect of fluid properties such as
saturation, porosity , and permeability \cite[]{Winkler82}. The
Q-estimation methods are implemented in the time domain, frequency
domain, or time-frequency domain.

The time-domain methods mainly include the amplitude decay method, the
rise-time method \cite[]{Kjartansson79}, the wavelet modeling method
\cite[]{Jannsen85}, and the analytic signal method
\cite[]{Engelhard96}. The signal-to-noise ratio of acquired seismic
data is low and the amplitude information is unreliable because of the
influence of several environmental factors and complex medium
structure. Thus, the accuracy and stability of Q values estimated in
the time domain often cannot meet actual production
requirements. Frequency information is relatively less affected by
noise and data amplitude, so its stability is generally better than
that of the time-domain method \cite[]{Gong09, Yan01}. The frequency
domain methods include the spectral ratio (SR) method \cite[]{Bath74},
the matching method \cite[]{Raikes84}, the spectral modeling method
\cite[]{Jannsen85}, etc. The SR method is a widely used method with
high accuracy in estimating the Q values without noise; however, it is
greatly affected by the selected spectral bandwidth and noise
\cite[]{Tonn91}. Considering these factors, some scholars improved the
SR method. \cite{Wang15a} calculated the Q values using the
logarithmic spectral area difference (LSAD) method, and \cite{Liu18}
introduced frequency-based linear fitting in the SR
method. \cite{Quan97} proposed the centroid frequency shift (CFS)
method based on the fact that the centroid frequency experiences a
downshift when seismic waves propagate in viscoelastic media. This
method deduces the relationship between the Q-factor and the centroid
frequency when the amplitude spectrum shape is Gaussian, rectangular,
or triangular. However, these are different from the shape of the
actual seismic amplitude spectrum, so the CFS method may produce
systematic errors. Some scholars analyzed and improved the CFS method,
such as the Q-estimation method based on the combinations of
statistical frequency attributes \cite[]{Zhao13} and the Q-value
inversion using the centroid frequency of the energy spectrum
\cite[]{Wang15b}. Both techniques avoid the assumption of the
amplitude spectrum of seismic wavelets. \cite{Zhang02} proposed the
peak frequency shift (PFS) method to estimate the Q values of common
midpoint (CMP) gathers, and \cite{Gao07} used this method to estimate
the Q values of vertical seismic profiling (VSP) data. \cite{Li15a}
fitted the amplitude spectrum using the frequency-weighted-exponential
function and derived a formula for estimating the Q value. \cite{Hu13}
established the relationship between the centroid and peak frequencies
under the assumption of the Ricker wavelet and proposed an improved
frequency shift (IFS) method.  \cite{Li15b} estimated the Q values
using the dominant and centroid frequencies. These frequency-domain
methods must use the Fourier transform to calculate the frequency
spectrum of the seismic signal in the selected time window. However,
the actual seismic reflection data interfere with each other, causing
some difficulties in selecting the appropriate type and length of the
time window function. Furthermore, the frequency spectrum obtained
using the Fourier transform reflects the average eff ect of a certain
frequency component, reducing calculation accuracy of the Q-factors.

Several scholars calculate the Q values in the timefrequency domain to
avoid selecting time windows and reduce the influence of spectral
interference between adjacent seismic refl ection
waves. \cite{Reine09} compared the Q-factors estimated using the SR
method in four different time-frequency transform domains and
concluded that the time-frequency transform with the variable window
could achieve more accurate and stable calculation results.
\cite{Wang04} transformed poststack seismic data from the time domain
to the time-frequency domain using the Gabor transform and then
calculated the Q values according to the one-dimensional function of
the product of frequency and time. \cite{Lupinacci15} estimated the
Q-factors of the poststack seismic section in the Gabor transform
domain using three different methods.  \cite{An15} used the energy
density ration method to estimate the Q values of CMP gathers based on
an improved S-transform. \cite{Hao16} derived the Q-estimation formula
of the SR method in the generalized S-transform domain.  \cite{Liu11}
improved the division operation of the SR method using shaping
regularization and proposed a stable Q-estimation method based on the
S-transform. Both \cite{Liu16a} and \cite{Liu16b} estimated the
Q-factors based on the local time-frequency transform
(LTFT). \cite{Wu18} proposed a continuous spectral ratio slope method
based on the generalized S-transform to avoid the extraction of target
layers. Because the time-frequency transform can characterize the
local attributes of the seismic data, this paper develops a continuous
time-varying Q-estimation method in the time-frequency domain. The
LTFT is an adaptive time-frequency transform proposed by \cite{Liu13},
based on the Fourier transform and solves the underdetermined problem
of the least-squares solution of the adaptive Fourier series through
shaping regularization \cite[]{Fomel07b}. LTFT can adjust the
frequency range and frequency sampling interval while providing
variable resolution in the time direction.

When estimating the Q value in the frequency domain, accurately
extracting the instantaneous spectrum of the target layer’s top and
bottom interfaces is necessary. However, this is often difficult to
achieve in actual data processing. This paper proposes a continuous
time-varying Q-estimation method in the time-frequency domain based on
the local centroid frequency to avoid picking up the target layer. The
local centroid frequency is obtained by solving the inversion problem
of the centroid frequency in the time-frequency domain within the
constraint of the shaping regularization condition
\cite[]{Fomel07b}. This frequency is not calculated according to the
instantaneous spectrum information at a specific moment but by
adjusting the smoothing parameter to use the spectrum information
around the moment.  Therefore, we can get a relatively stable local
centroid frequency at times without effective spectrum information. In
this paper, the local centroid frequency and CFS method are combined
to calculate the timevarying equivalent Q-factors and the time-varying
interval Q-factors. Furthermore, we implement inverse Q filtering
processing \cite[]{Wang02, Wang06} using the continuous time-varying
equivalent Q-factors.

\section{Theory}

\subsection{Local centroid frequency}

The centroid frequency $f_c$ with respect to the amplitude spectrum $F(f)$
can be defined as \cite[]{Quan97}
\begin{equation}
  \label{eq:1}
  f_c=\frac{\int_{0}^{\infty} fF(f)df}{\int_{0}^{\infty} F(f)df},
\end{equation}
where $F(f)$ is the Fourier amplitude spectrum of the signal, and the
centroid frequency $f_c$ represents the first moment along the frequency
direction of the amplitude spectrum.

The variance $\sigma_c^2$ of the centroid frequency can be defined as
\begin{equation}
  \label{eq:2}
  \sigma_c^2=\frac{\int_{0}^{\infty} (f-f_c)^2F(f)df}{\int_{0}^{\infty} F(f)df},
\end{equation}

The time-frequency spectrum $B(f,t)$ replaces the Fourier amplitude spectrum
$F(f)$ in equations~\ref{eq:1} and equations~\ref{eq:2}, and the
instantaneous centroid frequency $f_c(t)$ and instantaneous variance
$\sigma_c^2(t)$ of the amplitude spectrum are defined as
\begin{equation}
  \label{eq:3}
  f_c(t)=\frac{\int_{0}^{\infty} fB(f,t)df}{\int_{0}^{\infty} B(f,t)df},
\end{equation}
\begin{equation}
  \label{eq:4}
  \sigma_c^2(t)=\frac{\int_{0}^{\infty} (f-f_c(t))^2B(f,t)df}
        {\int_{0}^{\infty} B(f,t)df}.
\end{equation}

The above equations show that the instantaneous centroid frequency and
variance are calculated instantaneously using the amplitude spectrum
information at a specific time. However, at times without effective
spectrum information, reasonable results cannot be obtained using this
calculation method.

\cite{Fomel07a} defined the local attributes of the seismic signals
such as local frequency and local similarity using shaping
regularization. In this paper, we use a similar method to define the
local centroid frequency and local variance. Equation~\ref{eq:3} shows
that the instantaneous centroid frequency is a division regarding two
integrals and can be expressed in linear algebraic notation as
\begin{equation}
  \label{eq:5}
  \mathbf{f}=\mathbf{L}^{-1}\mathbf{n},
\end{equation}
where 
$\mathbf{n}=[\int_{0}^{\infty}fB(f,t_0)df \cdots \int_{0}^{\infty}fB(f,t_n)df]^T$
represents the vector of the numerator in equation~\ref{eq:3}, $\mathbf{f}$
represents the vector of the instantaneous centroid frequency $f_c(t)$, and
$\mathbf{L}=\begin{bmatrix}\int_{0}^{\infty}B(f,t_0)df & 0 & \cdots & 0 \\ 0 & \int_{0}^{\infty}B(f,t_1)df & \cdots & 0 \\ \vdots & \vdots & \ddots & \vdots \\ 0 & \cdots & 0 \int_{0}^{\infty}B(f,t_n)df \end{bmatrix}$
is a diagonal matrix composed of the denominators in equation~\ref{eq:3}.
Equation~\ref{eq:5} can be regarded as an inversion problem. We use the
least-squares criterion to calculate $\mathbf{f}$ to meet the objective 
function $\min\parallel{\mathbf{n}-\mathbf{L}\mathbf{f}}\parallel_2^2$, where
$\parallel \cdot \parallel_2^2$ represents the $l_2$ norm. The
regularization condition is added to constrain the inversion problem to
satisfy the local property. For example, the traditional Tikhonov
regularization constrains the model $\mathbf{f}$ to satisfy the local
smoothness, and the objective function becomes
\begin{equation}
  \label{eq:6}
  \begin{cases}
    \mathbf{L}\mathbf{f}=\mathbf{n} \\
    \varepsilon\mathbf{D}\mathbf{f}\approx0
  \end{cases},
\end{equation}
where $\varepsilon$ controls the weight of the regularization term and
$\mathbf{D}$ is the Tikhonov regularization operator. In this case, the
least-squares solution of the objective function under the regularization
constraint is the local centroid frequency
\begin{equation}
  \label{eq:7}
  \mathbf{f}_{loc}=(\mathbf{L}^{T}\mathbf{L}+\varepsilon^2\mathbf{D}^{T}
    \mathbf{D})^{-1}\mathbf{L}^{T}\mathbf{n}.
\end{equation}

The theory of shaping regularization comes from data smoothing. It has
fewer parameters and a faster convergence speed than the traditional
Tikhonov regularization method. When considering shaping
regularization, the shaping operator $\mathbf{S}$ can be defined as
\begin{equation}
  \label{eq:8}
  \mathbf{S}=(\mathbf{I}+\varepsilon^2\mathbf{D}^{T}\mathbf{D})^{-1},
\end{equation}
and,
\begin{equation}
  \label{eq:9}
  \varepsilon^2\mathbf{D}^{T}\mathbf{D}=\mathbf{S}^{-1}\mathbf{I}.
\end{equation}

The least-squares solution under the shaping regularization constraint can
be obtained by substituting the above equation into equation~\ref{eq:7}
\begin{equation}
  \label{eq:10}
  \mathbf{f}_{loc}=(\mathbf{L}^{T}\mathbf{L}+\mathbf{S}^{-1}-\mathbf{I})^{-1}
  \mathbf{L}^{T}\mathbf{n}
  =[\mathbf{I}+\mathbf{S}(\mathbf{L}^{T}\mathbf{L}-\mathbf{I})]^{-1}\mathbf{S}
  \mathbf{L}^{T}\mathbf{n},
\end{equation}
when the iterative algorithm is used to solve the above problem, the scale
parameter $\lambda$ can be introduced to improve the convergence speed and
preserve physical dimensionality. Usually, $\lambda$ is chosen as the
least-squares norm of $\mathbf{L}$ \cite[]{Fomel07a}, and the local centroid
frequency $\mathbf{f}_{loc}$ is
\begin{equation}
  \label{eq:11}
  \mathbf{f}_{loc}=[\lambda^2\mathbf{I}+\mathbf{S}(\mathbf{L}^{T}\mathbf{L}-
    \lambda^2\mathbf{I})]^{-1}\mathbf{S}\mathbf{L}^{T}\mathbf{n}.
\end{equation}

Similarly, the local variance $\mathbf{\sigma}_{loc}^2$ can be
calculated using the above method. When calculating the local centroid
frequency, only one smoothing parameter is needed to control the
locality and smoothness of the local centroid frequency. The local
centroid frequency is not calculated instantaneously using the
information at a specific time or calculated globally in a time window
but is calculated locally using the information around the time. Thus,
a relatively reasonable local centroid frequency can be continuously
and smoothly calculated at the time of missing information (such as
when the amplitude spectrum is zero). In this paper, we use the local
centroid frequency to estimate the continuous time-varying Q values of
the formation.

\subsection{Local centroid frequency shift (LCFS) method}

Seismic waves propagating in underground media experience amplitude
attenuation and phase distortion. High-frequency components attenuate faster
than low-frequency components, so the centroid frequency of the amplitude
spectrum experiences a downshift in the propagation process. \cite{Quan97}
proposed the CFS method according to the above phenomenon.

When considering seismic wave propagation in the viscoelastic medium, the
amplitude spectrum of seismic waves with different travel times can be
approximately expressed as \cite[]{Zhang02}
\begin{equation}
  \label{eq:12}
  B(f,t)=A(t)B(f,t_0)\exp(-\frac{\pi f \Delta t}{Q}),
\end{equation}
where Q is the quality factor, $\Delta t=t-t_0$ is the travel time difference
, $A(t)$ is a frequency-independent factor (including spherical diffusion,
transmission loss, etc.), $B(f,t_0)$ is the seismic amplitude spectrum at
time $t_0$, and $B(f,t)$ is the amplitude spectrum at travel time $t$.

The CFS method assumes that the amplitude spectrum of the source wavelet
satisfies the Gaussian distribution and can be expressed as
\begin{equation}
  \label{eq:13}
  B(f,t_0)=\exp(-\frac{(f-f_c(t_0))^2}{2\sigma_c^2(t_0)}),
\end{equation}
where $f_c(t_0)$ and $\sigma_c^2(t_0)$ represent the instantaneous centroid
frequency and instantaneous variance of the amplitude spectrum at time $t_0$,
respectively. The timevarying Q-factors estimated using the CFS method can
be obtained from equations~\ref{eq:3}, ~\ref{eq:4}, and ~\ref{eq:12}:
\begin{equation}
  \label{eq:14}
  Q(t)=\frac{\pi\sigma_c^2(t_0)(t-t_0)}{f_c(t_0)-f_c(t)}.
\end{equation}

By replacing the instantaneous centroid frequency and instantaneous variance
in equation~\ref{eq:14} with the local centroid frequency and local variance,
the time-varying Q-estimation equation can be rewritten as
\begin{equation}
  \label{eq:15}
  Q_{loc}(t)=\frac{\pi\sigma_{loc}^2(t_0)(t-t_0)}{f_{loc}(t_0)-f_{loc}(t)}.
\end{equation}
where $f_{loc}(t_0)$ and $\sigma_{loc}^2(t_0)$ represent the local centroid
frequency and local variance of the amplitude spectrum at time $t_0$,
respectively, and $f_{loc}(t)$ is the local centroid frequency of the
amplitude spectrum at time $t$. The above method of estimating the Q
values using the local centroid frequency is called the LCFS. It can be seen
from the equation that this method must estimate the Q value in the
time-frequency domain.

The CFS method assumes that the amplitude spectrum of the source
wavelet is Gaussian spectrum and that the variance of the amplitude
spectrum does not change with the attenuation effect. However, the
amplitude spectrum of the actual seismic wave usually does not satisfy
the Gaussian distribution. The absorption and attenuation effect would
make the variance smaller and the bandwidth narrower, so the CFS
method would produce the systematic error proportional to the travel
time difference $\Delta t$. When the travel time difference of the two
reflected waves is small, the variances of the two waves are
approximately equal. Thus, this paper improves the Q-estimation
accuracy by reducing the travel time difference. Assuming that each
time sampling point corresponds to a stratum interface, the above
equation can be used to calculate the interval Q-factors between every
two adjacent time sampling points. Then, the interval Q-factors can be
used to further estimate the equivalent Q-factors between the
reference and the target layers. The amplitude spectrum of layer $n$
can be expressed as \cite[]{Zhang02}
\begin{equation}
  \label{eq:16}
  B(f,t_n)=A(t_n)B(f,t_0)\exp(-\pi f\sum_{i=1}^n\frac{\Delta t_i}{Q_i}),
  i=1,2,\cdots,n ,
\end{equation}
where $\Delta t_i=t_i-t_{i-1}$ and $Q_i$ are the travel time and quality
factor in layer $i$, respectively.

The above equation can be expressed by the equivalent Q theory as
\begin{equation}
  \label{eq:17}
  \exp(-\pi f\frac{t_n}{Q_{n,eff}})=\exp(-\pi f\sum_{i=1}^n\frac{\Delta t_i}
      {Q_i}),
\end{equation}

The above equation can be simplifi ed to
\begin{equation}
  \label{eq:18}
  Q_{n,eff}=\frac{t_n}{\sum_{i=1}^n\frac{\Delta t_i}{Q_i}},
\end{equation}
where $t_n=\sum_{i=1}^n\Delta t_i$ represents the total travel time of a
reflection.

By substituting the equation of interval Q-factors estimated using the LCFS
method into the above equation, the equivalent Q-factor of layer $n$ ($n$th
time sampling point) can be expressed as
\begin{equation}
  \label{eq:19}
  Q_{n,eff}=\frac{t_n}{\sum_{i=1}^n\frac{\Delta t_i}{\frac{\pi\sigma_{loc}^2
        (t_{i-1})\Delta t_i}{f_{loc}(t_{i-1})-f_{loc}(t_i)}}}=\frac{\pi t_n}
  {\sum_{i=1}^n\frac{f_{loc}(t_{i-1})-f_{loc}(t_i)}{\sigma_{loc}^2(t_{i-1})}}
\end{equation}
where $Q_{n,eff}$ represents the equivalent Q-factor from the reference layer
to layer $n$ estimated using the LCFS method.

\section{Synthetic data processing}

\subsection{Estimation of local centroid frequency}

To verify the feasibility of calculating the local centroid frequency
using the LTFT method, a nonstationary signal
(Figure~\ref{fig:ref,sig}b) is generated by convolving the Ricker
wavelet with a random reflection coefficient
(Figure~\ref{fig:ref,sig}a). The dominant frequency of the signal is a
function varying with time $f_d(t)=100-75t^2$.
Figure~\ref{fig:fpp,sigltft,sigst}a shows a time-frequency spectrum
consisting of the Ricker wavelet’s frequency spectrum (the dominant
frequency is $f_d(t)=100-75t^2$). According to the dominant frequency
of the Ricker wavelet, we can calculate the theoretical centroid
frequency (black line in Figure~\ref{fig:fpp,sigltft,sigst}a) of the
Ricker wavelet $f_c(t)=2f_d(t)/\surd{\pi}$ \cite[]{Hu13}.
Figures~\ref{fig:fpp,sigltft,sigst}b and ~\ref{fig:fpp,sigltft,sigst}c
show the time-frequency spectrum of the synthetic signal obtained
using the LTFT and S-transform, respectively. We estimated the local
centroid frequency from these two time-frequency spectra, as shown in
Figure 3 (the blue line is estimated from the LTFT, and the purple
line is estimated from the S-transform). Compared with the theoretical
centroid frequency (black line in Figures~\ref{fig:fpp,sigltft,sigst}a
and ~\ref{fig:difcf}), the local centroid frequency obtained using the
LTFT method is closer to the theoretical curve, so the LTFT analysis
method is selected for calculating the local centroid frequency and
time-varying Q-factors.

\inputdir{lcfest}
\multiplot{2}{ref,sig}{width=0.47\columnwidth}{Theoretical model. Random
              reflectivity series (a), synthetic nonstationary signal (b).}
\multiplot{3}{fpp,sigltft,sigst}{width=0.3\columnwidth}{Time-frequency 
              spectrum. Theoretical time-frequency spectrum (the black line 
              represents the theoretical centroid frequency) (a), 
              time-frequency spectrum of the LTFT (b), time-frequency 
              spectrum of the S-transform (c).}
\plot{difcf}{width=0.5\columnwidth}{Local centroid frequency estimation
      (the black line represents the theoretical centroid frequency, the
      blue line is estimated using the LTFT method, and the purple line is
      estimated using the S-transform).}

\subsection{Attenuated model with constant Q-factors}

Although the CFS and LCFS methods assume that the amplitude spectrum
is Gaussian shape, they are still applicable to other spectra that fit
well to the Gaussian spectrum, such as the Ricker wavelet
\cite[]{Quan97}. The solid line in Figure~\ref{fig:gau-ric} is the
amplitude spectrum of the Ricker wavelet (the dominant frequency is 50
Hz). According to the centroid frequency ($f_c=56.42$ Hz) and variance
($\sigma_{loc}^2=566.9$) of the amplitude spectrum, the Gaussian
spectrum can be synthesized (dotted line in
Figure~\ref{fig:gau-ric}). The two amplitude spectra can be well
fitted. In this paper, the Ricker wavelet is selected as the source
wavelet to verify the effectiveness of the LCFS method in estimating
the Q-factors. The Ricker wavelet with the dominant frequency of 50 Hz
is used as the source wavelet to convolute with the reflectivity
series to generate the attenuation model
(Figure~\ref{fig:asignal,fp,qq,cmsig}a). The time sampling interval is
1 ms, and the maximum propagation time is 1 s. There is one reflection
interface at 0.1, 0.2, 0.4, 0.6, 0.7, and 0.9 s respectively, and the
Q-factor of the formation is set to a constant value of 60, wherein
the first layer is set to be
unattenuated. Figure~\ref{fig:asignal,fp,qq,cmsig}b shows the
time-frequency spectrum of the attenuated signal obtained using the
LTFT method. The LTFT method can accurately represent the
time-frequency characteristics, wherein the black line is the local
centroid frequency estimated using shaping regularization. By taking
the local centroid frequency at 0.1 s as the referzence value, the
time-varying Q-value curve calculated using equation~\ref{eq:19} is
shown as the red line in Figure~\ref{fig:asignal,fp,qq,cmsig}c. The
black dotted line is the theoretical Q value. We can see that the
time-varying Q value estimated using the proposed method is close to
the theoretical Q value. Inverse Q filtering is performed using the
estimated time-varying Q-factors, and the result is shown in
Figure~\ref{fig:asignal,fp,qq,cmsig}d. It can be seen that the
amplitude and phase of the wavelets are well recovered.
Figure~\ref{fig:fp1,qq1,cmsig1} is the result estimated using the
proposed method in the S-transform
domain. Figure~\ref{fig:fp1,qq1,cmsig1}a is the time-frequency
spectrum of the S-transform, and the black line is the local centroid
frequency. Figure~\ref{fig:fp1,qq1,cmsig1}b shows the time-varying Q
values estimated using the LCFS method, where the estimated result
before 0.6 s has a large error relative to the result estimated using
the LTFT method, and the estimated result after 0.6 s is close to the
theoretical one.  Figure~\ref{fig:fp1,qq1,cmsig1}c shows the inverse Q
filtering result, and it can be seen that the amplitude and phase are
also reasonably recovered.  According to the attenuation model test
with constant Q, we conclude that the LCFS method can also get
relatively accurate results at times without spectrum information, and
more reasonable results can be obtained using a reasonable
time-frequency transform.

The traditional methods for estimating Q-factors in the time-frequency
domain need to extract spectral information from the target layer’s
top and bottom interfaces. In contrast, the method proposed in this
paper only needs to extract information from a reference interface. We
compare the proposed method with the traditional SR and CFS methods to
further clarify its effect.  We first extract the frequency spectrum
of the corresponding time at the maximum amplitude of the six
interfaces (extracted in Figure~\ref{fig:asignal,fp,qq,cmsig}b) and
estimate the Q-factors using the SR and CFS methods and interpolate
them. The results are shown in the red lines in
Figures~\ref{fig:difsrq-1,difcfsq-1,difsrq-2,difcfsq-2}a and
~\ref{fig:difsrq-1,difcfsq-1,difsrq-2,difcfsq-2}b, respectively (the
black dotted line represents the theoretical value). Then, we change
the picking time and estimate the Q-factors using the SR
(Figure~\ref{fig:difsrq-1,difcfsq-1,difsrq-2,difcfsq-2}c) and CFS
(Figure~\ref{fig:difsrq-1,difcfsq-1,difsrq-2,difcfsq-2}d) methods. In
Figure~\ref{fig:difsrq-1,difcfsq-1,difsrq-2,difcfsq-2}, we can see
that the conventional Q-estimation methods need to pick the reflection
interface accurately, but it is difficult to achieve in the actual
seismic data processing.

We estimated Q-factors in noise-added attenuated signals to confirm
the stability and effectiveness of the LCFS method. Random noise of
diff erent levels is added to the attenuated signal, as shown in
Figures~\ref{fig:asignal,fp,qq,asignal1,fp1,qq1,asignal2,fp2,qq2}a,
~\ref{fig:asignal,fp,qq,asignal1,fp1,qq1,asignal2,fp2,qq2}d, and
~\ref{fig:asignal,fp,qq,asignal1,fp1,qq1,asignal2,fp2,qq2}g,
respectively.  The time-frequency distributions
(Figures~\ref{fig:asignal,fp,qq,asignal1,fp1,qq1,asignal2,fp2,qq2}b,
~\ref{fig:asignal,fp,qq,asignal1,fp1,qq1,asignal2,fp2,qq2}e, and
~\ref{fig:asignal,fp,qq,asignal1,fp1,qq1,asignal2,fp2,qq2}h) of the
noise-added signals are calculated using the LTFT method, wherein the
black lines represent the local centroid frequencies. The red lines in
Figures~\ref{fig:asignal,fp,qq,asignal1,fp1,qq1,asignal2,fp2,qq2}c,
~\ref{fig:asignal,fp,qq,asignal1,fp1,qq1,asignal2,fp2,qq2}f, and
~\ref{fig:asignal,fp,qq,asignal1,fp1,qq1,asignal2,fp2,qq2}i represent
the time-varying Q-factors estimated using the proposed method. From
the processing results of the noise-added signals, the LCFS method can
also obtain relatively stable Q-factors in the presence of noise.

\inputdir{lcfest}
\plot{gau-ric}{width=0.5\columnwidth}{Amplitude spectrum of the Ricker
      wavelet (solid line) and the Gaussian spectrum (dotted line).}

\inputdir{conq-model}
\multiplot{4}{asignal,fp,qq,cmsig}{width=0.2\columnwidth}{Attenuated model
              and estimated results. The attenuated model with constant Q
              (a), the time-frequency spectrum of the LTFT (the black line
              represents the local centroid frequency) (b), Q estimated
              using the LCFS method (red line) and theoretical Q (black
              dotted line) (c), inverse Q filtering result (d).}
\multiplot{3}{fp1,qq1,cmsig1}{width=0.2\columnwidth}{Results estimated
              using the S-transform. The time-frequency spectrum of the
              S-transform (the black line represents the local centroid
              frequency) (a), Q estimated using the LCFS method (red line)
              and theoretical Q (black dotted line) (b), inverse Q filtering
              result (c).}
\multiplot{4}{difsrq-1,difcfsq-1,difsrq-2,difcfsq-2}{width=0.2\columnwidth}{
              Estimated Q-factors using different methods (red line:
              estimated Q-factors; black dotted lines: theoretical
              Q-factors). The SR method (maximum amplitude) (a), CFS method
              (maximum amplitude) (b), SR method (nonmaximum amplitude) (c),
              CFS method (nonmaximum amplitude) (d).}

\inputdir{conq-noise}
\multiplot{9}{asignal,fp,qq,asignal1,fp1,qq1,asignal2,fp2,qq2}
             {width=0.25\columnwidth}{Noise-added attenuated signal and
              estimated results. The noise-added attenuated signal, and
              noise intensity increases gradually (a), (d), (g), the
              time-frequency spectrum of the LTFT (black lines represent the
              local centroid frequencies) (b), (e), (h), Q estimated using
              the LCFS method (red line) and theoretical Q
              (black dotted line) (c), (f), (i).}

\subsection{Attenuated model with variable Q-factors}

We synthesized an attenuated model with variable Q-factors to further verify
the applicability of the proposed method. The time sampling interval of the
synthetic signal is 1 ms, and there is a reflection interface at 0.1, 0.2,
0.4, 0.6, 0.7, and 0.9 s respectively. The attenuated seismic trace is
synthesized according to the variable Q model (from 0 to 0.1 s, Q is
infinite; from 0.1 to 0.2 s, Q is 50; from 0.2 to 0.4 s, Q is 80; from 0.4 to
0.6 s, Q is 30; from 0.6 to 0.7 s, Q is 100; from 0.7 to 0.9 s, Q is 120), as
shown in Figure~\ref{fig:vasignal,vfp,vqq,vcmsig}a. The timefrequency
spectrum (Figure~\ref{fig:vasignal,vfp,vqq,vcmsig}b) and the local centroid
frequency (black line in Figure~\ref{fig:vasignal,vfp,vqq,vcmsig}b) are
obtained using the LTFT method. The results of the time-varying Q values (as
shown in Figure~\ref{fig:vasignal,vfp,vqq,vcmsig}c, the green dotted line
represents the theoretical equivalent Q, the red line represents the
estimated time-varying equivalent Q, the black dotted line represents the
theoretical interval Q, and the purple line represents the time-varying
interval Q calculated using the adjacent time information) using the LCFS
method show that the error of the time-varying equivalent Q-factors estimated
using this method is small and close to the trend of the theoretical
equivalent Q-factors. However, the error of the time-varying interval
Q-factors is large except for the third layer, which is mainly caused by the
instability of the traditional Q-factor inversion method.
Figure~\ref{fig:vasignal,vfp,vqq,vcmsig}d shows the inverse Q
filtering result calculated according to the time-varying equivalent Q, and
the amplitude and phase of the wavelet are well compensated.

As with the attenuated model with constant Q, the proposed method is compared
with the SR and CFS methods. We extract the spectrum information at the
moment of the maximum amplitude in Figure~\ref{fig:vasignal,vfp,vqq,vcmsig}b
and estimate the interval Q-factors using the SR and CFS methods, as shown by
the purple line in
Figures~\ref{fig:difsrq-1,difcfsq-1,difsrq-2,difcfsq-2,fp1,qq1}a
and ~\ref{fig:difsrq-1,difcfsq-1,difsrq-2,difcfsq-2,fp1,qq1}b. Then, we
change the picking time of the spectrum information. The Q-factors estimated
using the SR and CFS methods are shown as purple lines in
Figures~\ref{fig:difsrq-1,difcfsq-1,difsrq-2,difcfsq-2,fp1,qq1}c and
~\ref{fig:difsrq-1,difcfsq-1,difsrq-2,difcfsq-2,fp1,qq1}d, respectively. It
can be seen that the accuracy of the interface picking has a great influence
on the estimated result of Q. The equivalent Q-factors are calculated
according to the estimated interval Q-factors (the red lines in
Figures~\ref{fig:difsrq-1,difcfsq-1,difsrq-2,difcfsq-2,fp1,qq1}a,
~\ref{fig:difsrq-1,difcfsq-1,difsrq-2,difcfsq-2,fp1,qq1}b,
~\ref{fig:difsrq-1,difcfsq-1,difsrq-2,difcfsq-2,fp1,qq1}c, and
~\ref{fig:difsrq-1,difcfsq-1,difsrq-2,difcfsq-2,fp1,qq1}d). Compared with the
time-varying equivalent Q-factors estimated using the LCFS method, the error
of the traditional method is larger. The CFS method can also calculate
continuous Q-factors without the extraction of interfaces.
Figure~\ref{fig:difsrq-1,difcfsq-1,difsrq-2,difcfsq-2,fp1,qq1}e is the
time-frequency spectrum of the attenuated signal (
Figure~\ref{fig:vasignal,vfp,vqq,vcmsig}a), and the black line represents the
instantaneous centroid frequency calculated using equation~\ref{eq:3}. We
select the instantaneous centroid frequency at 0.1 s as the reference value
and calculate the time-varying equivalent Q-factors (red line in
Figure~\ref{fig:difsrq-1,difcfsq-1,difsrq-2,difcfsq-2,fp1,qq1}f) and the
time-varying interval Q-factors (purple line in
Figure~\ref{fig:difsrq-1,difcfsq-1,difsrq-2,difcfsq-2,fp1,qq1}f) based on the
instantaneous centroid frequency. Compared with the LCFS method, this method
can only obtain relatively accurate equivalent Q-factors when the spectrum is
available. Moreover, the estimated time-varying interval Q-factors are 
extremely unstable.            
          
\inputdir{varq-model}
\multiplot{4}{vasignal,vfp,vqq,vcmsig}{width=0.2\columnwidth}{Attenuated 
              modeland estimated results. The attenuated model with variable 
              Q (a), the time-frequency spectrum of the LTFT (the black line
              represents the local centroid frequency) (b), Q estimated
              using the LCFS method (the green dotted line represents the
              theoretical equivalent Q, the red line represents the
              estimated time-varying equivalent Q, black dotted line
              represents the theoretical interval Q, and the purple line
              represents the estimated time-varying interval Q calculated)
              (c), inverse Q filtering result (d).}
\multiplot{6}{difsrq-1,difcfsq-1,difsrq-2,difcfsq-2,fp1,qq1}
             {width=0.2\columnwidth}{Q-factors estimated using different
              methods. The SR method (maximum amplitude) (a), CFS method
              (maximum amplitude) (b), SR method (nonmaximum amplitude) (c),
              CFS method (nonmaximum amplitude) (d), time-frequency spectrum
              of the LTFT (the black line represents the instantaneous
              centroid frequency) (e), Q estimated using the CFS method based
              on instantaneous centroid frequency (f).}

\subsection{Field data}

For the field data test, we select a 2D poststack seismic section
(Figure~\ref{fig:powdata,agcpowdata,smqts,cmsig}a) that contains 201
seismic traces, each of which has a time length of 3 s and a time
sampling interval of 4 ms. For comparison, we perform the automatic
gain control (AGC) processing on the original seismic section. The
processed result after AGC is shown in
Figure~\ref{fig:powdata,agcpowdata,smqts,cmsig}b. AGC can enhance the
weak amplitude of the deep part, but it cannot improve the resolution
of the deep data. Next, we estimate the time-varying equivalent
Q-factors for each seismic trace using the LCFS method and select the
time corresponding to the maximum value of the local centroid
frequency as the reference time. The estimated equivalent Q profile is
smoothed to some extent in the spatial direction to preserve the
lateral continuity.  Figure~\ref{fig:powdata,agcpowdata,smqts,cmsig}c
shows that the Q-factors calculated using the LCFS method have a
reasonable distribution range and the characteristics of continuous
variation in the direction of time and
space. Figure~\ref{fig:powdata,agcpowdata,smqts,cmsig}d shows the
seismic section after inverse Q filtering based on the time-varying
equivalent Q-factors. Comparing
Figures~\ref{fig:powdata,agcpowdata,smqts,cmsig}a,
~\ref{fig:powdata,agcpowdata,smqts,cmsig}b, and
~\ref{fig:powdata,agcpowdata,smqts,cmsig}d, we can see that the
resolution of the deep part of the seismic data after inverse Q
filtering has improved, the absorption and attenuation of the
formation are reasonably compensated, and the original structural
characteristics are well maintained.
          
\inputdir{rdata}
\multiplot{4}{powdata,agcpowdata,smqts,cmsig}{width=0.45\columnwidth}
             {Processing result of field data. Poststack field data (a), 
               the seismic data after AGC (b), the estimated time-varying
               equivalent Q-factors (c), the seismic data after inverse Q
               filtering (d).}

\section{Conclusion}
             
We define the local centroid frequency using shaping regularization in
the time-frequency domain and discuss the theoretical basis of the
local centroid frequency in detail in this paper. The local centroid
frequency is calculated locally using information around a certain
time so that a reasonable result still can be obtained at the time
without information. By combining the local centroid frequency with
the CFS method, we propose a continuous time-varying Q-estimation
method, namely, the LCFS method. The theoretical model shows that the
proposed method can obtain reasonable time-varying Q-factors even in
an environment with random noise. Compared with the traditional
methods, the proposed method avoids picking the target layer and can
still obtain a stable estimation result in the absence of spectrum
information. At the same time, experiments with synthetic examples and
field data confirm that the inverse Q filtering method based on
continuous time-varying Q-factors can better recover the phase and
amplitude of the attenuation signal and improve the resolution of
seismic data.

\section{Acknowledgments}

We would like to thank Guochang Liu of China University of Petroleum
(Beijing) for his discussion and suggestions.



\bibliographystyle{seg}
\bibliography{paper}
