\chapter{Seismic data processing}

Although core Madagascar programs are applicable for analyzing any
kinds of data, there also programs specialized for seismic reflection
data processing. In this chapter, we will walk through processing one
particular dataset. Multiple additional examples are available under
\texttt{\$RSFSRC/book/data} and \texttt{\$RSFSRC/book/geo384s}. The
latter contains computational assignments from a seismic data
processing course at the University of Texas at Austin. An example
processing of a seismic land 3D dataset (Teapot Dome) is provided by
\cite{oren2018overview} and included in
\texttt{\$RSFSRC/book/data/teapotdome/canoren}.

\section{Poland dataset}
\inputdir{freeusp}

This dataset was made publicly available by Geofizyka Torun S.A. and
provided initially by Waldek Ogonowski for a tutorial on OpenUSP, a
seismic processing package from Amoco and
BP\footnote{\url{https://www.freeusp.org/RaceCarWebsite/TechTransfer/Tutorials/Processing_2D/Processing_2D.html}}.

The 2D reflection seismic survey was acquired on land using vibroseis
sources. The goal of a seismic processing project is to process the
dataset all the way to a seismic image.

\section{Analyzing acquisition geometry}

The seismic acquisition geometry involves the distribution of sources
and recievers on the surface.

\plot{SR0}{width=\textwidth}{Seismic acquisition geometry.}

Figure~\ref{fig:SR0} indicates sources with ... and receivers with
... Despite small deviations, the overall geometry is fairly regular
for a 2-D survey, with a uniform distribution along the line. In the
rest of the processing, we will treat is as such.

\section{Static corrections}

In land surveys, the topography of the observation surface plays a
role and requires a correction (time shifts), known as elevation
statics. The elevation static correction in this case is small and
ranges... (Figure~\ref{fig:elevation}.)

\plot{elevation}{width=\textwidth}{Elevation statics.}

An additional static correction is needed to account for variations of
seismic velocities in the near surface. This correction was provided
with the data and is shown in Figure~\ref{fig:static}.

\plot{static}{width=\textwidth}{Additional statics correction.}

Figure~\ref{fig:lines,lines2} shows the seismic data before and after
applying all statics corrections. The change is subtle, but the
continuity of reflection events gets improved.

\multiplot{2}{lines,lines2}{width=0.45\textwidth}{Seismic data before (a) and after (b)
  applying the static corrections.}

\section{Stacking diagram}

In preparation for processing seismic reflection data through
common-midpoint (CMP) stacking, it helps to analyze the distribution of
traces among different common-midpoint bins. The stacking diagram is
shown in Figure~\ref{fig:diagram} and indicates....

\plot{diagram}{width=\textwidth}{Stacking diagram.}

\section{CMP gathers}

The field data comes in shot records, arranged in shot-offset
coordinates. Before CMP stacking, all traces need to be sorted in
midpoint-offset coordinates (Figure~\ref{fig:rcmps}.)
 
\multiplot{2}{rcmps,cmps}{width=\textwidth}{Data sorted in CMP gathers
  before (a) and after (b) noise attenuation.}

The data are contaminated by surface waves (groundroll noise), which
need to be attenuated to reveal the useful reflection signal from
deeper targets.

\section{Noise attenuation}

We attenuate the groundroll noise using the technique of the local
time-frequency decomposition \cite[]{liu2013seismic}, which takes
advantages of the different behavior of the noise and signal
components in the frequency domain. The result is shown in
Figure~\ref{fig:cmps}.

\section{Velocity analysis and CMP stacking}

The next step is normal-moveout (NMO) velocity analysis and
stacking. Figure~\ref{fig:vel} shows...

\plot{vel}{width=\textwidth}{Stacking velocity.}

\section{Prestack time migration}

\bibliographystyle{seg}
\bibliography{data}

