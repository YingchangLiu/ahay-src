\chapter{Migration Gallery}

In seismic imaging, multiple algorithms exist for performing the task
of seismic migration. The algorithms differ in accuracy and
efficiency, which suggests a case for comparison. There are also
multiple known synthetic benchmarks for comparing different imaging methods.

The migration gallery project was born at the first Madagascar
\emph{working workshop} in
2013\footnote{\url{https://reproducibility.org/wiki/Austin_2013}}. The
main idea was to organize examples of multiple algorithms in the form
of a matrix, where rows would represent different benchmark examples
and columns would represent different migration algorithms. This
arrangement make it convenient to make direct comparisons of either
efficiency or accuracy wnen new algorithms are developed or previously
developed algorithms are tried on new data examples.

\section{Importance of benchmarking}

The common-task method was pioneered by DARPA (Defense Advanced
Research Projects Agency) and applied successuflly in the field of
Human Language Technologies \cite[]{liberman2020human}. \cite{donoho}
calls the common-task method ``a secret sauce'' in the success of
predictive methods, such as those developed in machine learning.

The method calls for an open competition among a number of
participants working on a common task. The key conditions is an
objective metric used to judge the results (thus keeping the contest
fair) and a requirement for the participants to share not only their
results but also their methods and not only with the research sponsor
but also with one another. Examples from different branches of science
and technology show that a sustained application of the common-task
method leads to measurable progress over the years, with occasional
breakthrough inventions.

The hope of the migration gallery project is to provide a platform
where different seismic imaging methods could be compared objectively
and improved over time, with improvements concerning either accuracy
or efficiency of different algorithms.

\section{Gallery project}

Example workflows from the migration gallery are organized by dataset
and located in \texttt{\$RSFSRC/book/data} directory. Thus, each data
directory serves as a chapter in the ``migration gallery'' book.

\section{Benchmark datasets}

\subsection{Constant velocity}

\subsection{$V(z)$ model}

\subsection{Velocity gradient}

\subsection{French model}

\subsection{Marmousi}

\subsection{Sigsbee}

\subsection{Pluto}

\subsection{Model-1994}

\subsection{Statics-1994}

\subsection{BP-2004}

\subsection{BP-2007}

\subsection{Hess VTI}

\subsection{2.5D model}

\subsection{Overthrust}

\subsection{SEG Salt}

\subsection{Teapot Dome}

\section{Migration methods}

\section{Conclusions}

\bibliographystyle{seg}
\bibliography{gallery}
