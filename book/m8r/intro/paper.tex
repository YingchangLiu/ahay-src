\chapter{Introduction}

This book is designed to provide a comprehensive documentation for
Madagascar, an open-source package for geophysical data
analysis. Madagascar is a community project that has involved dozens
of developers from around the world\footnote{According to OpenHub, a
  cite that analyzes open-source projects, 145 people from around the
  world contributed to the Madagascar development over the years:
  \url{https://www.openhub.net/p/m8r}.}. Rather than being simply a
collection of computational tools, Madagascar is also a repository of
knowledge, because it contains complete research papers and book
chapters with reproducible examples. The repository grows with time,
while the reproducibility of previously published research is
continuously tested in order to enable subsequent research.

Before we dive into details of Madagascar and explain how to use it,
let us first review the basic principles behind this project.

\section{What is Madagascar?}

The Madagascar software package is designed for analysis of
large-scale multidimensional data, such as those occurring in
exploration geophysics. The data analysis examples provided with
Madagascar fit into a framework for reproducible
research. \emph{Reproducible research} in this context refers to the
discipline of attaching software codes and data to computational
results reported in research publications \cite[]{fomel2008guest}. The
Madagascar package contains a collection of (a) computational modules,
(b) data-processing scripts, and (c) research papers. The main Madagascar
website is \url{https://www.reproducibility.org/}.

The concept of reproducible research goes back to 1990s and was
pioneered by \cite{Claerbout.sep.67.139}. ``Claerbout's principle'',
as formulated by \cite{buckheit1995wavelab}, states that ``An article
about computational science in a scientific publication is not the
scholarship itself, it is merely advertising of the scholarship. The
actual scholarship is the complete software development environment
and the complete set of instructions which generated the figures.''
The Madagascar software package implements a computational environment
that is designed both for conducting computational experiments in the
area of large-scale geophysical data analysis and for attaching links
to software code and data in scientific publications in order to
enable both the author of the original paper and other research to
reproduce the reported computational experiments and extend the
previous research. As of October 2021, Madagascar includes nearly 300
scientific papers and book chapters complete with software codes
necessary for independent verification and replication of
computational results\footnote{See
  \url{https://www.reproducibility.org/wiki/Reproducible_Documents}.}.

The work on the Madagascar project started in 2003, and the beta
version of the package was publicly released in June 2006. Since then,
many people have joined the project and contributed to the code. The
first fully functional 1.0 version was released in 2010 and tested by
an open community. The community stays in touch using mailing lists,
social networks, and annual meetings.  Although the main applications
have focused so far on applied geophysics and exploration seismology
in particular, the core package is suitable for other scientific
fields that require reproducible analysis of large-scale
multidimensional data.

\section{Madagascar design principles}

The design of Madagascar follows the Unix principle: ``Write programs
that do one thing and do it well. Write programs to work
together. Write programs to handle text streams, because that is a
universal interface.'' \cite[]{salus1994quarter} Analysis of complex multidimensional data,
such as those occurring in exploration seismology requires multiple
steps. In addition, the data size can be too large for storing data
objects in memory. A typical modern seismic survey can easily generate
terabytes of data. Research in computational methods for this kind of
data traditionally pushes the envelope on the size of the data
available for computer processing. For these reasons, Madagascar
breaks data-analysis workflows into multiple steps by writing short
programs that implement individual steps (``do one thing and do it
well'') with control parameters specified on the Unix command
line. The programs implementing different steps act as filters (``work
together'') by taking input from a file on disk or from a Unix pipe
and writing either to disk or to another pipe. Madagascar adopts a universal
data format, called RSF (regularly sampled file). The RSF format is
based on a text description (``because that is a universal
interface'') that points to the raw binary data stored in a separate
file. Conceptually, an RSF file represents a regularly sampled
multi-dimensional hypercube, while the corresponding binary data are
stored (or passed through a Unix pipe) in simple contiguous arrays for
optimally efficient input/output operations.  To assemble complex
workflows from individual programs, Madagascar has adopted SCons, a
Python-based utility \cite[]{knight2005building}. SCons configuration files
(SConstruct scripts) are written in Python and specify the database of
dependencies between input files, programs, and target files. SCons
supports other useful features, such as multithreaded execution. Madagascar's extension of SCons, we define four specific commands for
establishing data-processing dependencies \cite[]{fomel2007reproducible}:
\begin{description}
\item[Fetch] escribes a rule for downloading data
files from a remote data server or a local data directory. 
\item[Flow] describes a rule (command or Unix pipeline) for
generating one or more target files from one or more (or none) source
files.
\item[Plot] is similar to \textbf{Flow} but the target file is a figure.  
\item[Result] is similar to \textbf{Plot} but the
target file is a final ``result'' figure for inclusion
in a publication.
\end{description}

One can think of the Madagascar environment as existing on three
different levels that correspond to three different stages of typical research
activities of the computational scientist:
\begin{enumerate}
\item Implementing a new computational algorithm for data analysis. This
level involves writing low-level programs (command-line modules).
\item Testing a new algorithm or a new workflow by applying them to
data. This level involves assembling workflows from existing
command-line modules and tuning their parameters through repeated
computational experiments to achieve the desired result.
\item Publishing new results. Results from computational experiments
(figures in our case) get referenced in papers and included in
publications.
\end{enumerate}

Madagascar adopts SCons for the third level as well, to simplify creation of
documents that include results from the second level. Customized SCons
commands create documents from \LaTeX sources with output either in PDF
or HTML format. The HTML format is produced using \LaTeX2HTML \cite[]{drakos1994text}. In
the HTML version, reproducible figures are followed by links to
SConstruct scripts from level 2 and low-level programs from level 1 to
help the reader to verify the exact computational details of the
reported numerical experiment and and to reproduce the experiment.

\inputdir{.}
\plot{architecture}{width=0.9\textwidth}{Madagascar Software Architecture (reproduced from \protect\cite[]{fomel2013madagascar}.)}

\section{History of the project}

Madagascar stands on ``the shoulders of the giants''. Its immediate
predecessor is SEPlib, a popular library developed at the Stanford
Exploration Project (SEP) under the direction of Jon Claerbout
\cite[]{Claerbout.sep.70.413,Nichols.sep.82.257,Clapp.sep.102.bob1}. Generations
of SEP students and researchers contributed to SEPlib. Some of the most important
contributions came from Rob Clayton, Jon Claerbout, Dave Hale, Stew
Levin, Rick Ottolini, Joe Dellinger, Steve Cole, Dave Nichols, Martin
Karrenbach, Biondo Biondi, and Bob Clapp. Bob Clapp currently maintans
the package \cite[]{clapp2012seplib}. Madagascar reimplements most of
the classic SEPlib's functionality, as well as SEP's framework for
reproducible research
\cite[]{Schwab.sep.89.217,Fomel.sep.94.matt3}. It also borrows Vplot,
a graphics library originally developed at SEP \cite[]{Cole.sep.60.349,Dellinger.sep.61.327}.

Another popular package with a direct influence on Madagascar is
Seismic Unix (SU). The main contributors to SU included Einar
Kjartansson, Shuki Ronen, Jack Cohen, Chris Liner, Dave Hale, and John
Stockwell. John Stockwell maintained SU at the Colorado School of
Mines \cite[]{stockwell1997free}. Starting from release 40 in 2007, SU
adopted an open-source BSD-style license, which allowed some of its
codes to be directly borrowed by Madagascar\footnote{See
  \url{https://reproducibility.org/blog/2010/08/17/madagascar-for-users-of-seismic-unix/}.}.

The work on the Madagascar package started by Sergey Fomel at the
University of Texas in Austin in 2003. By 2006, the package was
publicly released with the GPL license and announced at the EAGE
Workshop in Vienna in June
2006\footnote{\url{http://sepwww.stanford.edu/oldsep/joe/Vienna/}}. Multiple
users and developers quickly joined the project, and, in August 2006,
the first school and workshop took place in
Vancouver{\footnote{\url{https://reproducibility.org/wiki/RSF_School_and_Workshop,_Vancouver_2006}}}. Around
the same time, the name Madagascar was
adopted\footnote{\url{https://reproducibility.org/blog/2006/04/19/madagascar/}}
instead of the previous RSF (from Regularly Sampled Format). Schools
and workshops became a regular tradition and occured almost every year
since then...

\section{Book conventions}

\subsection{Note on reproducibility}

\bibliographystyle{seg}
\bibliography{intro,SEP2}

