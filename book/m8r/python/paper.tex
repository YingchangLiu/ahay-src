\chapter{Python interface to Madagascar}

Python is a highly popular scripting programming languages. It is used by Madagascar in several contexts:
\begin{enumerate}
\item Like many other languages, Python can be used for writing Madagascar ``filter'' programs.
\item Python is used by SCons, the preferred Madagascar way of
  managing data analysis workflows. SCons scripts (\texttt{SConstruct}
  files) are written in Python and can use the full power of the
  language.
\item Python can also be used for convenient scripting of operations
  with RSF files and Madagascar programs. This is particularly
  convenient for interfacing with other standard Python packages for
  data analysis.
\end{enumerate}
This chapter focuses on the latter case.

\section{File objects}
\lstset{language=python,showstringspaces=false}

The \texttt{File} class in the Madagascar interface defines an RSF file object. Suppose we have a file \texttt{spike.rsf} created with
\begin{verbatim}
bash$ sfspike n1=5 k1=2 > spike.rsf
\end{verbatim}
Inside a Python script, the corresponding file object can defined as
\begin{lstlisting}
import m8r
spike = m8r.File('spike.rsf')  
\end{lstlisting}
Of course, we can also recreate the command that produced it.
\begin{lstlisting}
import m8r
spike = m8r.spike(n1=5,k1=2)[0]
\end{lstlisting}
Running \texttt{type(spike)} identifies \texttt{spike} as an object of
\texttt{m8r.File} class. Printing file by running
\texttt{print(spike)} outputs the name of the associated RSF file.

\subsection{Extracting file attributes}

Data attributes can be extracted from the file objects. We can ask, for example, for the mean
value with
\begin{lstlisting}
spike.mean()
\end{lstlisting}
or the standard deviation with
\begin{lstlisting}
spike.std()
\end{lstlisting}

The following table shows different attributes and their descriptions:

\begin{tabular}{ l | l }
  \hline 
  rms & root mean square \\
  \hline   mean & mean value \\
  \hline  norm & the square norm \\
  \hline var & variance \\
  \hline std & standard deviation \\
  \hline max & maximum value \\
  \hline min & minimum value \\
  \hline nonzero & number of nonzero values \\
  \hline samples & number of samples \\
  \hline
\end{tabular}

\subsection{Operations on files}

\subsection{Plotting files}

\subsection{Converting to and from numpy arrays}

\subsection{Open files for input and output}

\section{Filter objects}

\subsection{Pipes}

\section{Python interface for Vplot}

\subsection{Vplot in Jupyter notebooks}

\section{Working with \texttt{numpy} arrays}

\section{Other scripting interfaces}

\subsection{Matlab}

\subsection{Julia}
