\chapter{Managing processing flows using SCons}

There are several options for doing data processing using Madagascar:
you can run programs on the command line, you can collect a sequence
of commands in a Shell script or a Python script. A superior
alternative, as this chapter explains, is to use a workflow management
tool, such as SCons.

\section{What is SCons?}

SCons stands for \emph{Sofware Construction} and is a tool designed
primarily for compiling software \cite[]{scons}, as a modern
replacement for the Unix \texttt{make} utility \cite[]{make}. The
SCons configuration files (\texttt{SConstruct} scripts) are written in
the Python programming language.

SCons is a tool traditionally used for compiling programs. An \texttt{SConstruct} file for compilation may look like the following
\lstset{language=python,numbers=left,numberstyle=\tiny,showstringspaces=false}
\begin{lstlisting}
env = Environment()
env.Append(CPPFLAGS=['-Wall','-g'])
env.Program('hello',['hello.c', 'main.c'])
\end{lstlisting}
and produce something like
\begin{verbatim}
bash$ scons -Q
gcc -o hello.o -c -Wall -g hello.c
gcc -o main.o -c -Wall -g main.c
gcc -o hello hello.o main.o
\end{verbatim}
to compile the \texttt{hello} program from the source files \texttt{hello.c} and \texttt{main.c}.

Madagascar uses SCons to compile its programs from the source. The
more frequent usage, however, comes from adopting SCons to manage data
processing flows.

\subsection{Useful SCons options}

The \texttt{scons} program takes a number of options. Some of the commonly used options are:
\begin{description}
  \item[scons -h] (help) displays a help message.
  \item[scons -Q] (quiet) suppresses progress messages.
  \item[scons -n] (no exec) outputs the commands required for building the specified target (or the default targets if no target is specified) without actually executing them. It can be used to generate a shell script out of SConstruct script, as follows:
\begin{verbatim}
scons -nQ > script.sh
\end{verbatim}
\end{description}

\section{Why use SCons?}

In the compilation task, the target (an executible program) may
require multiple sources. The sources may have multiple dependencies
among them. When one of the sources changes, the target should be
recompiled. However, it would be wasteful to compile sources that are
not affected by the change. The famous Unix \texttt{make} utility was
developed to keep track of the dependencies and to make the
compilation task more efficient. A free, open-source version of
\texttt{make}, \emph{GNU make}, was developed by Richard Stallman
\cite[]{make}.

Creating data analysis workflows has a lot in common with compiling
programs: input data files enter the workflow and are getting
transformed into other files, generating intermediate outputs and,
finally, the resulting figures or other outputs. When processing is
done in stages and then certain parameters are ge

When the concept of reproducible research was pioneered by Jon
Claerbout at the Stanford Exploration Project (SEP), the tool of
choice for handling processing flows was an old dialect of
\texttt{make} called \texttt{cake}
\cite[]{Nichols.sep.61.341,Claerbout.sep.67.145,Claerbout.sep.73.451,Claerbout.sep.77.427}. Later
SEP converted to using \emph{GNU make} \cite{Schwab.sep.89.217}.

SCons offers a comparable functionality but with a superior design
\cite[]{dubois,scons,ieee}.  The SCons initial design won the Software
Carpentry competition sponsored by Los Alamos National Laboratory in
2000 in the category of ``a dependency management tool to replace
make''.  SCons is implemented as a Python script, its "configuration
files" (\texttt{SConstruct} files) are also Python scripts. Madagascar
uses SCons to compile software, to manage data processing flowing, and
to assemble reproducible documents.

\subsection{Using SCons for compilation}

SCons was designed primarily for compiling software code. An \texttt{SConstruct} file for compilation may look like

\definecolor{frame}{rgb}{0.905,0.905,0.905}
\lstset{language=Python,backgroundcolor=\color{frame},showstringspaces=false,numbers=left,numberstyle=\tiny}

\begin{lstlisting}
env = Environment()
env.Append(CPPFLAGS=['-Wall','-g'])
env.Program('hello',['hello.c', 'main.c'])
\end{lstlisting}

and produce something like

\begin{verbatim}
bash$ scons -Q
gcc -o hello.o -c -Wall -g hello.c
gcc -o main.o -c -Wall -g main.c
gcc -o hello hello.o main.o
\end{verbatim}
to compile the hello program from the source files \texttt{hello.c} and \texttt{main.c}.

Madagascar uses SCons to compile its programs from the source. The more frequent usage, however, comes from adopting SCons to manage data processing flows.

\section{Managing data processing flows using \texttt{rsf.proj}}

The \texttt{rsf.proj} module provides SCons rules for Madagascar data processing workflows. An example \texttt{SConstruct} file is shown below and can be found in \href{http://ahay.org/RSF/book/bei/sg/denmark.html}{bei/sg/denmark} directory under \texttt{\$RSFSRC/book}.

\begin{lstlisting}
from rsf.proj import *

Fetch('wz.35.H','wz')

Flow('wind','wz.35.H',
     'dd form=native | window n1=400 j1=2 | smooth rect1=3')
Plot('wind','pow pow1=2 | grey')

Flow('mute','wind','mutter v0=0.31 half=n')
Plot('mute','pow pow1=2 | grey')

Result('denmark','wind mute','SideBySideAniso')

End()
\end{lstlisting}

Note that \texttt{SConstruct} by itself does not do any job other than setting rules for building different targets. The targets get built when one executes \texttt{scons} on the command line. Running \texttt{scons} produces

\begin{verbatim}
bash$ scons
scons: Reading SConscript files ...
scons: done reading SConscript files.
scons: Building targets ...
retrieve(["wz.35.H"], [])
< wz.35.H /RSF/bin/sfdd form=native | /RSF/bin/sfwindow n1=400 j1=2 | \
/RSF/bin/sfsmooth rect1=3 > wind.rsf
< wind.rsf /RSF/bin/sfpow pow1=2 | /RSF/bin/sfgrey > wind.vpl
< wind.rsf /RSF/bin/sfmutter v0=0.31 half=n > mute.rsf
< mute.rsf /RSF/bin/sfpow pow1=2 | /RSF/bin/sfgrey > mute.vpl
/RSF/bin/vppen yscale=2 vpstyle=n gridnum=2,1 wind.vpl mute.vpl > Fig/denmark.vpl
scons: done building targets.
\end{verbatim}

Obviously, one could also run similar commands with a shell script. What makes SCons convenient is the way it behaves when we make changes in the input files or in the script. Let us change, for example, the mute velocity parameter in the second \texttt{Flow} command. You can do that with an editor or on the command line as

\begin{verbatim}
bash$ sed -i s/v0=0.31/v0=0.32/ SConstruct
\end{verbatim}

Now let us run \texttt{scons} again

\begin{verbatim}
bash$ scons -Q
< wind.rsf /RSF/bin/sfmutter v0=0.32 half=n > mute.rsf
< mute.rsf /RSF/bin/sfpow pow1=2 | /home/fomels/RSF/bin/sfgrey > mute.vpl
/RSF/bin/vppen yscale=2 vpstyle=n gridnum=2,1 wind.vpl mute.vpl > Fig/denmark.vpl
\end{verbatim}

We can see that \texttt{scons} executes only the parts of the data
processing flow that were affected by the change. By keeping track of
dependencies, SCons makes it easier to modify existing workflows
without the need to rerun everything after each change.

\subsection{SConstruct commands}

The four basic commands defined in \texttt{rsf.proj} are \texttt{Fetch}, \texttt{Flow}, \texttt{Plot}, \texttt{Result}, and \texttt{End}.

\subsubsection{Fetch(file[s],directory,[options])}

\texttt{Fetch} defines a rule for downloading files from a data server.

In the example above, the command
\begin{lstlisting}
Fetch('wz.35.H','wz')
\end{lstlisting}
signifies that the file \texttt{wz.35.H} will be downloaded from the default Madagascar data server, where it is stored in the \texttt{wz} directory.

The optional parameters in \texttt{Fetch} are summarized in Table~\ref{tbl:fetch}.

\tabl{fetch}{Optional parameters for \texttt{Fetch}}{
  \begin{tabular}{l|l|l} \hline
    parameter & meaning & default \\ \hline
    server & URL of the data server & \url{https://reproducibility.org} \\
    top & top directory on the server & \texttt{data} \\
    usedatapath & if download to the data path & \texttt{True} \\ \hline 
    \end{tabular}
  }

Explain server=local...

\subsubsection{Flow(target[s],source[s],command,[options])}
\subsubsection{Plot(target,[source],command,[options])}
\subsubsection{Result(target,[source],command,[options])}
\subsubsection{End()}

\subsection{Seismic Unix processing using  \texttt{rsf.suproj}}

\section{Creating documents using \texttt{rsf.tex}}

\section{Creating books and reports using \texttt{rsf.book}}

\bibliographystyle{seg}
\bibliography{scons,SEP2}
