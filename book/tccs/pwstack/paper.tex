\published{SEG Expanded Abstracts, 4069-4074, (2016)}

\title{High-resolution recursive stacking using plane-wave construction}

\author{Kelly Regimbal$^{*1}$ and Sergey Fomel$^{1}$
\\
$^1$The University of Texas at Austin}

\maketitle

\lefthead{Regimbal and Fomel}
\righthead{PWC stacking}
\footer{TCCS}

\begin{abstract}
We propose an approach to normal moveout (NMO) stacking, which eliminates the effects of ``NMO stretch" and 
restores a wider frequency band by replacing conventional stacking with a regularized inversion to zero offset. 
The resulting stack is a model that best fits the data using additional constraints imposed by 
shaping regularization. We introduce a recursive stacking scheme
using plane-wave construction in the backward operator of shaping regularization to achieve a higher resolution stack. 
The advantage of using recursive stacking along local slopes in the application to NMO 
and stack is that it avoids ``stretching effects" caused by NMO correction and is insensitive to 
non-hyperbolic moveout in the data. Numerical tests 
demonstrate the algorithm's ability to attain a higher frequency 
stack with a denser temporal sampling interval compared to those of the conventional stack and to
minimize stretching effects caused by NMO correction. We apply this method to a 2-D 
field dataset from the North Sea and achieve 
noticeable resolution improvements in the stacked section compared with that of conventional NMO and stack. 
\end{abstract}

\section{Introduction}
In seismic data processing, common midpoint (CMP) stacking is one of the most fundamental processes
that combines NMO-corrected traces across a CMP gather to produce a single trace with a 
higher signal-to-noise ratio \cite[]{rashed}. 
Many problems arise with the assumptions and principles that set the foundation for conventional CMP stacking. 
Traditional stacking assumes that the NMO-corrected gather has perfectly aligned seismic reflections \cite[]{yilmaz}. 
However, NMO correction is an approximation that assumes the travel-time as a function of offset follows a 
hyperbolic trajectory in a CMP gather, which may fail in common geologic settings that involve 
velocity variations or anisotropy. NMO correction also causes undesirable distortions of signals on a seismic trace 
known as ``NMO stretch'', which lowers the frequency content of the corrected reflection event at far offsets \cite[]{claerbout3}.
This violates the assumption of a uniform distribution of phase and frequency of 
seismic reflections across the corrected gather. Common procedures to eliminate this stretching effect
involve muting samples with severe distortions. This causes a decrease in fold and may destroy useful 
far-offset information essential for amplitude variation with offset (AVO) analysis \cite[]{swan}. Inaccuracy in stretch
muting with residual ``stretching'' effects may also produce a lower-amplitude and lower-resolution stack \cite[]{miller}.  

Several algorithms were developed to improve CMP stacking and enhance resolution 
of stacked sections by reducing stretching effects. 
\cite{claerbout} described inverse NMO stack, which recasts 
NMO correction and stacking as an inversion process in the constant velocity case. This approach 
combines conventional NMO and stack into one step by solving a set of simultaneous equations using
iterative least-squares optimization. \cite{sun} extended Claerbout's idea to the case of depth-variable 
velocity. The inverse NMO stack operator applied depends on hyperbolic moveout relation and can 
be employed to remove non-hyperbolic events and random noise. \cite{trickett} uses a variation of 
Claerbout's inverse NMO stack in his stretch-free stacking method to avoid ``NMO stretch". 
Trickett's results tend to be higher frequency but noisier 
than a conventional stack. Multiple other algorithms have been proposed that aim to reduce 
NMO stretching effects \cite[]{byun,hicks,hilterman,rupert,perroud,masoomzadeh,zhang,kazemi}.
\cite{wisecup} introduced random sample interval imaging (RSI$^2$), which
maps the CMP gather into the ``after NMO space'' using the exact moveout times and no 
interpolation. The NMO-corrected values are collected in the stack, rather than 
summed, where the input sample values are mapped to their correct time values in the stack. 
\cite{shatilo} proposed a constant NMO correction strategy, which applies a constant NMO shift 
within a finite time interval that is equal to the wavelet length of a trace. This approach eliminates wavelet 
stretch and preserves higher frequencies than the conventional method, resulting in a higher resolution stack. However, 
samples that exist in overlapping time windows are used twice during the correction, resulting in 
an amplitude distortion. \cite{stark} discussed the idea of signal recovery beyond the conventional Nyquist 
frequency using an approach similar to the RSI$^2$ algorithm. The method proposed is an output-driven process, 
where the stack is defined as a merge trace and has a potentially higher sampling rate than the input 
traces. Using this approach, the final stacked sections are not necessarily limited to the data-collected Nyquist frequencies. 
More recently, \cite{ma} proposed a stacking technique based 
on a sparse inversion algorithm that computes the stack directly from a CMP gather by solving an optimization 
problem using principles of compressive sensing. This method eliminates the stretch effect of conventional 
CMP stacking and improves resolution in the stacked section. \cite{silva} introduced a recursive stacking approach using local slopes to compute a stack without stretching effects. In our previous work \cite[]{regimbal}, we proposed a method that
computed NMO and stack in an iterative fashion using shaping regularization to achieve a higher resolution 
stack that avoids the effects of ``NMO stretch". 

In this paper, we extend the method of shaping NMO stack \cite[]{regimbal} further by introducing recursive stacking
using plane-wave construction (PWC) \cite[]{fomel7} in the backward operator of the shaping regularization scheme \cite[]{fomel}.
PWC stacking is equivalent to computing the zero scale of the seislet transform \cite[]{fomel3}.
Shaping regularization implies a mapping of the input 
model to a space of acceptable models. The shaping operator is integrated in an iterative inversion 
algorithm and provides explicit control on the estimation result. We start by reviewing shaping regularization 
in the context of NMO and stack and define the operators used in recursive PWC stacking. 
We test this approach on synthetic examples to demonstrate the algorithm's ability to minimize 
stretching effects and improve resolution. We then apply this method to a 2-D field dataset from 
the North Sea and achieve noticeable resolution improvements in the stacked section in comparison 
with conventional NMO stack.

\section{Method}
\inputdir{.}

In geophysical estimation problems, regularization is 
used to solve ill-posed problems by providing additional constraints on the estimated model.
Shaping regularization \cite[]{fomel,fomel2} implies a mapping of the input model $\mathbf{m}$ to the space of acceptable functions. The mapping is controlled by the shaping operator $\mathbf{S_m}$. In the linear case, the solution of the estimation problem using shaping regularization is defined as:
\begin{equation}
\label{eq:inv}
\mathbf{\hat{m}=[I+S_m(BF-I)]^{-1}S_mBd},
\end{equation}
where $\mathbf{F}$ is the forward operator, $\mathbf{B}$ is the backward operator, and $\mathbf{d}$
is the data. We implement the Generalized Minimum Residual (GMRES) algorithm \cite[]{saad} to perform the linear inversion 
in equation~\ref{eq:inv}.

Shaping regularization in the application to NMO stack utilizes signal from different offsets
to reconstruct a high resolution stack. The model $\mathbf{m}$ is in this case a seismic trace at zero-offset and the 
data $\mathbf{d}$ is a CMP gather. We define the linear operators used in the shaping regularization 
scheme as:
\begin{itemize}
\item{$\mathbf{F}$ (forward operator) applies predictive painting \cite[]{fomel6} to spread information using a known dip field 
	and then subsamples in time.}
\item{$\mathbf{B}$ (backward operator) interpolates the data in time to a denser grid and stacks in a recursive fashion using local slopes.}
\item{$\mathbf{S_m}$ (shaping operator) is a bandpass filter that controls frequency content.}
\end{itemize}

% Recursive stacking
We implement PWC stacking in the backward operator of shaping regularization which follows
local slopes of a CMP gather. The key idea of this stacking procedure is
to start at the farthest offset trace of the gather and make a local slope prediction of the preceding trace using 
PWC (Figure~\ref{fig:schem5}a). The partially corrected trace is then stacked with the uncorrected neighbor, 
which is the input for the next local prediction (Figure~\ref{fig:schem5}b). The process is repeated in the 
offset direction until the zero offset trace is reconstructed (Figure~\ref{fig:schem5}c). 
This recursive stacking approach results in higher resolution stacks compared 
to conventional NMO and stack. The procedure is equivalent to computing the zero scale of the seislet transform 
\cite[]{fomel3}. Advantages of PWC stacking include eliminating 
the effects of ``NMO stretch" as well as the problem of non-hyperbolic moveout. 
The approximate inverse of PWC stacking is defined by predictive painting \cite[]{fomel6}. 
This algorithm is comprised of two main steps, namely estimating 
local slopes of seismic events using plane-wave destruction (PWD) \cite[]{fomel5} and spreading information from a seed 
trace inside a volume. In this application, we use the updated model to spread information 
across the CMP gather using the estimated dip field.

\plot{schem5}{width=0.7\textwidth}{Schematic of the PWC stacking algorithm. (a) Stack far offset trace T$_2$ with neighboring trace T$_1$, 
	(b) stack updated trace T'$_1$ with neighboring trace T$_0$, (c) final accumulated stack.} 

In PWC stacking, each seismic trace is predicted from its neighbors that are shifted 
along the event slopes. Slopes are estimated by PWD, which minimizes the prediction error to estimate optimal slopes.
PWD can be sensitive to conflicting slopes at far offsets of a CMP gather when the dip is large and 
cause PWC stacking to fail in characterizing an optimal stack.
To account for this, we first apply a constant velocity NMO correction to the CMP gather, which results in 
smoothly varying slopes without crossing events. 
We then estimate the moveout $t(x)$ of the corrected seismic events at offset $x$ as follows:
\begin{equation} 
\label{eq:NMO2}
t(x) = \sqrt{t_{0}^{2}+\frac{x^{2}}{v_{0}^{2}} + x^{2} \left(\frac{1}{v^{2}} - \frac{1}{v_{0}^{2}}\right)},
\end{equation}
where $t_0$ is the zero offset travel-time, $v$ is the NMO velocity estimated by a conventional method 
and $v_0$ is a constant velocity.
Adding the correction factor due to the constant velocity NMO correction from equation~\ref{eq:NMO2}, the 
dip field becomes:
\begin{equation} 
\label{eq:dip2}
p={\frac{x}{t}}\left(\frac{1}{v^{2}}-\frac{1}{v_{0}^{2}}\right).
\end{equation}
We use this estimated dip as the initial model for PWD. 
This dip estimation scheme follows the velocity-dependent formulation of the seislet transform \cite[]{liu}
and provides us with a better estimation of the dip field for CMP gathers with large dipping events and conflicting slopes at 
far offsets. We implement this dip estimation method to compute the PWC stack in an iterative 
fashion while using shaping regularization (equation~\ref{eq:inv}) to yield a high resolution stack.

\section{Examples}
\inputdir{synseis}
In our first experiment, we generated a synthetic trace with a sampling interval of 1 ms and 
used it as a reference trace. This trace was then inverse NMO corrected and subsampled to 4 ms 
to produce a CMP gather (Figure~\ref{fig:cmp2}), which is the input data for PWC stack and conventional NMO and stack.
The result of applying a constant velocity NMO correction is displayed in Figure~\ref{fig:nmo0} and is used to compute
the dip field.

\multiplot{2}{cmp2,nmo0}{width=0.45\textwidth}{(a) Synthetic CMP gather with 4-ms sampling interval and (b) constant velocity NMO-corrected gather to separate crossing events at far offsets.}
%\plot{nmodip}{width=0.45\textwidth}{(a) Synthetic CMP gather with 4 ms sampling interval and (b) constant velocity NMO corrected gather to separate crossing events at far offsets.} 

The convergence of the GMRES algorithm in this example required only 4 iterations to achieve a misfit tolerance 
of $10^{-5}$. The estimated PWC stack is shown in Figure~\ref{fig:niter} as a function of iteration, where iteration 0 is the initial model
and iteration 4 is the final estimation result using PWC stack. The conventional stack
results in lower amplitude and lower frequency content, while the PWC stack achieves results similar to the reference trace (Figure~\ref{fig:niter}).
We next compare the frequency content of the PWC stack with the conventional stack (Figure~\ref{fig:spec5}) and the 
reference trace (Figure~\ref{fig:spec6}). The conventional stack fails to recover useful frequencies ranging 
from 110 Hz to 175 Hz, whereas the PWC stack contains frequencies up to 175 Hz. Using inversion, we 
accurately preserve the true amplitude scale and spectrum of the reference trace.
One notable observation is the high frequency information recovered beyond
the Nyquist frequency of the input data (125 Hz). \cite{ronen2} justifies the possibility of such recovery with the
idea that a signal can be recovered with the combination of different aliased sequences. Since different offsets
correspond to different propagation paths, we obtain different information from each offset. Therefore,
the combination of all offsets allows us to reconstruct a higher frequency signal that is not limited
by the Nyquist frequency of the input data. In this simple synthetic example, by implementing shaping 
regularization, we accurately preserve information in the recovered zero-offset trace with a 1-ms sampling 
interval by using input data with only a 4-ms sampling interval. 

% Synthetic results

\plot{niter}{width=0.7\textwidth}{Estimated PWC stack as function of iteration using shaping regularization 
compared with conventional NMO stack and the reference trace.}

\multiplot{3}{spec5,spec6,speclow}{width=0.3\textwidth}{ Spectral comparison of the PWC stack (dashed magenta) with (a) conventional NMO and stack (solid blue) and (b) the reference trace (solid blue) with a 1-ms sampling interval. (c) PWC stack (dashed magenta) recovers lower frequencies in comparison with the conventional stack (dot dash black) and is consistent with the reference trace (solid blue).} 

Low frequencies play an important role in seismic inversion for velocity and impedance models \cite[]{kroode}.
We next evaluate the algorithm's ability to recover low frequency information in the 
estimated stack. We apply a low cut filter to the synthetic CMP gather to remove all of the useful low frequencies below 25 Hz.
A spectral comparison of the resulting PWC stack to conventional NMO stack
and the reference trace is shown in Figure~\ref{fig:speclow}. We zoom in to the low end of the frequency spectrum
to see how well each method does in comparison to the reference trace. Conventional NMO stack fails
to recover frequencies below 25 Hz, while PWC stack recovers more low frequency information that is consistent with the reference trace. 

To demonstrate how PWC stack reduces the effects of ``NMO stretch'', we apply this method to 
each trace of the CMP gather, setting other traces to zero. After repeating this process 
for all traces in the gather, we concatenate the output PWC stacks to extract the effective NMO. 
Figure~\ref{fig:nmo3} displays the conventional NMO-corrected gather with a stretch mute applied, 
where far offset information is lost. Figures~\ref{fig:nmo2} and ~\ref{fig:nsnmo} compare  
the NMO-corrected gather before stretch muting to the effective NMO-corrected gather using PWC stacking. 
The results indicate that the stretching effects that are prominent at far offsets and early times in the NMO-corrected gather 
without a stretch mute are effectively reduced by implementing PWC stack.

\multiplot{3}{nmo3,nmo2,nsnmo}{width=0.3\textwidth}{(a) Conventional NMO correction, (b) NMO correction without stretch muting and (c) effective NMO using PWC stack.}

We next apply this method to a 2-D field dataset from the North Sea and compare the results to conventional NMO stack. 
This dataset has 1,000 CMP locations and 800 samples per trace with a sampling interval of 4 ms. 
We use the 4-ms data as a reference for testing the accuracy of our method and subsample the data to 8 ms 
to provide the input data for PWC stack and conventional NMO stack. The shaping operator used is a bandpass filter 
ranging from 2 Hz to 100 Hz, which was designed based on the frequency content of the data. 
The resulting stacked sections are displayed in Figure~\ref{fig:istack8,ungmres2}. 
PWC stack recovers significantly higher frequencies compared to using conventional NMO stack. 
Thin layers that are unclear in the conventional stacked section become clearly resolved 
in the PWC stacked section. Overall, events become more continuous and coherent throughout the section 
using PWC stack, and resolution is noticeably improved. 

\inputdir{elf}

\multiplot{2}{istack8,ungmres2}{width=0.8\textwidth}{NMO and stack using (a) conventional method and (b) shaping regularization using
plane-wave construction.}

\section{Conclusions}
Conventional NMO stack may result in lower resolution stacked sections due to distortions caused 
by NMO correction and stretch muting. Treating the process of NMO and stack using regularized inversion 
allows us to compute an optimal stack with higher frequency content. Low frequency content also plays
an important role in seismic data processing and imaging. As demonstrated by our numerical examples, 
PWC stack has the ability to recover both higher and lower frequencies compared to conventional NMO and stack.
By implementing PWC stack, we gain resolution by utilizing signal from different offsets 
and minimizing stretching effects.
The final stacked section has improved bandwidth and higher resolution, which may aid in interpretation 
and inversion of small-scale features such as thin layers and diffractions.   

\section{Acknowledgments}
We would like to thank the Texas Consortium for Computational
Seismology (TCCS) members for helpful discussions
and TCCS sponsors for their financial support.

\onecolumn
\bibliographystyle{seg}
\bibliography{seg2016}
